\documentclass[a4paper,11pt]{article}
\usepackage[utf8]{inputenc}
\usepackage[T1]{fontenc}
\usepackage{lmodern}

\usepackage{amsthm,amsmath,amsfonts,amssymb,bbm,mathrsfs,stmaryrd}
\usepackage{mathtools}
\usepackage{enumitem}
\usepackage{url}
\usepackage{dsfont}
\usepackage{appendix}
\usepackage{amsthm}
\usepackage[dvipsnames,svgnames]{xcolor}
\usepackage{graphicx}

\usepackage{fancyhdr,lastpage,titlesec,verbatim,ifthen}

\usepackage[colorlinks=true, linkcolor=black, urlcolor=black, citecolor=black]{hyperref}

\usepackage[french]{babel}

\usepackage{caption,tikz,subfigure}

\usepackage[top=2cm, bottom=2cm, left=2cm, right=2cm]{geometry}

%%%%%%%%%%%%%%%%%%%%%%%%%%%%%%%%%%%%%%%%%%%%%%%%%%%%%%%%%%%%%%%%%%%%%%%

%%%%%%%% Taille de la legende des images %%%%%%%%%%%%%%%%%%%%%%%%%%%%%%%
\renewcommand{\captionfont}{\footnotesize}
\renewcommand{\captionlabelfont}{\footnotesize}

%%%%%%%% Numeration des enumerates en romain et chgt de l'espace %%%%%%%
\setitemize[1]{label=$\rhd$, font=\color{NavyBlue},leftmargin=0.8cm}
\setenumerate[1]{font=\color{NavyBlue},leftmargin=0.8cm}
\setenumerate[2]{font=\color{NavyBlue},leftmargin=0.49cm}
%\setlist[enumerate,1]{label=(\roman*), font = \normalfont,itemsep=4pt,topsep=4pt} 
%\setlist[itemize,1]{label=\textbullet, font = \normalfont,itemsep=4pt,topsep=4pt} 

%%%%%%%% Pas d'espacement supplementaire avant \left et apres \right %%%
%%%%%%%% Note : pour les \Big(, utiliser \Bigl( \Bigr) %%%%%%%%%%%%%%%%%
\let\originalleft\left
\let\originalright\right
\renewcommand{\left}{\mathopen{}\mathclose\bgroup\originalleft}
\renewcommand{\right}{\aftergroup\egroup\originalright}

%%%%%%%%%%%%%%%%%%%%%%%%%%%%%%%%%%%%%%%%%%%%%%%%%%%%%%%%%%%%%%%%%%%%%%%

\newcommand{\N}{\mathbb{N}}
\newcommand{\Z}{\mathbb{Z}}
\newcommand{\Q}{\mathbb{Q}}
\newcommand{\R}{\mathbb{R}}
\newcommand{\C}{\mathbb{C}}
\newcommand{\T}{\mathbb{T}}
\renewcommand{\P}{\mathbb{P}}
\newcommand{\E}{\mathbb{E}}

\newcommand{\1}{\mathbbm{1}}

\newcommand{\cA}{\mathcal{A}}
\newcommand{\cB}{\mathcal{B}}
\newcommand{\cC}{\mathcal{C}}
\newcommand{\cE}{\mathcal{E}}
\newcommand{\cF}{\mathcal{F}}
\newcommand{\cG}{\mathcal{G}}
\newcommand{\cH}{\mathcal{H}}
\newcommand{\cI}{\mathcal{I}}
\newcommand{\cJ}{\mathcal{J}}
\newcommand{\cL}{\mathcal{L}}
\newcommand{\cM}{\mathcal{M}}
\newcommand{\cN}{\mathcal{N}}
\newcommand{\cP}{\mathcal{P}}
\newcommand{\cS}{\mathcal{S}}
\newcommand{\cT}{\mathcal{T}}
\newcommand{\cU}{\mathcal{U}}

\newcommand{\Ec}[1]{\mathbb{E} \left[#1\right]}
\newcommand{\Pp}[1]{\mathbb{P} \left(#1\right)}
\newcommand{\Ppsq}[2]{\mathbb{P} \left(#1\middle|#2\right)}

\newcommand{\e}{\varepsilon}

\newcommand{\ii}{\mathrm{i}}
\DeclareMathOperator{\re}{Re}
\DeclareMathOperator{\im}{Im}
\DeclareMathOperator{\Arg}{Arg}

\newcommand{\diff}{\mathop{}\mathopen{}\mathrm{d}}
\DeclareMathOperator{\Var}{Var}
\DeclareMathOperator{\Cov}{Cov}
\newcommand{\supp}{\mathrm{supp}}

\newcommand{\abs}[1]{\left\lvert#1\right\rvert}
\newcommand{\abso}[1]{\lvert#1\rvert}
\newcommand{\norme}[1]{\left\lVert#1\right\rVert}
\newcommand{\ps}[2]{\langle #1,#2 \rangle}

\newcommand{\petito}[1]{o\mathopen{}\left(#1\right)}
\newcommand{\grandO}[1]{O\mathopen{}\left(#1\right)}

\newcommand\relphantom[1]{\mathrel{\phantom{#1}}}

\newcommand{\NB}[1]{{\color{NavyBlue}#1}}
\newcommand{\DSB}[1]{{\color{DarkSlateBlue}#1}}

%%%%%%%% Theorems styles %%%%%%%%%%%%%%%%%%%%%%%%%%%%%%%%%%%%%%%%%%%%%%
\theoremstyle{plain}
\newtheorem{theorem}{Theorem}[section]
\newtheorem{proposition}[theorem]{Proposition}
\newtheorem{lemma}[theorem]{Lemme}
\newtheorem{corollary}[theorem]{Corollaire}
\newtheorem{conjecture}[theorem]{Conjecture}
\newtheorem{definition}[theorem]{Définition}

\theoremstyle{definition}
\newtheorem{remark}[theorem]{Remarque}
\newtheorem{example}[theorem]{Exemple}
\newtheorem{question}[theorem]{Question}

%%%%%%%% Macros spéciales TD %%%%%%%%%%%%%%%%%%%%%%%%%%%%%%%%%%%%%%%%%%

%%%%%%%%%%%% Changer numérotation des pages %%%%%%%%%%%%%%%%%%%%%%%%%%%%
\pagestyle{fancy}
\cfoot{\thepage/\pageref{LastPage}} %%% numéroter page / total de pages
\renewcommand{\headrulewidth}{0pt} %%% empêcher qu'il y ait une ligne horizontale en haut
%%%%%%%%%%%% Ne pas numéroter les pages %%%%%%%%%%%%%%%%%
%\pagestyle{empty}

%%%%%%%%%%%% Supprimer les alineas %%%%%%%%%%%%%%%%%%%%%%%%%%%%%%%%%%%%%
\setlength{\parindent}{0cm} 

%%%%%%%%%%%% Exercice %%%%%%%%%%%%%%%%%%%%%%%%%%%%%%%%%% 
\newcounter{exo}
\newenvironment{exo}[1][vide]
{\refstepcounter{exo}
	{\noindent \textcolor{DarkSlateBlue}{\textbf{Exercice \theexo.}}}
	\ifthenelse{\equal{#1}{vide}}{}{\textcolor{DarkSlateBlue}{(#1)}}
}{}

%%%%%%%%%%%% Partie %%%%%%%%%%%%%%%%%%%%%%%%%%%%%%%%%%%%
\newcounter{partie}
\newcommand\partie[1]{
	\stepcounter{partie}%
	{\bigskip\large\textbf{\DSB{\thepartie.~#1}}\bigskip}
	}

%%%%%%%%%%%% Separateur entre les exos %%%%%%%%%%%%%%%%%
\newcommand{\separationexos}{
	\bigskip
%	{\centering\hfill\DSB{\rule{0.4\linewidth}{1.2pt}}\hfill}\medskip
	}

%%%%%%%%%%%% Corrige %%%%%%%%%%%%%%%%%%%%%%%%%%%%%%%%%%% 
%\renewenvironment{comment}{\medskip\noindent \textcolor{BrickRed}{\textbf{Corrigé.}}}{}

%%%%%%%%%%%% Titre %%%%%%%%%%%%%%%%%%%%%%%%%%%%%%%%%%%%%%
\newcommand\titre[1]{\ \vspace{-1cm}
	
	\DSB{\rule{\linewidth}{1.2pt}}
	{\small Probabilités et statistiques continues avancées}
	\hfill {\small Université Paul Sabatier}
	
	{\small KMAXPP03}
	\hfill {\small Licence 3, Printemps 2023}\medskip
	\begin{center}
		{\Large\textbf{\DSB{#1}}}\vspace{-.2cm}
	\end{center}
	\DSB{\rule{\linewidth}{1.2pt}}\medskip
}

%%%%%%%%%%%%%%%%%%%%%%%%%%%%%%%%%%%%%%%%%%%%%%%%%%%%%%%%%%%%%%%%%%%%%%%
\begin{document}
%%%%%%%%%%%%%%%%%%%%%%%%%%%%%%%%%%%%%%%%%%%%%%%%%%%%%%%%%%%%%%%%%%%%%%%

\titre{TD 10 -- Révisions et convergence en loi}


%%%%%%%%%%%%%%%%%%%%%%%%%%%%%%%%%%%%%%%%%%%%%%%%%%%
\partie{Révisions}
%%%%%%%%%%%%%%%%%%%%%%%%%%%%%%%%%%%%%%%%%%%%%%%%%%%


\begin{exo}
	Soit $(\lambda_n)_{n\geq 1}$ une suite de réels strictement positifs.
	Soit $(X_n)_{n\geq 1}$ une suite de v.a. indépendantes telle que $X_n$ ait loi exponentielle de paramètre $\lambda_n$.
	\begin{enumerate}
		\item Trouver une condition nécessaire et suffisante sur la suite $(\lambda_n)_{n\geq 1}$ pour que $(X_n)_{n\geq 1}$ converge dans $L^1$ vers 0.
		%%
		\item Trouver une condition nécessaire et suffisante sur la suite $(\lambda_n)_{n\geq 1}$ pour que $(X_n)_{n\geq 1}$ converge en probabilité vers 0.
		%%
		\item Trouver une condition nécessaire et suffisante sur la suite $(\lambda_n)_{n\geq 1}$ pour que $(X_n)_{n\geq 1}$ converge p.s. vers 0.
	\end{enumerate}
\end{exo}

%%%%%%%%%%%%%%%%%%%%%%%%%%%%%%%%%%%%%%%%%%%%%%%%%%%
\partie{Convergence en loi}
%%%%%%%%%%%%%%%%%%%%%%%%%%%%%%%%%%%%%%%%%%%%%%%%%%%

\begin{exo} \label{exo:convergence_mesures_discretes_1}
	\begin{enumerate}
	\item Pour $n \geq 1$, soit $U_n$ une v.a. uniforme sur $\{1,\dots,n\}$.
	Montrer que $(U_n/n)_{n\geq 1}$ converge en loi vers une v.a. uniforme sur $[0,1]$. 
	%%
	\item Soit $\lambda > 0$. 
	Pour $n \geq 1$, soit $U_n$ une v.a de loi géométrique de paramètre $e^{-\lambda/n}$.
	Montrer que $(X_n/n)_{n\geq 1}$ converge en loi vers une v.a. exponentielle de param\`etre $\lambda$.
\end{enumerate}
\end{exo}


%%%%%%
\separationexos
%%%%%%

\begin{exo}[Stabilité par opérations ?]
	Soit $(X_n)_{n\geq 1}$ une suite de v.a. convergeant en loi vers $X$.
	\begin{enumerate}
		\item Soit $f \colon \R\to\R$ continue. Montrer que $(f(X_n))_{n\geq 1}$ converge en loi vers $f(X)$.
		%%
		\item Si $(Y_n)_{n\geq 1}$ est une suite de v.a. convergeant en loi vers $Y$, a-t-on convergence en loi de $(X_n+Y_n)_{n\geq 1}$ ?
	\end{enumerate}
\end{exo}


\begin{comment}
\begin{enumerate}
\item Soit $g \colon \R \to \R$ est continue bornée. Alors $g \circ f$ est continue bornée et donc $ \Ec{g(f(X_n))} \to \Ec{g(f(X))}$. Cela montre la convergence.
%%
\item C'est faux.
\end{enumerate}
\end{comment}


%%%%%%
\separationexos
%%%%%%

\begin{exo}
	Soit $(X_n)_{n\geq 1}$ une suite de v.a. convergeant en loi vers une v.a. constante $X$. 
	Soit $(b_n)_{n\geq 1}$ une suite de réels.
	Montrer que $(X_n + b_n)_{n\geq 1}$ converge en loi si et seulement si $(b_n)_{n\geq 1}$ converge dans~$\R$.
\end{exo}



%%%%%%
\separationexos
%%%%%%

\begin{exo}
	Soit $(X_n)_{n\geq 1}$ une suite de v.a. convergeant en loi vers $X$. 
	A-t-on $\E[X_n] \to \E[X]$ ?
\end{exo}

\begin{comment}
Faux: même la convergence p.s. ne permet pas d'assurer la convergence des espérances.
Par exemple, on peut prendre $(\Omega,\cA,\P)=([0,1],\cB(\R),\lambda)$ et $X_n(t)$ la fonction tente telle que $X_n(0)=0, X_n(1/n)=n,X_n(2/n)=0$. 
Alors $X_n$ converge p.s. vers $0$ mais $\Ec{X_n}=1 \neq \Ec{0}$. 
\end{comment}

%%%%%%%%%%%%%%%%%%%%%%%%%%%%%%%%%%%%%%%%%%%%%%%%%%%
\partie{Compléments}
%%%%%%%%%%%%%%%%%%%%%%%%%%%%%%%%%%%%%%%%%%%%%%%%%%%


\begin{exo}[Discrétisation de mesure]
	\begin{enumerate}
		\item Soit $\mu$ une mesure de probabilité sur $(\R,\cB(\R))$. 
		Pour tout $n\geq 1$, on définit une mesure de probabilité $\mu_n$ sur $(\R,\cB(\R))$ par~:
		\[
		\mu_n \coloneqq \sum_{k\in \Z} \mu([k/n,(k+1)/n[) \delta_{k/n}.
		\]
		On veut montrer que $\mu_n$ converge étroitement vers $\mu$ quand $n\to\infty$. 
		\begin{enumerate}
			\item Soit $X$ une v.a. de loi $\mu$. On pose $X_n=\lfloor nX\rfloor/n$ ($\lfloor x\rfloor$ désigne la partie enti\`ere de $x$).
			Montrer que $X_n$ suit la loi $\mu_n$.
			\item Montrer que $(X_n)_{n\geq 1}$ converge p.s. vers $X$.
			\item Conclure.
		\end{enumerate}
		%%
		\item Reprendre l'exercice \ref{exo:convergence_mesures_discretes_1} en utilisant la question 1.
	\end{enumerate}
\end{exo}

\begin{comment}
\begin{enumerate}
\item Soit $X$ une variable aléatoire réelle définie sur un espace $(\Omega,\cA,\P)$ de loi $\mu$. 
Alors on voit que pour tout $n\geq1$, la variable aléatoire $X_n=\lfloor nX\rfloor/n$ ($\lfloor x\rfloor$ désigne la partie enti\`ere de $x$) suit la loi $\mu_n$. 
Et $X_n\to X$ p.s. 
Donc $X_n\to X$ en loi, ce qui signifie que $\mu_n\to \mu$ étroitement. 
%%
\item On pose $m(\diff x)=e^{-x}\1_{\{x>0\}}\diff x$. 
Alors on vérifie que pour tout $n\geq1$, $m_n$ est la loi de la variable aléatoire $X_n/n$. 
En effet, 
\[
\P(X_n=k)=e^{-k/n}-e^{-(k+1)/n}=\int_{k/n}^{(k+1)/n}e^{-x}\diff x=m_n(\{k/n\}).
\] 
\end{enumerate}
\end{comment}


\end{document}
