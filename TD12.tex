\documentclass[a4paper,11pt]{article}
\usepackage[utf8]{inputenc}
\usepackage[T1]{fontenc}
\usepackage{lmodern}

\usepackage{amsthm,amsmath,amsfonts,amssymb,bbm,mathrsfs,stmaryrd}
\usepackage{mathtools}
\usepackage{enumitem}
\usepackage{url}
\usepackage{dsfont}
\usepackage{appendix}
\usepackage{amsthm}
\usepackage[dvipsnames,svgnames]{xcolor}
\usepackage{graphicx}

\usepackage{fancyhdr,lastpage,titlesec,verbatim,ifthen}

\usepackage[colorlinks=true, linkcolor=black, urlcolor=black, citecolor=black]{hyperref}

\usepackage[french]{babel}

\usepackage{caption,tikz,subfigure}

\usepackage[top=2cm, bottom=2cm, left=2cm, right=2cm]{geometry}

%%%%%%%%%%%%%%%%%%%%%%%%%%%%%%%%%%%%%%%%%%%%%%%%%%%%%%%%%%%%%%%%%%%%%%%

%%%%%%%% Taille de la legende des images %%%%%%%%%%%%%%%%%%%%%%%%%%%%%%%
\renewcommand{\captionfont}{\footnotesize}
\renewcommand{\captionlabelfont}{\footnotesize}

%%%%%%%% Numeration des enumerates en romain et chgt de l'espace %%%%%%%
\setitemize[1]{label=$\rhd$, font=\color{NavyBlue},leftmargin=0.8cm}
\setenumerate[1]{font=\color{NavyBlue},leftmargin=0.8cm}
\setenumerate[2]{font=\color{NavyBlue},leftmargin=0.49cm}
%\setlist[enumerate,1]{label=(\roman*), font = \normalfont,itemsep=4pt,topsep=4pt} 
%\setlist[itemize,1]{label=\textbullet, font = \normalfont,itemsep=4pt,topsep=4pt} 

%%%%%%%% Pas d'espacement supplementaire avant \left et apres \right %%%
%%%%%%%% Note : pour les \Big(, utiliser \Bigl( \Bigr) %%%%%%%%%%%%%%%%%
\let\originalleft\left
\let\originalright\right
\renewcommand{\left}{\mathopen{}\mathclose\bgroup\originalleft}
\renewcommand{\right}{\aftergroup\egroup\originalright}

%%%%%%%%%%%%%%%%%%%%%%%%%%%%%%%%%%%%%%%%%%%%%%%%%%%%%%%%%%%%%%%%%%%%%%%

\newcommand{\N}{\mathbb{N}}
\newcommand{\Z}{\mathbb{Z}}
\newcommand{\Q}{\mathbb{Q}}
\newcommand{\R}{\mathbb{R}}
\newcommand{\C}{\mathbb{C}}
\newcommand{\T}{\mathbb{T}}
\renewcommand{\P}{\mathbb{P}}
\newcommand{\E}{\mathbb{E}}

\newcommand{\1}{\mathbbm{1}}

\newcommand{\cA}{\mathcal{A}}
\newcommand{\cB}{\mathcal{B}}
\newcommand{\cC}{\mathcal{C}}
\newcommand{\cE}{\mathcal{E}}
\newcommand{\cF}{\mathcal{F}}
\newcommand{\cG}{\mathcal{G}}
\newcommand{\cH}{\mathcal{H}}
\newcommand{\cI}{\mathcal{I}}
\newcommand{\cJ}{\mathcal{J}}
\newcommand{\cL}{\mathcal{L}}
\newcommand{\cM}{\mathcal{M}}
\newcommand{\cN}{\mathcal{N}}
\newcommand{\cP}{\mathcal{P}}
\newcommand{\cS}{\mathcal{S}}
\newcommand{\cT}{\mathcal{T}}
\newcommand{\cU}{\mathcal{U}}

\newcommand{\Ec}[1]{\mathbb{E} \left[#1\right]}
\newcommand{\Pp}[1]{\mathbb{P} \left(#1\right)}
\newcommand{\Ppsq}[2]{\mathbb{P} \left(#1\middle|#2\right)}

\newcommand{\e}{\varepsilon}

\newcommand{\ii}{\mathrm{i}}
\DeclareMathOperator{\re}{Re}
\DeclareMathOperator{\im}{Im}
\DeclareMathOperator{\Arg}{Arg}

\newcommand{\diff}{\mathop{}\mathopen{}\mathrm{d}}
\DeclareMathOperator{\Var}{Var}
\DeclareMathOperator{\Cov}{Cov}
\newcommand{\supp}{\mathrm{supp}}

\newcommand{\abs}[1]{\left\lvert#1\right\rvert}
\newcommand{\abso}[1]{\lvert#1\rvert}
\newcommand{\norme}[1]{\left\lVert#1\right\rVert}
\newcommand{\ps}[2]{\langle #1,#2 \rangle}

\newcommand{\petito}[1]{o\mathopen{}\left(#1\right)}
\newcommand{\grandO}[1]{O\mathopen{}\left(#1\right)}

\newcommand\relphantom[1]{\mathrel{\phantom{#1}}}

\newcommand{\NB}[1]{{\color{NavyBlue}#1}}
\newcommand{\DSB}[1]{{\color{DarkSlateBlue}#1}}

%%%%%%%% Theorems styles %%%%%%%%%%%%%%%%%%%%%%%%%%%%%%%%%%%%%%%%%%%%%%
\theoremstyle{plain}
\newtheorem{theorem}{Theorem}[section]
\newtheorem{proposition}[theorem]{Proposition}
\newtheorem{lemma}[theorem]{Lemme}
\newtheorem{corollary}[theorem]{Corollaire}
\newtheorem{conjecture}[theorem]{Conjecture}
\newtheorem{definition}[theorem]{Définition}

\theoremstyle{definition}
\newtheorem{remark}[theorem]{Remarque}
\newtheorem{example}[theorem]{Exemple}
\newtheorem{question}[theorem]{Question}

%%%%%%%% Macros spéciales TD %%%%%%%%%%%%%%%%%%%%%%%%%%%%%%%%%%%%%%%%%%

%%%%%%%%%%%% Changer numérotation des pages %%%%%%%%%%%%%%%%%%%%%%%%%%%%
\pagestyle{fancy}
\cfoot{\thepage/\pageref{LastPage}} %%% numéroter page / total de pages
\renewcommand{\headrulewidth}{0pt} %%% empêcher qu'il y ait une ligne horizontale en haut
%%%%%%%%%%%% Ne pas numéroter les pages %%%%%%%%%%%%%%%%%
%\pagestyle{empty}

%%%%%%%%%%%% Supprimer les alineas %%%%%%%%%%%%%%%%%%%%%%%%%%%%%%%%%%%%%
\setlength{\parindent}{0cm} 

%%%%%%%%%%%% Exercice %%%%%%%%%%%%%%%%%%%%%%%%%%%%%%%%%% 
\newcounter{exo}
\newenvironment{exo}[1][vide]
{\refstepcounter{exo}
	{\noindent \textcolor{DarkSlateBlue}{\textbf{Exercice \theexo.}}}
	\ifthenelse{\equal{#1}{vide}}{}{\textcolor{DarkSlateBlue}{(#1)}}
}{}

%%%%%%%%%%%% Partie %%%%%%%%%%%%%%%%%%%%%%%%%%%%%%%%%%%%
\newcounter{partie}
\newcommand\partie[1]{
	\stepcounter{partie}%
	{\bigskip\large\textbf{\DSB{\thepartie.~#1}}\bigskip}
	}

%%%%%%%%%%%% Separateur entre les exos %%%%%%%%%%%%%%%%%
\newcommand{\separationexos}{
	\bigskip
%	{\centering\hfill\DSB{\rule{0.4\linewidth}{1.2pt}}\hfill}\medskip
	}

%%%%%%%%%%%% Corrige %%%%%%%%%%%%%%%%%%%%%%%%%%%%%%%%%%% 
%\renewenvironment{comment}{\medskip\noindent \textcolor{BrickRed}{\textbf{Corrigé.}}}{}

%%%%%%%%%%%% Titre %%%%%%%%%%%%%%%%%%%%%%%%%%%%%%%%%%%%%%
\newcommand\titre[1]{\ \vspace{-1cm}
	
	\DSB{\rule{\linewidth}{1.2pt}}
	{\small Probabilités et statistiques continues avancées}
	\hfill {\small Université Paul Sabatier}
	
	{\small KMAXPP03}
	\hfill {\small Licence 3, Printemps 2023}\medskip
	\begin{center}
		{\Large\textbf{\DSB{#1}}}\vspace{-.2cm}
	\end{center}
	\DSB{\rule{\linewidth}{1.2pt}}\medskip
}

%%%%%%%%%%%%%%%%%%%%%%%%%%%%%%%%%%%%%%%%%%%%%%%%%%%%%%%%%%%%%%%%%%%%%%%
\begin{document}
%%%%%%%%%%%%%%%%%%%%%%%%%%%%%%%%%%%%%%%%%%%%%%%%%%%%%%%%%%%%%%%%%%%%%%%

\titre{TD 12 -- Convergence en loi}



%%%%%%%%%%%%%%%%%%%%%%%%%%%%%%%%%%%%%%%%%%%%%%%%%%%
\partie{Convergence en loi et fonction de répartition}
%%%%%%%%%%%%%%%%%%%%%%%%%%%%%%%%%%%%%%%%%%%%%%%%%%%

\begin{exo}
	Soit $(X_n)_{n\geq 1}$ une suite de v.a. réelles dont les fonctions de répartitions sont données ci-dessous. Dans chaque cas, justifier s'il y a convergence en loi ou non et, si oui, déterminer la loi de la limite.
	\begin{enumerate}
		\item Pour tout $a \in \R$, $F_{X_n}(a) = [1-\frac{1}{2} (1-\frac{2a}{n})^n] \1_{0 \leq a < n/2} + \1_{a \geq n/2}$.
		\item Pour tout $a \in \R$, $F_{X_n}(a) = \frac{1}{2} \1_{a \geq (-1)^n} + \frac{1}{2} \1_{a \geq 3}$.
		\item Pour tout $a \in \R$, $F_{X_n}(a) = \frac{1}{3} \1_{a \geq 1/n} + \frac{2}{3} a^n \1_{a \in[0,1{[}} + \frac{2}{3} \1_{a \geq 1}$.
		\item Pour tout $a \in \R$, $F_{X_n}(a) = \frac{1}{2} [1-e^{-(a-n)}] \1_{a \geq n} + \frac{1}{2}  \1_{a \geq 0}$.
	\end{enumerate}
\end{exo}

\begin{comment}
\begin{enumerate}
	\item Pour tout $a <0$, on a $F_{X_n}(a) = 0 \to 0$.
	Pour $a \geq 0$, on a
	\[
	F_{X_n}(a) = 1- \frac{1}{2} \left( 1-\frac{2a}{n} \right)^n
	\xrightarrow[n\to\infty]{} 
	1-\frac{1}{2}  e^{-2a}.
	\]
	Donc $F_{X_n}$ converge en tout point vers $F$ définie par
	\[
	F(a) = \left[ 1- \frac{1}{2} e^{-2a} \right] \1_{a\geq 0}.
	\]
	Cette fonction est croissante, continue à droite et tend vers 1 en $+\infty$ et vers 0 en $-\infty$ donc $F$ est la fonction de répartition d'une v.a. $X$.
	Cela montre que $(X_n)_{n\geq 1}$ converge en loi vers $X$.
	Comme $F$ est $\cC^1$ sur $\R\setminus \{0\}$, la loi de $X$ est donnée par 
	\[
	F'(x) \diff x + (F(0)-F(0-)) \delta_0(\diff x)
	= e^{-2x} \1_{x \geq 0} \diff x + \frac{1}{2} \delta_0(\diff x).
	\]
	%%
	\item Pour tout $a \in {[}-1,1{[}$, 
	$F_{X_{2n}}(a) = 0$ et $F_{X_{2n+1}}(a) =\frac{1}{2}$. Il y a donc un ensemble non dénombrable de points $a$ pour lesquels $F_{X_n}(a)$ ne converge pas. Donc $(X_n)_{n\geq 1}$ ne converge pas en loi.
	%%
	\item Pour tout $a \in \R$, on a
	\[
	F_{X_n}(a) 
	\xrightarrow[n \to \infty]{}
	\begin{cases}
	0 & \text{si } a \leq 0, \\
	\frac{1}{3} & \text{si } a \in {]}0,1{[}, \\
	1 & \text{si } a \geq 1.
	\end{cases}
	\]
	On définit
	\[
	F(a) =
	\begin{cases}
	0 & \text{si } a < 0, \\
	\frac{1}{3} & \text{si } a \in {[}0,1{[}, \\
	1 & \text{si } a \geq 1.
	\end{cases}
	\]
	On a changé la valeur en 0 pour que $F$ soit continue à droite.
	En outre, $F$ est croissante et tend vers 1 en $+\infty$ et vers 0 en $-\infty$ donc $F$ est la fonction de répartition d'une v.a. $X$.
	On a convergence simple de $F_{X_n}$ vers $F$ partout sauf en 0 mais c'est un point de discontinuité de $F$.
	Donc $(X_n)_{n\geq 1}$ converge en loi vers $X$.
	Comme $F$ est $\cC^1$ sur $\R\setminus \{0,1\}$, la loi de $X$ est donnée par 
	\[
	\frac{1}{3} \delta_0 + \frac{2}{3} \delta_1.
	\]
	%%
	\item Pour tout $a \in \R$, on a 
	\[
	F_{X_n}(a) \xrightarrow[n \to \infty]{} \frac{1}{2} \1_{a \geq 0}.
	\]
	La fonction $F\colon a \mapsto \frac{1}{2} \1_{a \geq 0}$ ne tend pas vers 1 en $+\infty$.
	Donc $(X_n)_{n\geq 1}$ ne converge pas en loi.
	En effet s'il y avait convergence en loi vers une v.a. $X$ alors $F_{X_n}$ convergerait vers $F_X$ simplement sauf en ses points de discontinuité. On aurait donc $F = F_X$ sauf en un ensemble dénombrable de points. Ce n'est pas compatible avec le fait que $F_X$ tend vers 1 en l'infini.
\end{enumerate}
\end{comment}



%%%%%%
\separationexos
%%%%%%

\begin{exo}
	Soit $(X_n)_{n\geq 1}$ une suite de v.a. réelles convergeant en loi vers $X$.
	On suppose que $X$ n'a pas d'atome, i.e. $\P(X=a)=0$ pour tout $a \in \R$.
	\begin{enumerate}
		\item Montrer que, pour tout $a<b$, $\P(X_n \in {]}a,b]) \to \P(X \in {]}a,b])$ quand $n \to \infty$.
		\item Montrer que, pour tout $a \in \R$, $\P(X_n=a) \to 0$ quand $n \to \infty$.
		\item Montrer que, pour tout intervalle $I \subset \R$,  $\P(X_n \in I) \to \P(X \in I)$ quand $n \to \infty$.
	\end{enumerate}
\end{exo}

\begin{comment}
\begin{enumerate}
	\item Comme $X$ n'a pas d'atome, $F_X$ est continue. Donc, $F_{X_n}$ converge simplement vers $F_X$ en tout point de $\R$.
	Soit $a<b$. On a
	\[
	\P(X_n \in {]}a,b]) 
	= F_{X_n}(b) - F_{X_n}(a) 
	\xrightarrow[n\to\infty]{} 
	F_{X}(b) - F_{X}(a) 
	= \P(X \in {]}a,b]).
	\]
	\item Soit $b \in \R$. Soit $a < b$. On a 
	\[
	\P(X_n=b) \leq \P(X_n \in {]}a,b])	\xrightarrow[n\to\infty]{} F_{X}(b) - F_{X}(a).
	\]
	Soit $\varepsilon > 0$. Comme $F_X$ est continue, on peut choisir $a$ tel que $F_{X}(b) - F_{X}(a) < \varepsilon$. Alors pour tout $n$ à partir d'un certain rang $\P(X_n=b) < \varepsilon$. Cela montre que $\P(X_n=b) \to 0$.
	%%
	\item Commençons par le cas où $I$ est borné, de bornes $a$ et $b$.
	\begin{itemize}
		\item Le cas $I = {]}a,b]$ est déjà fait.
		\item Pour $I = [a,b]$, on a 
		\[
		\P(X_n \in [a,b]) = \P(X_n \in {]}a,b])  + \P(X_n=a)
		\xrightarrow[n\to\infty]{} 
		\P(X \in {]}a,b]) + 0
		= \P(X \in [a,b]),
		\]
		car $\P(X=a) = 0$.
		\item Pour $I = [a,b[$, on utilise le point prédécent avec le fait que $\P(X_n=b) \to 0 = \P(X=b)$.
		\item Pour $I = {]}a,b[$, on utilise le 1er point avec le fait que $\P(X_n=b) \to 0 = \P(X=b)$.
	\end{itemize}
	Considérons maintenant le cas de $I$ non borné.
	\begin{itemize}
		\item Si $I = {]}-\infty,b]$ alors cela découle directement de la convergence des fonctions de répartition. 
		\item On en déduit le cas de $I = {]}-\infty,b[$ en utilisant que $\P(X_n=b) \to 0 = \P(X=b)$.
		\item Pour $I = [b,\infty[$, on utilise que $\P(X_n \in [b,\infty[) = 1 - \P(X_n \in {]}-\infty,b[)$ et le point précédent.
		\item On procède similairement pour $I = {]}b,\infty[$.
		\item Le cas $I = \R$ est immédiat.
	\end{itemize}
\end{enumerate}
\end{comment}



%%%%%%%%%%%%%%%%%%%%%%%%%%%%%%%%%%%%%%%%%%%%%%%%%%%
\partie{Convergence en loi et fonction caractéristique}
%%%%%%%%%%%%%%%%%%%%%%%%%%%%%%%%%%%%%%%%%%%%%%%%%%%


\begin{exo}[Une loi faible par les fonctions caractéristiques] 
	Soit $(X_n)_{n\geq1}$ une suite de v.a. réelles i.i.d telle que $\mathbb{E}[|X_1|]<\infty$. 
	\begin{enumerate}
		\item En utilisant le théorème de Lévy, montrer que 
		\[ 
		\frac{X_1 +\dots+X_n}{n} \xrightarrow[n\to\infty]{\text{loi}} \mathbb{E}[X_1].
		\]
		\item En déduire que
		\[ 
		\frac{X_1 +\dots+X_n}{n} \xrightarrow[n\to\infty]{\P} \mathbb{E}[X_1].
		\]
	\end{enumerate}
\end{exo}

\begin{comment}
\begin{enumerate}
\item L'indépendance des $X_i$ implique que, pour $\xi\in \R$,
\[
\phi_{\frac{X_1+\dots+X_n}{n}}(\xi) 
= \left(\phi_{X_1}\left(\frac{\xi}{n}\right)\right)^{n}.
\] 
Comme $X_1$ a un moment d'ordre $1$ fini, $\phi_{X_1}$ est dérivable en $0$ de dérivée $im$, où $m \coloneqq \mathbb{E}[X_1]$. 
Par Taylor-Young en 0, on a donc
\[
\phi_{X_1}\left(\frac{\xi}{n}\right)
= 1+\frac{i\xi\mathbb{E}[X_1]}{n} + \petito{\frac{1}{n}} 
= e^{im\xi/n} + \petito{\frac{1}{n}}.
\]
Ainsi
\[
	\abs{\phi_{\frac{X_1+\dots+X_n}{n}}(\xi) - e^{im\xi} }
	= \abs{\left(\phi_{X_1}\left(\frac{\xi}{n}\right)\right)^{n} 
	- (e^{im\xi/n})^n }
	\leq n \abs{\phi_{X_1}\left(\frac{\xi}{n}\right) - e^{im\xi/n}}
	\xrightarrow[n\to\infty]{} 0,
\]
par l'inégalité vue en cours : $\abs{z^n-w^n} \leq n \abs{z-w}$ pour $z,w \in \C$ de modules inférieurs à 1.
Ainsi, la fonction caractéristique de $(X_1 +\dots+ X_n)/n$ converge simplement vers la fonction caractéristique de la v.a. constante égale à $m$.
Par le théorème de Lévy, on en déduit que 
\[ \frac{X_1+\dots+X_n}{n} 
\xrightarrow[n\to\infty]{\text{loi}} m=\mathbb{E}[X_1].
\]
%%
\item La convergence en loi vers une constante implique la convergence en probabilité vers cette constante, ce qui conclut.
\end{enumerate}
\end{comment}

%%%%%%%%%%%%%%%%%%%%%%%%%%%%%%%%%%%%%%%%%%%%%%%%%%%
\partie{Convergence en loi de v.a. discrètes}
%%%%%%%%%%%%%%%%%%%%%%%%%%%%%%%%%%%%%%%%%%%%%%%%%%%

\begin{exo}
	Soit $X,X_1,X_2,\dots$ des v.a. à valeurs dans $\Z$. 
	Montrer que $(X_n)_{n\geq 1}$ converge en loi vers $X$ si et seulement si
	\[
	\forall k \in \Z, \qquad 
	\Pp{X_n=k} \xrightarrow[n\to\infty]{} \Pp{X=k}.
	\]
	\emph{Indication.} Pour montrer $(\Rightarrow)$, on pourra utiliser une fonction $f$ qui vaut 1 sur $k$ et 0 sur les autres entiers. Pour montrer $(\Leftarrow)$, on pourra considérer une fonction test $f$ à support compact.
\end{exo}

\begin{comment}
\emph{Sens direct.} Supposons que $(X_n)_{n\geq 1}$ converge en loi vers $X$.
Soit $k \in \Z$. 
On considère une fonction $f \colon \R \to \R$ continue bornée telle que $f(k) = 1$ et $f$ est nulle sur tous les autres entiers (par exemple, $f(x) = \max(1-\abs{x-k},0)$).
Alors
\[
\Ec{f(X_n)} = \sum_{i\in \Z} \P(X_n=i) f(i) = \P(X_n=k)
\]
et de même $\Ec{f(X)} = \P(X=k)$.
Or, par convergence en loi, on a $\Ec{f(X_n)} \to \Ec{f(X)}$ quand $n \to \infty$ et donc $\P(X_n=k) \to \P(X=k)$.

\emph{Sens réciproque.} Supposons que, pour tout $k \in \Z$, on ait $\P(X_n=k) \to \P(X=k)$ quand $n \to \infty$.
Soit $f \colon \R \to \R$ à support compact. Soit $M > 0$ tel que $f(x) = 0$ pour tout $x \notin [-M,M]$.
Alors 
\begin{align*}
\Ec{f(X_n)} & = \sum_{k\in \Z} \P(X_n=k) f(k) 
= \sum_{k\in \Z \cap [-M,M]} \P(X_n=k) f(k) \\
& \xrightarrow[n\to\infty]{} \sum_{k\in \Z \cap [-M,M]} \P(X=k) f(k)
= \Ec{f(X)},
\end{align*}
où la convergence est justifiée car la somme est finie.
Cela implique la convergence en loi par le Théorème \ref{thm:CV_loi_C_infini}.
\end{comment}

%%%%%%
\separationexos
%%%%%%

\begin{exo}
	Soit $(X_n)_{n\geq 1}$ une suite de v.a. telle que $X_n$ ait une loi binomiale de paramètre $(n,p_n)$, où $(p_n)_{n\geq 1}$ est une suite de réels dans $[0,1]$ telle que 
	\[
	n p_n \xrightarrow[n\to\infty]{} \lambda >0.
	\]
	Montrer que $(X_n)_{n\geq 1}$ converge en loi vers une v.a. de loi de Poisson de paramètre $\lambda$.
	
	\emph{Rappel.} La formule de Stirling dit que $n! \sim \sqrt{2 \pi n} (\frac{n}{e})^n$ quand $n \to \infty$.
\end{exo}

\begin{comment}
Soit $k \in \N$ fixé. On a
\begin{align*}
\Pp{X_n = k} 
= \binom{n}{k} p_n^k (1-p_n)^{n-k}
= \frac{n!}{k ! (n-k)!} 
\left( \frac{p_n}{1-p_n} \right)^k 
\exp \left( n \log(1-p_n) \right).
\end{align*}
En utilisant la formule de Stirling, le fait que $\frac{p_n}{1-p_n} \sim \frac{\lambda}{n}$ et que $n \log(1-p_n) \sim -\lambda$, on obtient
\begin{align*}
\Pp{X_n = k} 
& \sim \frac{1}{k!} \cdot 
\frac{\sqrt{2\pi n} (\frac{n}{e})^{n}}{\sqrt{2\pi (n-k)} (\frac{n-k}{e})^{n-k}}
\cdot \left( \frac{\lambda}{n} \right)^k 
\exp \left( - \lambda \right)
\sim \frac{1}{k!} \cdot e^{-k}
\left( \frac{n}{n-k} \right)^{n}
\cdot \lambda^k e^{-\lambda}.
\end{align*}
Finalement, comme $(\frac{n-k}{n})^{n} = \exp( n \log (1-\frac{k}{n})) \to e^{-k}$, on obtient
\[
\Pp{X_n = k} 
\xrightarrow[n \to \infty]{}
\frac{\lambda^k}{k!} e^{-\lambda}. 
\]
Notons aussi que, pour tout $k < 0$, $\Pp{X_n = k}  = 0 \to 0$.
Par l'exercice précédent, cela montre que $(X_n)_{n\geq 1}$ converge en loi vers une v.a. de loi de Poisson de paramètre $\lambda$.
\end{comment}



%%%%%%%%%%%%%%%%%%%%%%%%%%%%%%%%%%%%%%%%%%%%%%%%%%%
\partie{Compléments}
%%%%%%%%%%%%%%%%%%%%%%%%%%%%%%%%%%%%%%%%%%%%%%%%%%%



\begin{exo}[Limite de gaussiennes]
	Soit $(Y_n)_{n\in\N}$ une suite de v.a. réelles de loi $\cN(m_n,\sigma_n^2)$ avec $m_n \in \R$ et $\sigma_n \in \R_+$. 
	Par convention, lorsque $\sigma=0$, la loi $\mathcal{N}(m, \sigma^2)$ est la masse de Dirac $\delta_m$.
	
	\emph{Rappel.} La fonction caractéristique de la loi $\mathcal{N}(m, \sigma^2)$ est $\theta \mapsto \exp(im\theta- \sigma^2 \theta^2/2)$ (on peut noter que cette formule fonctionne aussi dans le cas $\sigma = 0$). 
	\begin{enumerate}
		\item Supposons que $(m_n)_{n\in\N}$ et $(\sigma_n)_{n\in\N}$ convergent vers respectivement $m \in \R$ et $\sigma \in \R_+$. Montrer que $(Y_n)_{n\in\N}$ converge en loi et identifier sa limite.
		%%
		\item Supposons que $(Y_n)_{n\in\N}$ converge en loi.
		\begin{enumerate}
			\item En considérant $\abs{\phi_{Y_n}}$ montrer que $(\sigma_n)_{n\in\N}$ converge vers une limite $\sigma \in \R_+$.
			\item Montrer que $(m_n)_{n\in\N}$ est bornée.
			\item En déduire que $(m_n)_{n\in\N}$ converge vers une limite $m \in \R$
		\end{enumerate}
	\end{enumerate}
\end{exo}

\begin{comment}
\begin{enumerate}
\item Cela découle immédiatement du théor\`eme de Lévy et la loi limite est alors la loi gaussienne $\mathcal{N}(m, \sigma^2)$.
%%
\item Supposons que $(Y_n)_{n\in\N}$ converge en loi.
\begin{enumerate}
\item Le théor\`eme de Lévy garantit que $\exp(im_n t- \sigma_n^2 t^2/2)$ converge pour tout réel $t$ lorsque $n \rightarrow \infty$, et donc que $\exp(- \sigma_{n}^2 t^2/2)$ converge (en prenant le module).
Par positivité de $\sigma_n$, on obtient donc 
\[
\sigma_n \xrightarrow[n\to\infty]{} \sigma \in \R_+ \cup \{\infty\}.
\] 
Par l'absurde, si $\sigma = \infty$, alors on a, pour tout $t > 0$,
\[
\abs{\phi_Y(t)} = \lim_{n\to\infty} e^{-\sigma_n^2 t^2/2} = 0,
\] 
ce qui contredit que $\phi_Y(0)=1$ et que $\phi_Y$ est continue en 0.
Ainsi, on a bien $\sigma \in \R_+$.
%%
\item Comme $e^{-\sigma_n^2 t^2/2}$ converge vers une limite non nulle pour tout $t\in\R$, $e^{im_n t}$ converge pour tout $t\in\R$ lorsque $n \to \infty$. 
%Montrons donc que la suite $(m_n)_{n\in\N}$ est bornée.
On note $D_Y$ l'ensemble des points de discontinuité de $F_Y$.
Soit $A>0$ tel que $A, -A \notin D_Y$. Alors on a
\begin{align*}
\P(Y \in ]-\infty,-A{]}\cup {]}A,\infty])
& = F_Y(-A) + 1- F_Y(A) 
= \lim_{n \to \infty} F_{Y_n}(-A) + 1- F_{Y_n}(A) \\
& = \lim_{n \to \infty} \P(Y_n \in {]}-\infty,-A]\cup {]}A,\infty]).
\end{align*}
Soit $\varepsilon > 0$.
Choisissons $A$ suffisamment grand tel que $\P(\abs{Y} \geq A) <\varepsilon$ et $A, -A \notin D_Y$.
Alors, à partir d'un certain rang, $\P(\abs{Y_n} > A) < \varepsilon$ d'après l'équation ci-dessus.
D'autre part, si $\abs{m_n} > A$, alors
\[
\P(\abs{Y_n} > A)
\geq \P(\abs{Y_n} \geq \abs{m_n})
\geq \left\{
\begin{array}{ll}
\P(Y_n \geq m_n) & \text{si } m_n \geq 0 \\
\P(Y_n \leq m_n) & \text{si } m_n < 0
\end{array}
\right.
= \frac{1}{2},
\]
par symétrie de la gaussienne autour de sa moyenne.
Avec $\varepsilon = 1/2$, comme, à partir d'un certain rang, $\P(\abs{Y_n} > A) < \varepsilon$, on ne peut pas avoir $\abs{m_n} > A$.
Donc $\abs{m_n} \leq A$ à partir d'un certain rang.
%%
\item Comme $(m_n)_{n\in\N}$ est bornée, il suffit de montrer qu'elle a au plus une seule valeur d'adhérence réelle.
Mais si $m$ et $m'$ sont deux valeurs d'adhérence on a $\exp(imt)= \exp(im't)$ pour tout $t \in \R$, ce qui entra\^ine $m=m'$ (en dérivant en $t=0$ par exemple). 
\end{enumerate}
\end{enumerate}

%\emph{Méthode 2}. 
%Comme $e^{-\sigma_n^2 t^2/2}$ converge vers une limite non nulle pour tout $t\in\R$, $e^{im_n t}$ converge pour tout $t\in\R$ lorsque $n \to \infty$. 
%On note $\psi(\xi) = \lim_{n\to\infty} e^{i m_n \xi}$ pour tout $\xi \in \R$. Alors $\psi$ est continue de module 1.
%Si $m_n \neq 0$, on a, pour $a > 0$,
%\[
%\int_0^a e^{im_n\xi} \diff \xi = \frac{e^{i m_n a}-1}{im_n}.
%\]
%D'autre part, par convergence dominée, on a
%\[
%\int_0^a e^{im_n\xi} \diff \xi \xrightarrow[n\to\infty]{} \int_0^a \psi(\xi) \diff \xi.
%\]
%On choisit alors $a>0$ tel que $\int_0^a \psi(\xi) \diff \xi \neq 0$ (ce qui est possible car $\lvert \psi \rvert = 1$ et $\psi$ continue en 0).
%Alors pour $n$ suffisamment grand on a $\int_0^a e^{im_n\xi} \neq 0$ et donc
%\[
%m_n = \frac{e^{i m_n a}-1}{i\int_0^a e^{im_n\xi} \diff \xi},
%\]
%en remarquant que c'est aussi vrai si $m_n = 0$.
%Finalement, on obtient
%\[
%m_n \xrightarrow[n\to\infty]{} \frac{\psi(a) - 1}{i\int_0^a \psi(\xi) \diff \xi}
%\]
%donc on a bien montré la convergence de la suite $(m_n)_{n\in\N}$.
\end{comment}


%%%%%%
\separationexos
%%%%%%



\begin{exo}[Maximum d'exponentielles]
	Soit $(X_n)_{n\geq 1}$ une suite de v.a. i.i.d. de loi exponentielle de paramètre 1. Soit $M_n = \max(X_1,\dots,X_n)$. Montrer que $(M_n-\log n)_{n\geq 1}$ converge en loi et déterminer la limite.
\end{exo}


%%%%%%
\separationexos
%%%%%%



\begin{exo}
	Soit $(X_n)_{n\geq 1}$ une suite de v.a. à valeurs dans $\Z$. On suppose que pour tout $k \in \Z$, $\P(X_n=k)$ converge vers un réel $p_k$ quand $n \to \infty$. À quelle condition sur les $p_k$ a-t-on convergence en loi de $(X_n)_{n\geq 1}$ ?
\end{exo}


%%%%%%
\separationexos
%%%%%%



\begin{exo}
	On reprend la question 2 de l'exercice 2 du TD11.
	Soit $(X_n)_{n\geq 1}$ une suite de v.a. réelles continues telles que
	\[
	p_{X_n}(x) = \frac{1}{2 \sqrt{2\pi}} \left( e^{-x^2/2} + e^{-(x-n)^2/2} \right).
	\]
	En utilisant les fonctions de répartition, montrer que $(X_n)_{n\geq 1}$ ne converge pas.
\end{exo}



%%%%%%
\separationexos
%%%%%%



\begin{exo}[Convergence étroite de mesures finies]
	Soit $\mu,\mu_1,\mu_2,\dots$ des mesures finies sur $\R$. 
	On dit que $(\mu_n)_{n\geq 1}$ converge étroitement vers $\mu$ si, pour toute fonction $f \colon \R \to \R$ continue bornée,
	\[
	\int_\R f \diff \mu_n 
	\xrightarrow[n\to\infty]{} 
	\int_\R f \diff \mu.
	\]
	\begin{enumerate}
		\item Soit $(a_n)_{n\geq 1}$ une suite de réels positifs convergeant vers $a > 0$. Pour $n \geq 1$, soit $\mu_n$ la mesure de Lebesgue sur $[0,a_n]$.
		Montrer que $(\mu_n)_{n\geq 1}$ converge étroitement vers une limite à déterminer.
		%%
		\item Pour $n \geq 1$, soit $\mu_n = \delta_n$. Soit $\mu$ la mesure nulle. Montrer que, pour toute fonction $f \colon \R \to \R$ continue à support compact, $\int_\R f \diff \mu_n \to \int_\R f \diff \mu$,
%		\[
%			\int_\R f \diff \mu_n 
%			\xrightarrow[n\to\infty]{} 
%			\int_\R f \diff \mu,
%		\]
		mais que $(\mu_n)_{n\geq 1}$ ne converge pas étroitement vers $\mu$.
		%%
		\item Soit $\mu,\mu_1,\mu_2,\dots$ des mesures finies sur $\R$. 
		Montrer que $(\mu_n)_{n\geq 1}$ converge étroitement vers $\mu$ si et seulement si $\mu_n(\R) \to \mu(\R)$ et, pour toute fonction $f \colon \R \to \R$ continue à support compact, $\int_\R f \diff \mu_n \to \int_\R f \diff \mu$.
%		\[
%		\int_\R f \diff \mu_n 
%		\xrightarrow[n\to\infty]{} 
%		\int_\R f \diff \mu.
%		\]
	\end{enumerate}
\end{exo}


%%%%%%
\separationexos
%%%%%%


\begin{exo}[Une partie du théorème de Prokhorov sur $\R$] \label{exo:prokhorov}
	Soit $(X_n)_{n\in\N}$ une suite de v.a. réelles. 
	La suite $(X_n)_{n\in\N}$ est dite \emph{tendue} si
	\[
	\forall \varepsilon > 0, \quad \exists M>0, \quad \forall n \in\N, \quad
	P(\abs{X_n} \geq M) \leq \varepsilon.
	\]
	L'objectif de cet exercice est de montrer que, si la suite $(X_n)_{n\in\N}$ est tendue, alors elle admet une sous-suite qui converge en loi. 
	On suppose donc que $(X_n)_{n\in\N}$ est tendue.
	\begin{enumerate}
		\item On note $F_n$ la fonction de répartition de $X_n$. Soit $q \in \Q$.
		Montrer qu'il existe une extractrice $\psi \colon \N \to \N$ telle que $F_{\psi(n)}(q)$ converge vers une limite $G(q)$ quand $n \to \infty$.
		%%
		\item En déduire qu'il existe une extractrice $\varphi \colon \N \to \N$ telle que, pour tout $q \in \Q$, $F_{\varphi(n)}(q)$ converge vers $G(q)$ quand $n \to \infty$.
		%%
		\item On définit, pour $x \in \R$, $F(x) \coloneqq \inf \{G(q) : q \in \Q \, \cap \, ]x,\infty[ \}$.
		Montrer qu'il existe une v.a. réelle $X$ telle que $F$ soit la fonction de répartition de $X$.
		%%
		\item Montrer que $(X_{\varphi(n)})_{n\in\N}$ converge en loi vers $X$.
	\end{enumerate}
\end{exo}


\begin{comment}
\begin{enumerate}
\item Pour $q \in \Q$, la suite $(F_n(q))_{n\in\N}$ est bornée dans $\R$ donc admet une sous-suite convergente.
%%
\item C'est le procédé d'extraction diagonale qui permet de construire une extractrice $\varphi$ commune à tous les $q \in \Q$.
%%
\item Il suffit de montrer que $F \colon \R \to \R$ est croissante, continue à droite et telle que $F(-\infty) = 0$ et $F(+\infty) =1$.
\begin{itemize}
\item Les $F_n$ sont croissantes donc $G$ est croissante sur $\Q$, et ainsi $F$ est croissante.
\item Soit $x \in \R$. Montrons la continuité à droite de $F$ en $x$. On sait déjà que $F(x+)$ existe.
D'autre part, il existe $(q_k)_{k\in\N}$ une suite de rationnels strictement supérieurs à $x$ telle que $G(q_k) \to F(x)$ quand $k \to \infty$.
Alors on a, pour tout $k \in \N$, 
\[
F(x+) \leq F \left( \frac{x+q_k}{2} \right) \leq G(q_k).
\]
En passant à la limite, on obtient $F(x+) \leq F(x)$.
Par croissance on en conclut que $F(x+) = F(x)$.
%%
\item Montrons que $F(-\infty) = 0$ et $F(+\infty) =1$. Comme $G$ est à valeurs dans $[0,1]$, $F$ aussi et donc on a déjà $0 \leq F(-\infty) \leq F(+\infty) \leq 1$.
Soit $\varepsilon>0$. 
Par la tension, il existe $M \in \N$ tel que, pour tout $n \in \N$, 
\[
\P(X_n\in[-M,M]) \geq 1-\varepsilon.
\]
Alors, on a $F_n(M) - F_n(-M) \geq 1-\varepsilon$ et, en passant à la limite $n\to\infty$,
\[
F(+\infty)- F(-\infty) 
\geq G(M)-G(M)
\geq 1-\varepsilon.
\]
Avec $\varepsilon\to0$, on obtient $F(+\infty)- F(-\infty)  \geq 1$ et cela donne le résultat espéré.
\end{itemize}
%%
\item Il suffit de montrer, pour $x \in\R$ point de continuité de $F$, que $F_{\varphi(n)}(x) \to F(x)$ quand $n\to\infty$.
Soit $\varepsilon>0$. 
Il existe $\eta > 0$ tel que, pour tout $y \in [x-\eta,x+\eta]$, $\lvert F(x)-F(y) \rvert \leq \varepsilon$.
Soit $x-\eta < q < x < r < x+\eta$ avec $q,r \in \Q$,
\begin{align*}
\lvert F_{\varphi(n)}(x) - F(x) \rvert
& \leq \lvert F_{\varphi(n)}(x) - F_{\varphi(n)}(r) \rvert 
+ \lvert F_{\varphi(n)}(r) - G(r) \rvert 
+ \lvert G(r) - F(x) \rvert \\
& \leq [F_{\varphi(n)}(r) - F_{\varphi(n)}(q)]
+ \lvert F_{\varphi(n)}(r) - G(r) \rvert 
+ [F(x+\eta) - F(x)]
\end{align*}
et donc 
\begin{align*}
\limsup_{n\to\infty} \, \lvert F_{\varphi(n)}(x) - F(x) \rvert
& \leq [G(r) - G(q)] + [F(x+\eta) - F(x)]
\leq 2 [F(x+\eta) - F(x-\eta)]
\leq 4 \varepsilon.
\end{align*}
Cela montre que $F_{\varphi(n)}(x) \to F(x)$ quand $n\to\infty$.
\end{enumerate}
\end{comment}


%%%%%%%
%\separationexos
%%%%%%%
%
%\begin{exo}[Théorème de Lévy fort] Soit $(X_n)_{n\in\N}$ une suite de variables aléatoires réelles. 
%	On suppose qu'il existe une fonction $\phi \colon \R \rightarrow \C$, continue en $0$, telle que
%	\[
%	\forall t \in \R, \quad \phi_{X_n}(t) \xrightarrow[n\to\infty]{} \phi(t).
%	\]
%	\begin{enumerate}
%		\item Montrer que, pour tout $a>0$, on a
%		\[
%		\Pp{\abs{X_n} > \frac{2}{a}} 
%		\leq \frac{1}{a} \int_{-a}^a \left( 1- \mathrm{Re}( \phi_{X_n}(u)) \right) \diff u
%		= \frac{1}{a} \int_{-a}^a \left( 1-\phi_{X_n}(u) \right) \diff u.
%		\]
%		%%
%		\item Montrer que, pour tout $\varepsilon>0$, il existe $u>0$ et $n_0 \in \N$ tels que, pour tout $n \geq n_0$, on ait
%		\[ 
%		\frac{1}{u} \int_{-u}^u  \abs{1- \phi_{X_n}(t)} \diff t < \varepsilon.
%		\]
%		%%
%		\item En déduire que la suite $(X_n)_{n\in\N}$ est tendue (c'est-\`a-dire que la suite des lois $(P_{X_n})_{n\in\N}$ est tendue).
%		%%
%		\item Montrer que la suite $(X_n)_{n\in\N}$ converge en loi.
%		
%		\emph{Rappel}. \`A l'exercice 9 du TD 12, il a été montré que si une suite de variables aléatoires est tendue, alors elle admet une sous-suite convergeant en loi.
%	\end{enumerate}
%\end{exo}
%
%
%\begin{comment}
%\begin{enumerate}
%\item D'apr\`es le théor\`eme de Fubini--Tonelli,
%\begin{align*}
%\frac{1}{a} \int_{-a}^a (1- \phi_{X_n}(t)) \diff t 
%&= \frac{1}{a}\int_{-a}^a \left( \int_{ \R}(1- \e^{itx}) \P_{X_{n}}(\diff x) \right) \diff t 
%= \frac{1}{a} \int_{\R}\left(\int_{-a}^a(1- \e^{itx}) \diff t \right) \P_{X_{n}}(\diff x) \\
%&= 2 \int_{ \R} \left( 1- \frac{ \sin(ax)}{ax} \right)\P_{X_{n}}(\diff x) \geq  \int_{|x|>2/u} \left(1- \frac{1}{|ax|} \right) \P_{X_{n}}(\diff x)
%\end{align*}
%car pour tout réel $t$ on a $1- \sin(t)/t \geq 0$ et $ 1- \sin(t)/t \geq 1-1/|t|$. Comme
%\[2 \int_{|x|>2/a} \left(1- \frac{1}{|ax|} \right) \P_{X_{n}}(\diff x) \leq 2 \int_{|x|>2/a} \left(1- \frac{1}{2} \right) \P_{X_{n}}(\diff x) = \Pp{ |X_{n}|>2/a},\]
%ceci conclut.
%%%
%\item Comme $|\phi(t)| = \lim_{n \rightarrow \infty} | \phi_{X_n}(t)| \leq 1$ pour tout $t \in \R$, $ \phi(0)=\lim_{n \rightarrow \infty} | \phi_{X_n}(0)|=1$ et que $ \phi$ est continue en $0$, si $\varepsilon > 0$ est fixé, il existe $u>0$ tel que $|1- \phi(t)| < \varepsilon/2$ pour $|t|<u$. 
%Ainsi
%\[
%\frac{1}{u} \int_{-u}^u | 1- \phi(t)| \diff t < \varepsilon.
%\]
%Or $|1- \phi_{X_n}(t)| \leq 2$ et $ | 1- \phi_{X_n}(t)| \rightarrow  | 1- \phi(t)|$ quand $n \rightarrow \infty$. Par convergence dominée, il s'ensuit que 
%\[
%\frac{1}{u} \int_{-u}^u | 1- \phi_{X_n}(t)| \diff t 
%\xrightarrow[n \to \infty]{} 
%\frac{1}{u} \int_{-u}^u | 1- \phi(t)| \diff t 
%< \varepsilon.
%\]
%Le résultat désiré en découle.
%%%
%\item D'apr\`es la premi\`ere question, pour $n \geq n_{0}$, $\Pp{|X_{n}|> \frac{2}{u}}> 2 \varepsilon$. Comme les variables aléatoires $X_{i}$ sont réelles, il existe des réels positifs $a_{1},a_{2}, \ldots, a_{n_{0}-1}$ tels que
%\[
%\Pp{|X_{i}|> a_{i}}< 2 \varepsilon, \qquad 0 \leq i \leq n_{0}-1.
%\]
%En posant $a= \max(2/u,a_{1},a_{2}, \ldots, a_{n_{0}-1})$, on a bien
%\[
%\Pp{|X_{n}|>a}< 2 \varepsilon
%\]
%pour tout entier $n \geq 1$, c'est-à-dire 
%\[
%\Pp{X_n \in [-a,a]} \geq 1-2 \varepsilon.
%\]
%%%
%\item Comme la suite $(X_n)_{n\in\N}$ est tendue,  elle admet une sous-suite $(X_{\varphi(n)})_{n\in\N}$ qui converge en loi vers une variable aléatoire $X$.
%On a alors
%\[
%\forall t \in \R, \quad 
%\phi_{X_{\varphi(n)}}(t) \xrightarrow[n\to\infty]{} \phi_X(t).
%\]
%et donc $\phi_X = \phi$.
%Il suffit alors d'appliquer le théorème de Lévy faible.
%\end{enumerate}
%%\newpage
%\end{comment}


\end{document}
