\documentclass[a4paper,11pt]{article}
\usepackage[utf8]{inputenc}
\usepackage[T1]{fontenc}
\usepackage{lmodern}

\usepackage{amsthm,amsmath,amsfonts,amssymb,bbm,mathrsfs,stmaryrd}
\usepackage{mathtools}
\usepackage{enumitem}
\usepackage{url}
\usepackage{dsfont}
\usepackage{appendix}
\usepackage{amsthm}
\usepackage[dvipsnames,svgnames]{xcolor}
\usepackage{graphicx}

\usepackage{fancyhdr,lastpage,titlesec,verbatim,ifthen}

\usepackage[colorlinks=true, linkcolor=black, urlcolor=black, citecolor=black]{hyperref}

\usepackage[french]{babel}

\usepackage{caption,tikz,subfigure}

\usepackage[top=2cm, bottom=2cm, left=2cm, right=2cm]{geometry}

%%%%%%%%%%%%%%%%%%%%%%%%%%%%%%%%%%%%%%%%%%%%%%%%%%%%%%%%%%%%%%%%%%%%%%%

%%%%%%%% Taille de la legende des images %%%%%%%%%%%%%%%%%%%%%%%%%%%%%%%
\renewcommand{\captionfont}{\footnotesize}
\renewcommand{\captionlabelfont}{\footnotesize}

%%%%%%%% Numeration des enumerates en romain et chgt de l'espace %%%%%%%
\setitemize[1]{label=$\rhd$, font=\color{NavyBlue},leftmargin=0.8cm}
\setenumerate[1]{font=\color{NavyBlue},leftmargin=0.8cm}
\setenumerate[2]{font=\color{NavyBlue},leftmargin=0.49cm}
%\setlist[enumerate,1]{label=(\roman*), font = \normalfont,itemsep=4pt,topsep=4pt} 
%\setlist[itemize,1]{label=\textbullet, font = \normalfont,itemsep=4pt,topsep=4pt} 

%%%%%%%% Pas d'espacement supplementaire avant \left et apres \right %%%
%%%%%%%% Note : pour les \Big(, utiliser \Bigl( \Bigr) %%%%%%%%%%%%%%%%%
\let\originalleft\left
\let\originalright\right
\renewcommand{\left}{\mathopen{}\mathclose\bgroup\originalleft}
\renewcommand{\right}{\aftergroup\egroup\originalright}

%%%%%%%%%%%%%%%%%%%%%%%%%%%%%%%%%%%%%%%%%%%%%%%%%%%%%%%%%%%%%%%%%%%%%%%

\newcommand{\N}{\mathbb{N}}
\newcommand{\Z}{\mathbb{Z}}
\newcommand{\Q}{\mathbb{Q}}
\newcommand{\R}{\mathbb{R}}
\newcommand{\C}{\mathbb{C}}
\newcommand{\T}{\mathbb{T}}
\renewcommand{\P}{\mathbb{P}}
\newcommand{\E}{\mathbb{E}}

\newcommand{\1}{\mathbbm{1}}

\newcommand{\cA}{\mathcal{A}}
\newcommand{\cB}{\mathcal{B}}
\newcommand{\cC}{\mathcal{C}}
\newcommand{\cE}{\mathcal{E}}
\newcommand{\cF}{\mathcal{F}}
\newcommand{\cG}{\mathcal{G}}
\newcommand{\cH}{\mathcal{H}}
\newcommand{\cI}{\mathcal{I}}
\newcommand{\cJ}{\mathcal{J}}
\newcommand{\cL}{\mathcal{L}}
\newcommand{\cM}{\mathcal{M}}
\newcommand{\cN}{\mathcal{N}}
\newcommand{\cP}{\mathcal{P}}
\newcommand{\cS}{\mathcal{S}}
\newcommand{\cT}{\mathcal{T}}
\newcommand{\cU}{\mathcal{U}}

\newcommand{\Ec}[1]{\mathbb{E} \left[#1\right]}
\newcommand{\Pp}[1]{\mathbb{P} \left(#1\right)}
\newcommand{\Ppsq}[2]{\mathbb{P} \left(#1\middle|#2\right)}

\newcommand{\e}{\varepsilon}

\newcommand{\ii}{\mathrm{i}}
\DeclareMathOperator{\re}{Re}
\DeclareMathOperator{\im}{Im}
\DeclareMathOperator{\Arg}{Arg}

\newcommand{\diff}{\mathop{}\mathopen{}\mathrm{d}}
\DeclareMathOperator{\Var}{Var}
\DeclareMathOperator{\Cov}{Cov}
\newcommand{\supp}{\mathrm{supp}}

\newcommand{\abs}[1]{\left\lvert#1\right\rvert}
\newcommand{\abso}[1]{\lvert#1\rvert}
\newcommand{\norme}[1]{\left\lVert#1\right\rVert}
\newcommand{\ps}[2]{\langle #1,#2 \rangle}

\newcommand{\petito}[1]{o\mathopen{}\left(#1\right)}
\newcommand{\grandO}[1]{O\mathopen{}\left(#1\right)}

\newcommand\relphantom[1]{\mathrel{\phantom{#1}}}

\newcommand{\NB}[1]{{\color{NavyBlue}#1}}
\newcommand{\DSB}[1]{{\color{DarkSlateBlue}#1}}

%%%%%%%% Theorems styles %%%%%%%%%%%%%%%%%%%%%%%%%%%%%%%%%%%%%%%%%%%%%%
\theoremstyle{plain}
\newtheorem{theorem}{Theorem}[section]
\newtheorem{proposition}[theorem]{Proposition}
\newtheorem{lemma}[theorem]{Lemme}
\newtheorem{corollary}[theorem]{Corollaire}
\newtheorem{conjecture}[theorem]{Conjecture}
\newtheorem{definition}[theorem]{Définition}

\theoremstyle{definition}
\newtheorem{remark}[theorem]{Remarque}
\newtheorem{example}[theorem]{Exemple}
\newtheorem{question}[theorem]{Question}

%%%%%%%% Macros spéciales TD %%%%%%%%%%%%%%%%%%%%%%%%%%%%%%%%%%%%%%%%%%

%%%%%%%%%%%% Changer numérotation des pages %%%%%%%%%%%%%%%%%%%%%%%%%%%%
\pagestyle{fancy}
\cfoot{\thepage/\pageref{LastPage}} %%% numéroter page / total de pages
\renewcommand{\headrulewidth}{0pt} %%% empêcher qu'il y ait une ligne horizontale en haut
%%%%%%%%%%%% Ne pas numéroter les pages %%%%%%%%%%%%%%%%%
%\pagestyle{empty}

%%%%%%%%%%%% Supprimer les alineas %%%%%%%%%%%%%%%%%%%%%%%%%%%%%%%%%%%%%
\setlength{\parindent}{0cm} 

%%%%%%%%%%%% Exercice %%%%%%%%%%%%%%%%%%%%%%%%%%%%%%%%%% 
\newcounter{exo}
\newenvironment{exo}[1][vide]
{\refstepcounter{exo}
	{\noindent \textcolor{DarkSlateBlue}{\textbf{Exercice \theexo.}}}
	\ifthenelse{\equal{#1}{vide}}{}{\textcolor{DarkSlateBlue}{(#1)}}
}{}

%%%%%%%%%%%% Partie %%%%%%%%%%%%%%%%%%%%%%%%%%%%%%%%%%%%
\newcounter{partie}
\newcommand\partie[1]{
	\stepcounter{partie}%
	{\bigskip\large\textbf{\DSB{\thepartie.~#1}}\bigskip}
	}

%%%%%%%%%%%% Separateur entre les exos %%%%%%%%%%%%%%%%%
\newcommand{\separationexos}{
	\bigskip
%	{\centering\hfill\DSB{\rule{0.4\linewidth}{1.2pt}}\hfill}\medskip
	}

%%%%%%%%%%%% Corrige %%%%%%%%%%%%%%%%%%%%%%%%%%%%%%%%%%% 
\renewenvironment{comment}{\medskip\noindent \textcolor{BrickRed}{\textbf{Corrigé.}}}{}

%%%%%%%%%%%% Titre %%%%%%%%%%%%%%%%%%%%%%%%%%%%%%%%%%%%%%
\newcommand\titre[1]{\ \vspace{-1cm}
	
	\DSB{\rule{\linewidth}{1.2pt}}
	{\small Probabilités et statistiques continues avancées}
	\hfill {\small Université Paul Sabatier}
	
	{\small KMAXPP03}
	\hfill {\small Licence 3, Printemps 2023}\medskip
	\begin{center}
		{\Large\textbf{\DSB{#1}}}\vspace{-.2cm}
	\end{center}
	\DSB{\rule{\linewidth}{1.2pt}}\medskip
}

%%%%%%%%%%%%%%%%%%%%%%%%%%%%%%%%%%%%%%%%%%%%%%%%%%%%%%%%%%%%%%%%%%%%%%%
\begin{document}
%%%%%%%%%%%%%%%%%%%%%%%%%%%%%%%%%%%%%%%%%%%%%%%%%%%%%%%%%%%%%%%%%%%%%%%

\titre{TD 6 -- Inégalités de concentration et intervalles de confiance}


%%%%%%%%%%%%%%%%%%%%%%%%%%%%%%%%%%%%%%%%%%%%%%%%%%%
\partie{Inégalités}
%%%%%%%%%%%%%%%%%%%%%%%%%%%%%%%%%%%%%%%%%%%%%%%%%%%


\begin{exo}[Inégalité de Jensen]
	Soit $X$ une v.a. réelle telle que $\E[\abs{x}]< \infty$. 
	Soit $\varphi \colon \R \to \R_+$ une fonction convexe.
	Montrer que 
	\[
		\Ec{\varphi(X)} \geq \varphi(\E[X]).
	\]
	Pour cela, on utilisera qu'une fonction convexe est égale au supremum de toutes les fonctions affines qui lui sont inférieures :
	\[
		\forall x \in \R, \quad \varphi(x) = \sup_{(a,b) \in \cE_\varphi} (ax+b),
	\]
	où $\cE_\varphi = \{ (a,b) \in \R^2 : \forall x \in \R, \varphi(x) \geq ax+b\}$.
\end{exo}


\begin{comment}
	Soit $(a,b) \in \cE_\varphi$. On a $\varphi(x) \geq ax+b$ pour tout $x \in \R$, donc
	\[
	\Ec{\varphi(X)} \geq \E[aX+b] = a \E[X] + b.
	\]
	En prenant le supremum sur tout $(a,b) \in \cE_\varphi$, on obtient
	\[
	\Ec{\varphi(X)} \geq \sup_{(a,b) \in \cE_\varphi} a \E[X] + b = \varphi(\E[X]).
	\]
\end{comment}

%%%%%%
\separationexos
%%%%%%

\begin{exo}
	Soit $X_1,\dots,X_n \in L^2(\Omega,\cA,\P)$ des v.a. indépendantes.
	Soit $\beta > 0$.
	Pour $k \in \llbracket 1,n \rrbracket$, supposons que $\E[X_k] = 0$ et $\Var(X_k) = k^{\beta}$. 
	\begin{enumerate}
		\item Montrer que, pour tout $\varepsilon > 0$, 
		\[
		\Pp{ \abs{\frac{X_1 + \dots + X_n}{n}} \geq \varepsilon}
		\leq \frac{n^{\beta-1}}{\varepsilon^2}.
		\]
		\item Pour quels $\beta$ peut-on dire que la moyenne empirique $(X_1 + \dots + X_n)/n$ se concentre en 0 quand $n \to \infty$?
	\end{enumerate}
\end{exo}

\begin{comment}
\begin{enumerate}
\item Soit $\overline{X} = (X_1 + \dots + X_n)/n$. 
On a $\E[\overline{X}] = (\E[X_1] + \dots + \E[X_n])/n = 0$. Donc par l'inégalité de Bienaymé--Chebychev,
\[
\Pp{ \abs{\frac{X_1 + \dots + X_n}{n}} \geq \varepsilon}
= \Pp{ \abs{\overline{X}- \E[\overline{X}]} \geq \varepsilon}
\leq \frac{\Var(\overline{X})}{\varepsilon^2}.
\]
Puis on a, par indépendance de $X_1,\dots,X_n$
\[
	\Var(\overline{X}) = \frac{1}{n^2} \sum_{k=1}^{n} \Var(X_k) 
	= \frac{1}{n^2} \sum_{k=1}^{n} k^{\beta} 
	\leq \frac{1}{n^2} \cdot n \cdot n^{\beta} = n^{\beta-1}.
\]
On en déduit l'inégalité demandée.
%%
\item Pour $\beta<1$ on peut dire que la moyenne empirique se concentre en 0 car pour tout $\varepsilon > 0$,  
\[
\Pp{ \abs{\frac{X_1 + \dots + X_n}{n}} \geq \varepsilon}
\xrightarrow[n \to \infty]{ }0.
\]
\end{enumerate}
\end{comment}




%%%%%%%%%%%%%%%%%%%%%%%%%%%%%%%%%%%%%%%%%%%%%%%%%%%
\partie{Intervalles de confiance}
%%%%%%%%%%%%%%%%%%%%%%%%%%%%%%%%%%%%%%%%%%%%%%%%%%%

\begin{exo}[Confiance exacte en les lois normales]
	On observe des v.a. $X_1,\dots,X_n$ i.i.d. dont la loi appartient à la famille $(P_\theta)_{\theta \in\R}$, où $P_\theta = \cN(\theta,1)$. Soit $\alpha \in {]}0,1{[}$. 
	\begin{enumerate}
		\item Si $X_1,\dots,X_n$ sont de loi $P_\theta$, montrer que $X_1 + \dots + X_n$ a la même loi que $\sqrt{n}Z+n\theta$ où $Z$ est une v.a. de loi $\cN(0,1)$.
		%%
		\item En déduire un intervalle de confiance $I$ pour $\theta$ de niveau \textit{exactement} $1-\alpha$, i.e. tel que, si $X_1,\dots,X_n$ ont loi $P_\theta$, alors
		\[
		\P(\theta \in I) = 1-\alpha.
		\]
		On exprimera cet intervalle en utilisant $\Phi^{-1}$, où $\Phi$ est la fonction de répartition de la loi $\cN(0,1)$.
	\end{enumerate}
\end{exo}

\begin{comment}
\begin{enumerate}
	\item Cela se montre en calculant la fonction caractéristique.
	%%
	\item Supposons que $X_1,\dots,X_n$ sont de loi $P_\theta$.
	Soit $\overline{X} = (X_1 + \dots + X_n)/n$. Par la question 1, $\overline{X}$ a la même loi que 
	\[
		\frac{\sqrt{n}Z+n\theta}{n} 
		= \frac{Z}{\sqrt{n}}  + \theta
	\]
	et donc, pour tout $\varepsilon > 0$,
	\[
	\Pp{\abs{\overline{X} - \theta} > \varepsilon}
	= \Pp{\abs{\frac{Z}{\sqrt{n}}} > \varepsilon}
	= \Pp{\abs{Z} > \varepsilon \sqrt{n}}
	= \Pp{Z > \varepsilon \sqrt{n}} + \Pp{Z > -\varepsilon \sqrt{n}}
	= 2 \Pp{Z > \varepsilon \sqrt{n}},
	\]
	par symétrie de $Z$ (la densité de $Z$ est une fonction paire).
	On a alors 
	\[
	\Pp{Z > \varepsilon \sqrt{n}} 
	= 1 - \Pp{Z \leq \varepsilon \sqrt{n}} 
	= 1 - \Phi(\varepsilon \sqrt{n}).
	\]
	On a donc 
	\[
	\Pp{\abs{\overline{X} - \theta} \geq \varepsilon}
	= 2 \left( 1 - \Phi(\varepsilon \sqrt{n}) \right).
	\]
	Ainsi
	\[
	\Pp{\theta \in \left[ \overline{X}-\varepsilon,\overline{X}+\varepsilon \right]}
	= 1 - 2 \left( 1 - \Phi(\varepsilon \sqrt{n}) \right).
	\]
	On veut donc choisir $\varepsilon$ tel que 
	\[
	2 \left( 1 - \Phi(\varepsilon \sqrt{n}) \right) = \alpha 
	\quad \Leftrightarrow \quad 
	\Phi(\varepsilon \sqrt{n}) = 1 - \frac{\alpha}{2}
	\quad \Leftrightarrow \quad
	\varepsilon = \frac{1}{\sqrt{n}} \Phi^{-1} \left( 1 - \frac{\alpha}{2} \right).
	\]
	On a donc l'intervalle 
	\[
		I = \left[ \overline{X} - \frac{1}{\sqrt{n}} \Phi^{-1} \left( 1 - \frac{\alpha}{2} \right),\overline{X} + \frac{1}{\sqrt{n}} \Phi^{-1} \left( 1 - \frac{\alpha}{2} \right) \right].
	\]
\end{enumerate}
\end{comment}


%%%%%%
\separationexos
%%%%%%

\begin{exo}[Lois uniformes]
	On observe des v.a. $X_1,\dots,X_n$ i.i.d. dont la loi appartient à la famille $(P_\theta)_{\theta \in [0,2]}$, où $P_\theta$ est la loi uniforme sur $[0,\theta]$.
	Soit $\alpha \in {]}0,1{[}$. 
	\begin{enumerate}
		\item 
		\begin{enumerate}
			\item Construire un intervalle de confiance pour $\theta$ de niveau $1-\alpha$ en utilisant l'inégalité de Bienaymé--Chebychev.
			%%
			\item Construire un intervalle de confiance pour $\theta$ de niveau $1-\alpha$ en utilisant l'inégalité de Hoeffding.
			%%
			\item Déterminer pour quelles valeurs de $\alpha$ le second intervalle de confiance est meilleur que le premier.
		\end{enumerate}
		%%
		\item Soit $M_n = \max(X_1,\dots,X_n)$. 
		\begin{enumerate}
			\item Si $X_1,\dots,X_n$ sont de loi $P_\theta$, calculer $\P(M_n \in [\theta - \varepsilon,\theta])$ pour $\varepsilon > 0$.
			%%
			\item En déduire un intervalle de confiance pour $\theta$ de niveau $1-\alpha$.
			%%
			\item Le comparer aux intervalles de confiance précédents dans le régime où $\alpha$ est fixé et $n \to \infty$.
		\end{enumerate}
	\end{enumerate}
\end{exo}

\begin{comment}
\begin{enumerate}
	\item 
	\begin{enumerate}
		\item Soit $\theta \in [0,2]$. Supposons que $X_1,\dots,X_n$ sont de loi $P_\theta$.
		On a alors $\E[X_1] = \theta/2$ et 
		\[
			\E[X_1^2] = \int_0^\theta x^2 \frac{1}{\theta} \diff x = \frac{\theta^2}{3},
		\]
		donc $\Var(X_1) = \theta^2/12$.
		Ainsi, par l'inégalité de Bienaymé--Chebychev, pour tout $\varepsilon > 0$, 
		\[
		\Pp{ \abs{\frac{X_1 + \dots + X_n}{n} - \frac{\theta}{2}} \geq \varepsilon}
		\leq \frac{\Var(X_1)}{n \varepsilon^2}
		\leq \frac{1}{3 n \varepsilon^2},
		\]
		où l'on a borné $\theta^2 \leq 4$ pour avoir une borne qui ne dépend pas de $\theta$.
		En posant $\overline{X} = (X_1 + \dots + X_n)/n$, on en déduit que
		\[
		\Pp{\theta \in \left[2\overline{X}-2\varepsilon,2\overline{X}+2\varepsilon\right]}
		= \Pp{\abs{\overline{X} - \frac{\theta}{2}} \leq \varepsilon}
		\geq 1 - \frac{1}{3 n \varepsilon^2}.
		\]
		pour avoir un intervalle de confiance de niveau $1-\alpha$, on choisit donc $\varepsilon$ tel que 
		\[
			\frac{1}{3 n \varepsilon^2} = \alpha 
			\quad \Leftrightarrow \quad 
			\varepsilon = \frac{1}{\sqrt{3n\alpha}}.
		\]
		Donc
		\[ 
		\left[2\overline{X}-\frac{2}{\sqrt{3n\alpha}},2\overline{X}+\frac{2}{\sqrt{3n\alpha}}\right]
		\]
		est un intervalle de confiance pour $\theta$ de niveau $1-\alpha$.
		%%
		\item Soit $\theta \in [0,2]$. Supposons que $X_1,\dots,X_n$ sont de loi $P_\theta$. Alors, par l'inégalité de Hoeffding,
		\[
		\Pp{ \abs{\overline{X} - \frac{\theta}{2}} \geq \varepsilon}
		\leq 2 \exp \left( - \frac{2 \varepsilon^2 n}{\theta^2} \right)
		\leq 2 \exp \left( - \frac{\varepsilon^2 n}{2} \right),
		\]
		où l'on a borné $\theta^2 \leq 4$.
		Ainsi
		\[
		\Pp{\theta \in \left[2\overline{X}-2\varepsilon,2\overline{X}+2\varepsilon\right]}
		\geq 1 - 2 \exp \left( - \frac{\varepsilon^2 n}{2} \right)
		\]
		et on choisit $\varepsilon$ tel que 
		\[
		2 \exp \left( - \frac{\varepsilon^2 n}{2} \right) = \alpha 
		\quad \Leftrightarrow \quad 
		\varepsilon = \sqrt{\frac{2 \log(2/\alpha)}{n}}.
		\]
		Donc
		\[ 
		\left[ 2\overline{X}-2\sqrt{\frac{2 \log(2/\alpha)}{n}},
		2\overline{X}+2\sqrt{\frac{2 \log(2/\alpha)}{n}} \right]
		\]
		est un intervalle de confiance pour $\theta$ de niveau $1-\alpha$.
		%%
		\item Remarquons tout d'abord que la longueur de l'intervalle est proportionnelle à $1/\sqrt{n}$ dans les deux cas.
		Le second intervalle de confiance est strictement meilleur que le premier ssi
		\[
		\sqrt{2 \log(2/\alpha)} < \frac{1}{\sqrt{3\alpha}}
		\quad \Leftrightarrow \quad 
		6 \alpha \log(2/\alpha) < 1
		\quad \Leftrightarrow \quad 
		\alpha < \alpha_0,
		\]
		où $\alpha_0$ est l'unique solution dans $[0,1]$ de l'équation $6 \alpha \log(2/\alpha) = 1$. On a $\alpha_0 \simeq 0.0435504\dots$
	\end{enumerate}
	%%
	\item Soit $M_n = \max(X_1,\dots,X_n)$. 
	\begin{enumerate}
		\item On a 
		\[
		\P(M_n \in [\theta - \varepsilon,\theta])
		= 1 - \P(M_n < \theta - \varepsilon) 
		= 1 - \left( \frac{\theta-\varepsilon}{\theta} \right)^n
		= 1 - \left( 1 - \frac{\varepsilon}{\theta} \right)^n,
		\]
		en utilisant l'indépendance de $X_1,\dots,X_n$ (voir le DM1 pour les détails !).
		%%
		\item Par la question précédente,
		\[
		\P(\theta \in [M_n,M_n+\varepsilon])
		= \P(M_n \in [\theta - \varepsilon,\theta])
		= 1 - \left( 1 - \frac{\varepsilon}{\theta} \right)^n
		\geq 1 - \left( 1 - \frac{\varepsilon}{2} \right)^n.
		\]
		Donc, il suffit de choisir $\varepsilon$ tel que
		\[
		\left( 1 - \frac{\varepsilon}{2} \right)^n = \alpha 
		\quad \Leftrightarrow \quad 
		\varepsilon = 2 \left( 1- \alpha^{1/n} \right).
		\]
		Donc
		\[ 
		\left[ M_n,M_n+ 2 \left( 1- \alpha^{1/n} \right) \right]
		\]
		est un intervalle de confiance pour $\theta$ de niveau $1-\alpha$.
		%%
		\item Fixons $\alpha$. La longueur de cet intervalle, quand $n \to \infty$, se comporte ainsi
		\[
		2 \left( 1- \alpha^{1/n} \right) 
		= 2 \left( 1- \exp \left( \frac{1}{n} \log \alpha \right) \right)
		\sim 2 \cdot \left( - \frac{1}{n} \log \alpha \right)
		= \frac{2 \log (1/\alpha)}{n}.
		\]
		La longueur décroît donc en $1/n$ alors que pour les intervalles de confiance précédents elle décroît en $1/\sqrt{n}$. On a donc fait mieux pour $n$ suffisamment grand!
		
		Dans les premières questions, on utilise la moyenne empirique pour estimer $\theta/2$ et dans ce cas on ne peut pas obtenir mieux que $1/\sqrt{n}$ pour la longueur de l'intervalle de confiance car, par le théorème central limite (qu'on verra plus tard), $1/\sqrt{n}$ est l'ordre des fluctuations typiques de la moyenne empirique autour de $\theta/2$. Il se trouve qu'on a trouvé une autre manière d'estimer, le maximum $M_n$ qui lui ne fluctue qu'à l'échelle $1/n$ autour de $\theta$, il permet donc de construire un intervalle de confiance plus précis.
	\end{enumerate}
\end{enumerate}
\end{comment}

%%%%%%%%%%%%%%%%%%%%%%%%%%%%%%%%%%%%%%%%%%%%%%%%%%%
\partie{Compléments}
%%%%%%%%%%%%%%%%%%%%%%%%%%%%%%%%%%%%%%%%%%%%%%%%%%%


\begin{exo}[Concentration avec un moment d'ordre 4]
	\begin{enumerate}
		\item Soit $X \in L^4(\Omega,\cA,\P)$. Montrer que, pour tout $a > 0$,
		\[
		\P(\abs{X-\E[X]} \geq a) \leq \frac{\E[(X-\E[X])^4]}{a^4}.
		\]
		%%
		\item Soit $Y_1,\dots,Y_n$ des v.a. réelles i.i.d. telles que $\E[Y_1^4] < \infty$ et $\E[Y_1] = 0$. Montrer que
		\[
		\Ec{(Y_1 + \dots + Y_n)^4}
		= 3 n (n-1) \Ec{Y_1^2}^2 + n \Ec{Y_1^4}.
		\]
		%%
		\item Soit $X_1,\dots,X_n$ des v.a. réelles i.i.d. telles que $\E[X_1^4] < \infty$.
		Montrer que, pour tout $\varepsilon > 0$, 
		\[
		\Pp{ \abs{\frac{X_1 + \dots + X_n}{n} - \E[X_1]} \geq \varepsilon}
		\leq \frac{1}{n^2\varepsilon^4} 
		\left( 3 \Var(X_1) + \frac{1}{n} \E[(X_1-\E[X_1])^4 \right).
		\]
	\end{enumerate}
\end{exo}

\begin{comment}
\begin{enumerate}
\item On passe à la puissance 4 l'inégalité à l'intérieur de la probabilité, puis on applique l'inégalité de Markov :
\[
\P(\abs{X-\E[X]} \geq a) 
= \Pp{(X-\E[X])^4 \geq a^4}
\leq \frac{\E[(X-\E[X])^4]}{a^4}.
\]
%%
\item On développe la puissance :
\[
\Ec{(Y_1 + \dots + Y_n)^4}
= \sum_{i,j,k,\ell=1}^n \E[Y_iY_jY_kY_\ell].
\]
Par indépendance et comme $\E[Y_k]=0$, l'espérance $\E[Y_iY_jY_kY_\ell]$ est nulle dès que l'un des indices $i,j,k,\ell$ n'apparaît qu'une fois. Il y a alors deux cas où cette espérance n'est pas nulle :
\begin{itemize}
\item Si les indices $i,j,k,\ell$ sont tous égaux. Alors l'espérance vaut $\E[Y_1^4]$ et il y a $n$ termes de ce type (il y a $n$ possibilités pour cette valeur commune).
\item Si les indices $i,j,k,\ell$ contiennent deux valeurs différentes et ce deux fois chacune. Alors l'espérance vaut $\E[Y_1^2 Y_2^2]$ et il y a $3n(n-1)$ termes de ce type (il y a $n$ manières de choisir la valeur de $i$, 3 manières de choisir l'indice parmi $j,k,\ell$ qui est égal à $i$, et $(n-1)$ manières de choisir la valeur des deux autres indices).
\end{itemize}
Cela donne le résultat souhaité.
%%
\item Comme $n\E[X_1] = \E[X_1 + \dots + X_n]$, on a
\begin{align*}
\Pp{ \abs{\frac{X_1 + \dots + X_n}{n} - \E[X_1]} \geq \varepsilon}
& = \Pp{ \abs{X_1 + \dots + X_n - \E[X_1 + \dots + X_n]} \geq n\varepsilon} \\
& \leq \frac{1}{(n\varepsilon)^4} 
\Ec{(X_1 + \dots + X_n - \E[X_1 + \dots + X_n])^4 }
\end{align*}
On pose $Y_i = X_i - \E[X_i]$. Alors, $Y_1,\dots,Y_n$ sont i.i.d. et $Y_i \in L^4$ et $\E[Y_i] = 0$.
Ainsi, en utilisant la question 2,
\begin{align*}
\Ec{(X_1 + \dots + X_n - \E[X_1 + \dots + X_n])^4 }
& = \Ec{(Y_1 + \dots + Y_n)^4} \\
& = 3 n (n-1) \Ec{Y_1^2}^2 + n \Ec{Y_1^4}  \\
& = 3 n (n-1) \Var(X_1)^2 + n \Ec{(X_1-\E[X_1])^4} \\
& \leq 3 n^2 \Var(X_1)^2 + n \Ec{(X_1-\E[X_1])^4}.
\end{align*}
On obtient donc
\begin{align*}
\Pp{ \abs{\frac{X_1 + \dots + X_n}{n} - \E[X_1]} \geq \varepsilon}
& \leq \frac{1}{(n\varepsilon)^4} 
\left( 3 n^2 \Var(X_1)^2 + n \Ec{(X_1-\E[X_1])^4} \right) \\
& \leq \frac{1}{n^2\varepsilon^4} 
\left( 3 \Var(X_1) + \frac{1}{n} \E[(X_1-\E[X_1])^4 \right).
\end{align*}
\end{enumerate}
\end{comment}

%%%%%%
\separationexos
%%%%%%


\begin{exo}[Lois normales de variance inconnue]
	On observe des v.a. $X_1,\dots,X_n$ i.i.d. dont la loi appartient à la famille $(P_\theta)_{\theta \in \Theta}$, où $P_\theta = \cN(0,\theta^2)$.
	Soit $\alpha \in {]}0,1{[}$. 
	\begin{enumerate}
		\item On suppose que $\Theta = {]} 0 , \sigma_{\max}]$, où $\sigma_{\max} > 0$ est un paramètre connu.
		\begin{enumerate}
			\item En considérant 
			\[
			\frac{X_1^2 + \dots + X_n^2}{n}
			\]
			et en utilisant l'inégalité de Bienaymé--Chebychev, construire un intervalle de confiance pour $\theta^2$ de niveau $1-\alpha$.
			%%
			\item En déduire un intervalle de confiance pour $\theta$ de niveau $1-\alpha$.
		\end{enumerate}
		%%
		\item On ne suppose plus que l'on connaît a priori une borne supérieure pour le paramètre $\theta$. On considère donc $\Theta = \R_+^*$.
		\begin{enumerate}
			\item Montrer que, pour tout $\varepsilon \in {]}0,1{[}$, 
			\[
			\Pp{ \abs{\frac{X_1^2 + \dots + X_n^2}{n} - \theta^2} \geq  \frac{X_1^2 + \dots + X_n^2}{n} \varepsilon} 
			\leq \Pp{ \abs{\frac{X_1^2 + \dots + X_n^2}{n} - \theta^2} \geq \frac{\theta^2 \varepsilon}{1-\varepsilon} }.
			\]
			%%
			\item En déduire un intervalle de confiance de niveau $1-\alpha$ pour $\theta^2$ puis pour $\theta$.
		\end{enumerate}
	\end{enumerate}
\end{exo}


\end{document}
