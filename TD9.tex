\documentclass[a4paper,11pt]{article}
\usepackage[utf8]{inputenc}
\usepackage[T1]{fontenc}
\usepackage{lmodern}

\usepackage{amsthm,amsmath,amsfonts,amssymb,bbm,mathrsfs,stmaryrd}
\usepackage{mathtools}
\usepackage{enumitem}
\usepackage{url}
\usepackage{dsfont}
\usepackage{appendix}
\usepackage{amsthm}
\usepackage[dvipsnames,svgnames]{xcolor}
\usepackage{graphicx}

\usepackage{fancyhdr,lastpage,titlesec,verbatim,ifthen}

\usepackage[colorlinks=true, linkcolor=black, urlcolor=black, citecolor=black]{hyperref}

\usepackage[french]{babel}

\usepackage{caption,tikz,subfigure}

\usepackage[top=2cm, bottom=2cm, left=2cm, right=2cm]{geometry}

%%%%%%%%%%%%%%%%%%%%%%%%%%%%%%%%%%%%%%%%%%%%%%%%%%%%%%%%%%%%%%%%%%%%%%%

%%%%%%%% Taille de la legende des images %%%%%%%%%%%%%%%%%%%%%%%%%%%%%%%
\renewcommand{\captionfont}{\footnotesize}
\renewcommand{\captionlabelfont}{\footnotesize}

%%%%%%%% Numeration des enumerates en romain et chgt de l'espace %%%%%%%
\setitemize[1]{label=$\rhd$, font=\color{NavyBlue},leftmargin=0.8cm}
\setenumerate[1]{font=\color{NavyBlue},leftmargin=0.8cm}
\setenumerate[2]{font=\color{NavyBlue},leftmargin=0.49cm}
%\setlist[enumerate,1]{label=(\roman*), font = \normalfont,itemsep=4pt,topsep=4pt} 
%\setlist[itemize,1]{label=\textbullet, font = \normalfont,itemsep=4pt,topsep=4pt} 

%%%%%%%% Pas d'espacement supplementaire avant \left et apres \right %%%
%%%%%%%% Note : pour les \Big(, utiliser \Bigl( \Bigr) %%%%%%%%%%%%%%%%%
\let\originalleft\left
\let\originalright\right
\renewcommand{\left}{\mathopen{}\mathclose\bgroup\originalleft}
\renewcommand{\right}{\aftergroup\egroup\originalright}

%%%%%%%%%%%%%%%%%%%%%%%%%%%%%%%%%%%%%%%%%%%%%%%%%%%%%%%%%%%%%%%%%%%%%%%

\newcommand{\N}{\mathbb{N}}
\newcommand{\Z}{\mathbb{Z}}
\newcommand{\Q}{\mathbb{Q}}
\newcommand{\R}{\mathbb{R}}
\newcommand{\C}{\mathbb{C}}
\newcommand{\T}{\mathbb{T}}
\renewcommand{\P}{\mathbb{P}}
\newcommand{\E}{\mathbb{E}}

\newcommand{\1}{\mathbbm{1}}

\newcommand{\cA}{\mathcal{A}}
\newcommand{\cB}{\mathcal{B}}
\newcommand{\cC}{\mathcal{C}}
\newcommand{\cE}{\mathcal{E}}
\newcommand{\cF}{\mathcal{F}}
\newcommand{\cG}{\mathcal{G}}
\newcommand{\cH}{\mathcal{H}}
\newcommand{\cI}{\mathcal{I}}
\newcommand{\cJ}{\mathcal{J}}
\newcommand{\cL}{\mathcal{L}}
\newcommand{\cM}{\mathcal{M}}
\newcommand{\cN}{\mathcal{N}}
\newcommand{\cP}{\mathcal{P}}
\newcommand{\cS}{\mathcal{S}}
\newcommand{\cT}{\mathcal{T}}
\newcommand{\cU}{\mathcal{U}}

\newcommand{\Ec}[1]{\mathbb{E} \left[#1\right]}
\newcommand{\Pp}[1]{\mathbb{P} \left(#1\right)}
\newcommand{\Ppsq}[2]{\mathbb{P} \left(#1\middle|#2\right)}

\newcommand{\e}{\varepsilon}

\newcommand{\ii}{\mathrm{i}}
\DeclareMathOperator{\re}{Re}
\DeclareMathOperator{\im}{Im}
\DeclareMathOperator{\Arg}{Arg}

\newcommand{\diff}{\mathop{}\mathopen{}\mathrm{d}}
\DeclareMathOperator{\Var}{Var}
\DeclareMathOperator{\Cov}{Cov}
\newcommand{\supp}{\mathrm{supp}}

\newcommand{\abs}[1]{\left\lvert#1\right\rvert}
\newcommand{\abso}[1]{\lvert#1\rvert}
\newcommand{\norme}[1]{\left\lVert#1\right\rVert}
\newcommand{\ps}[2]{\langle #1,#2 \rangle}

\newcommand{\petito}[1]{o\mathopen{}\left(#1\right)}
\newcommand{\grandO}[1]{O\mathopen{}\left(#1\right)}

\newcommand\relphantom[1]{\mathrel{\phantom{#1}}}

\newcommand{\NB}[1]{{\color{NavyBlue}#1}}
\newcommand{\DSB}[1]{{\color{DarkSlateBlue}#1}}

%%%%%%%% Theorems styles %%%%%%%%%%%%%%%%%%%%%%%%%%%%%%%%%%%%%%%%%%%%%%
\theoremstyle{plain}
\newtheorem{theorem}{Theorem}[section]
\newtheorem{proposition}[theorem]{Proposition}
\newtheorem{lemma}[theorem]{Lemme}
\newtheorem{corollary}[theorem]{Corollaire}
\newtheorem{conjecture}[theorem]{Conjecture}
\newtheorem{definition}[theorem]{Définition}

\theoremstyle{definition}
\newtheorem{remark}[theorem]{Remarque}
\newtheorem{example}[theorem]{Exemple}
\newtheorem{question}[theorem]{Question}

%%%%%%%% Macros spéciales TD %%%%%%%%%%%%%%%%%%%%%%%%%%%%%%%%%%%%%%%%%%

%%%%%%%%%%%% Changer numérotation des pages %%%%%%%%%%%%%%%%%%%%%%%%%%%%
\pagestyle{fancy}
\cfoot{\thepage/\pageref{LastPage}} %%% numéroter page / total de pages
\renewcommand{\headrulewidth}{0pt} %%% empêcher qu'il y ait une ligne horizontale en haut
%%%%%%%%%%%% Ne pas numéroter les pages %%%%%%%%%%%%%%%%%
%\pagestyle{empty}

%%%%%%%%%%%% Supprimer les alineas %%%%%%%%%%%%%%%%%%%%%%%%%%%%%%%%%%%%%
\setlength{\parindent}{0cm} 

%%%%%%%%%%%% Exercice %%%%%%%%%%%%%%%%%%%%%%%%%%%%%%%%%% 
\newcounter{exo}
\newenvironment{exo}[1][vide]
{\refstepcounter{exo}
	{\noindent \textcolor{DarkSlateBlue}{\textbf{Exercice \theexo.}}}
	\ifthenelse{\equal{#1}{vide}}{}{\textcolor{DarkSlateBlue}{(#1)}}
}{}

%%%%%%%%%%%% Partie %%%%%%%%%%%%%%%%%%%%%%%%%%%%%%%%%%%%
\newcounter{partie}
\newcommand\partie[1]{
	\stepcounter{partie}%
	{\bigskip\large\textbf{\DSB{\thepartie.~#1}}\bigskip}
	}

%%%%%%%%%%%% Separateur entre les exos %%%%%%%%%%%%%%%%%
\newcommand{\separationexos}{
	\bigskip
%	{\centering\hfill\DSB{\rule{0.4\linewidth}{1.2pt}}\hfill}\medskip
	}

%%%%%%%%%%%% Corrige %%%%%%%%%%%%%%%%%%%%%%%%%%%%%%%%%%% 
%\renewenvironment{comment}{\medskip\noindent \textcolor{BrickRed}{\textbf{Corrigé.}}}{}

%%%%%%%%%%%% Titre %%%%%%%%%%%%%%%%%%%%%%%%%%%%%%%%%%%%%%
\newcommand\titre[1]{\ \vspace{-1cm}
	
	\DSB{\rule{\linewidth}{1.2pt}}
	{\small Probabilités et statistiques continues avancées}
	\hfill {\small Université Paul Sabatier}
	
	{\small KMAXPP03}
	\hfill {\small Licence 3, Printemps 2023}\medskip
	\begin{center}
		{\Large\textbf{\DSB{#1}}}\vspace{-.2cm}
	\end{center}
	\DSB{\rule{\linewidth}{1.2pt}}\medskip
}

%%%%%%%%%%%%%%%%%%%%%%%%%%%%%%%%%%%%%%%%%%%%%%%%%%%%%%%%%%%%%%%%%%%%%%%
\begin{document}
%%%%%%%%%%%%%%%%%%%%%%%%%%%%%%%%%%%%%%%%%%%%%%%%%%%%%%%%%%%%%%%%%%%%%%%

\titre{TD 9 -- Lemmes de Borel--Cantelli et convergence de v.a.}

%%%%%%%%%%%%%%%%%%%%%%%%%%%%%%%%%%%%%%%%%%%%%%%%%%%
\partie{Lemmes de Borel--Cantelli}
%%%%%%%%%%%%%%%%%%%%%%%%%%%%%%%%%%%%%%%%%%%%%%%%%%%

\begin{exo}
	On tire une infinité de fois à pile ou face avec une pièce équilibrée de manière indépendante. Montrer que, presque sûrement, on obtient une infinité de fois face et une infinité de fois pile.
\end{exo}


\begin{comment}
On considère une suite de v.a. $(X_n)_{n\geq1}$ indépendantes de loi de Bernoulli($1/2$).
On considère que $X_n=1$ correspond à un $n$-ième lancer qui donne face.
Soit $A_n = \{X_n=1\}$. Les événements $A_n$ pour $n \geq 1$ sont indépendants et $\sum_{n\geq 1} \P(A_n) = \infty$ car $\P(A_n)=1/2$.
Donc par le second lemme de Borel--Cantelli,
\[
\Pp{\limsup_{n\to\infty} A_n} = 1,
\]
ce qui signifie que, p.s., on obtient une infinité de fois face.

On montre de même que, p.s., on obtient une infinité de fois pile. 

Comme l'intersection de deux événements de probabilité 1 a probabilité 1, on en conclut que, p.s., on obtient une infinité de fois face et une infinité de fois pile.
\end{comment}

%%%%%%
\separationexos
%%%%%%

\begin{exo}
	Soit $(X_n)_{n\geq 1}$ une suite de v.a. réelles. Supposons qu'il existe $M > 0$ tel que
	\[
		\sum_{n\geq 1} \P(\abs{X_n} > M) < \infty.
	\]
	Montrer que, p.s., la suite $(X_n)_{n\geq 1}$ est bornée, c'est-à-dire
	\[
	\Pp{\left\{ \omega \in \Omega : (X_n(\omega))_{n\geq 1} \text{ est bornée dans } \R \right\}} = 1.
	\]
	\vspace{-.3cm}
\end{exo}


\begin{comment}
Par le premier lemme de Borel--Cantelli, on déduit de l'hypothèse de l'exercice que
\[
\Pp{\limsup_{n\to\infty} \{\abs{X_n} > M\}} = 0.
\]
Cela signifie que, p.s., il n'y a qu'un nombre fini de $n$ tels que $\abs{X_n} > M$.
Autrement dit, p.s., pour tout~$n$ à partir d'un certain rang, $\abs{X_n} \leq M$.
Mais une suite de réels bornée à partir d'un certain rang est bornée.
Donc, p.s., la suite $(X_n)_{n\geq 1}$ est bornée.
\end{comment}



%%%%%%%%%%%%%%%%%%%%%%%%%%%%%%%%%%%%%%%%%%%%%%%%%%%
\partie{Limites supérieures et inférieures}
%%%%%%%%%%%%%%%%%%%%%%%%%%%%%%%%%%%%%%%%%%%%%%%%%%%

\begin{exo}
	Soit $(X_n)_{n\geq 1}$ une suite de v.a. réelles. Soit $a \in \R$.
	\begin{enumerate}
		\item Établir une série d'inclusion entre les événements suivants
		\[
			\limsup_{n\to\infty}\, \{ X_n > a \}, \qquad 
			\limsup_{n\to\infty}\, \{ X_n \geq a \}, \qquad 
			\left\{ \limsup_{n\to\infty} X_n > a \right\}, \qquad 
			\left\{ \limsup_{n\to\infty} X_n \geq a \right\}.
		\]
		Pour chaque inclusion, trouver un exemple tel qu'elle soit stricte.
		%%
		\item Montrer qu'aucune inclusion n'est vraie en général entre
		\[
		\limsup_{n\to\infty}\, \{ X_n \leq a \} 
		\quad \text{et} \quad 
		\left\{ \limsup_{n\to\infty} X_n \leq a \right\}.
		\]
		Que dire du cas avec des inégalités strictes ?
		%%
		\item Reprendre la question 1. avec les événements
		\[
		\liminf_{n\to\infty}\, \{ X_n < a \}, \qquad 
		\liminf_{n\to\infty}\, \{ X_n \leq a \}, \qquad 
		\left\{ \limsup_{n\to\infty} X_n < a \right\}, \qquad 
		\left\{ \limsup_{n\to\infty} X_n \leq a \right\}.
		\]
	\end{enumerate}
\end{exo}


\begin{comment}
\begin{enumerate}
\item On a
\[
\left\{ \limsup_{n\to\infty} X_n > a \right\}
\subset
\limsup_{n\to\infty}\, \{ X_n > a \} 
\subset
\limsup_{n\to\infty}\, \{ X_n \geq a \}
\subset
\left\{ \limsup_{n\to\infty} X_n \geq a \right\}.
\]
Montrons ces inclusions:
\begin{itemize}
\item \emph{1ère inclusion.} 
Soit $\omega$ tel que $\limsup_{n\to\infty} X_n(\omega) > a$. 
Mais il existe une sous-suite de $(X_n(\omega))_{n\geq 1}$ convergeant vers $\limsup_{n\to\infty} X_n(\omega)$. Donc les éléments de cette sous-suite sont strictement supérieurs à $a$ à partir d'un certain rang.
Cela montre qu'il y a une infinité de $n$ tels que $X_n(\omega) > a$.
Donc $\omega \in \limsup_{n\to\infty}\, \{ X_n > a \}$.
%%
\item \emph{2ème inclusion.} Cela découle simplement du fait que $\{ X_n > a \} 
\subset \{ X_n \geq a \}$ pour tout $n \geq 1$.
%%
\item \emph{3ème inclusion.} Soit $\omega \in \limsup_{n\to\infty}\, \{ X_n \geq a \}$. Alors, il y a une infinité de $n$ tels que $X_n(\omega) \geq a$.
Donc, pour tout $k\geq 1$, $\sup_{n\geq k} X_n(\omega) \geq a$.
Cela montre que 
\[
\limsup_{n\to\infty} X_n(\omega) = \lim_{k\to\infty} \sup_{n\geq k} X_n(\omega) \geq a.
\]
\end{itemize}
Contruisons des exemples tels que ces inclusions sont strictes:
\begin{itemize}
\item \emph{1ère inclusion.} 
On prend $X_n = a+ 1/n$. Alors $X_n > a$ pour tout $n \geq 1$ donc 
\[
\limsup_{n\to\infty}\, \{ X_n > a \} = \Omega.
\]
Mais $\limsup_{n\to\infty} X_n = a$ donc 
\[
\left\{ \limsup_{n\to\infty} X_n > a \right\}
= \varnothing.
\]
On a donc
\[
\left\{ \limsup_{n\to\infty} X_n > a \right\}
\varsubsetneq
\limsup_{n\to\infty}\, \{ X_n > a \}.
\]
%%
\item \emph{2ème inclusion.} 
Il suffit de prendre $X_n = a$.
%%
\item \emph{3ème inclusion.} 
On prend $X_n = a- 1/n$. Alors $X_n < a$ pour tout $n \geq 1$ donc 
\[
\limsup_{n\to\infty}\, \{ X_n \geq a \} = \varnothing.
\]
Mais $\limsup_{n\to\infty} X_n = a$ donc 
\[
\left\{ \limsup_{n\to\infty} X_n \geq a \right\}
= \varnothing.
\]
On a donc
\[
\limsup_{n\to\infty}\, \{ X_n \geq a \}
\subset
\left\{ \limsup_{n\to\infty} X_n \geq a \right\}.
\]
\end{itemize}
%%
\item En considérant $X_{2n} = a$ et $X_{2n+1} = a+1$, on a
\[
\limsup_{n\to\infty}\, \{ X_n \leq a \} = \Omega
\quad \text{et} \quad 
\left\{ \limsup_{n\to\infty} X_n \leq a \right\} = \varnothing.
\]
En considérant $X_n = a+1/n$, on a
\[
\limsup_{n\to\infty}\, \{ X_n \leq a \} = \varnothing
\quad \text{et} \quad 
\left\{ \limsup_{n\to\infty} X_n \leq a \right\} = \Omega.
\]
donc aucune inclusion entre ces deux événements n'est vraie en général.


Dans le cas d'inégalités strictes, montrons que 
\[
\left\{ \limsup_{n\to\infty} X_n < a \right\}
\subset 
\limsup_{n\to\infty}\, \{ X_n < a \}.
\]
En effet, si $\limsup_{n\to\infty} X_n(\omega) < a$, alors tous les $X_n(\omega)$ à partir d'un certain rang sont strictement inférieurs à $a$. Donc en particulier une infinité d'entre eux le sont. Donc $\omega \in \limsup_{n\to\infty}\, \{ X_n < a \}$.

En revanche, l'inclusion réciproque n'est pas vraie. On peut considérer par exemple $X_{2n} = a-1$ et $X_{2n+1} = a+1$.
%%
\item En passant au complémentaire dans la série d'inclusion montrée en 1., on a
\[
\left\{ \limsup_{n\to\infty} X_n < a \right\}
\subset
\liminf_{n\to\infty}\, \{ X_n < a \}
\subset
\liminf_{n\to\infty}\, \{ X_n \leq a \}
\subset
\left\{ \limsup_{n\to\infty} X_n \leq a \right\}.
\]
Les exemples pour avoir des inégalités strictes sont donc exactement les mêmes !
\end{enumerate}
\end{comment}



%%%%%%%%%%%%%%%%%%%%%%%%%%%%%%%%%%%%%%%%%%%%%%%%%%%
\partie{Convergence de variables aléatoires}
%%%%%%%%%%%%%%%%%%%%%%%%%%%%%%%%%%%%%%%%%%%%%%%%%%%


\begin{exo} 
Soit $(p_n)_{n\geq 1}$ une suite de réels dans $[0,1]$.
Soit $(X_n)_{n\geq 1}$ une suite de v.a. indépendantes à valeurs dans $\{0,1\}$ telles que, pour tout $ n\geq 1$, 
\[
p_n = \P(X_n=1) 
\quad \text{et} \quad 
1-p_n = \P(X_n=0).
\]
\begin{enumerate}
	\item Trouver une condition nécessaire et suffisante sur la suite $(p_n)_{n\geq 1}$ pour que $(X_n)_{n\geq 1}$ converge en probabilité vers 0.
	%%
	\item Trouver une condition nécessaire et suffisante sur la suite $(p_n)_{n\geq 1}$ pour que $(X_n)_{n\geq 1}$ converge p.s. vers 0.
\end{enumerate}
\end{exo}

\begin{comment}
\begin{enumerate}
	\item Montrons que
	\[
	X_n \xrightarrow[n\to\infty]{\P} 0
	\quad \Leftrightarrow \quad 
	p_n \xrightarrow[n\to\infty]{} 0.
	\]
	$(\Leftarrow)$ Supposons que $p_n \to 0$.
	Soit $\varepsilon > 0$.
	Alors 
	\[
		\Pp{\abs{X_n-0} > \varepsilon} \leq \Pp{X_n \neq 0} = p_n \xrightarrow[n\to\infty]{} 0.
	\]
	donc $X_n \to 0$ en probabilité.
	
	$(\Rightarrow)$ Supposons que $X_n \to 0$ en probabilité.
	Alors 
	\[
	\Pp{\abs{X_n-0} > \frac{1}{2}} \xrightarrow[n\to\infty]{} 0.
	\]
	Mais $\P(\abs{X_n-0} > \frac{1}{2}) = p_n$, donc $p_n \to 0$.
	%%
	\item Montrons que
	\[
	X_n \xrightarrow[n\to\infty]{\text{p.s.}} 0
	\quad \Leftrightarrow \quad 
	\sum_{n\geq 1} p_n < \infty.
	\]
	$(\Leftarrow)$ Supposons que $\sum_{n\geq 1} p_n < \infty$.
	Soit $\varepsilon > 0$.
	Alors 
	\[
	\sum_{n\geq 1} \Pp{\abs{X_n-0} > \varepsilon} \leq \sum_{n\geq 1} p_n < \infty,
	\]
	donc, par le critère vu en cours, $X_n \to 0$ p.s.
	
	$(\Rightarrow)$ On va montrer la contraposée : supposons que $\sum_{n\geq 1} p_n = \infty$.
	Alors $\sum_{n\geq 1} \P(X_n=1) = \infty$, mais comme les événements $\{X_n=1\}$ pour $n \geq 1$ sont indépendants, par le second lemme de Borel--Cantelli,
	\[
	\Pp{\limsup_{n\to\infty} \, \{X_n=1\}} = 1.
	\]
	Donc, p.s., il existe une infinité de $n$ tels que $X_n=1$.
	Cela implique que, p.s., $\limsup_{n\to\infty} X_n \geq 1$.
	Donc, p.s. , $(X_n)_{n\geq 1}$ est pas convergente vers 0.
	En particulier, $(X_n)_{n\geq 1}$ n'est pas p.s. convergente vers 0.
\end{enumerate}
\end{comment}


%%%%%%
\separationexos
%%%%%%

\newpage

\begin{exo} 
	Soit $(X_n)_{n\geq 1}$ une suite de v.a. réelles convergeant en probabilité vers $X$.
	\begin{enumerate}
		\item Construire par récurrence une fonction strictement croissante $\varphi \colon \N^* \to \N^*$ telle que
		\[
			\forall n \geq 1, \quad 
			\Pp{\lvert X_{\varphi(n)}-X \rvert \geq \frac{1}{n} } \leq \frac{1}{2^n}.
		\]
		%%
		\item En déduire que $(X_n)_{n\geq 1}$ possède une sous-suite convergeant p.s. vers $X$.
	\end{enumerate}
\end{exo}

\begin{comment}
\begin{enumerate}
	\item \emph{Initialisation.} Comme $(X_n)_{n\geq 1}$ converge en probabilité vers $X$, on sait que 
	\[
	\Pp{\lvert X_k-X \rvert \geq 1 } \xrightarrow[k\to\infty]{} 0.
	\]
	Donc il existe $k \geq 1$ tel que $\Pp{\lvert X_k-X \rvert \geq 1 } \leq 1/2$. On pose $\varphi(1) = k$, qui vérifie l'inégalité demandée.
	
	\emph{Étape de récurrence.} Soit $n \geq 2$. Supposons que $\varphi(1)<\dots<\varphi(n-1)$ sont bien définis.
	Comme $(X_n)_{n\geq 1}$ converge en probabilité vers $X$, on sait que 
	\[
	\Pp{\lvert X_k-X \rvert \geq \frac{1}{n}} \xrightarrow[k\to\infty]{} 0.
	\]
	Donc il existe $k > \varphi(n-1)$ tel que $\Pp{\lvert X_k-X \rvert \geq 1 } \leq 2^{-n}$. On pose $\varphi(n) = k$, qui vérifie l'inégalité demandée ainsi que $\varphi(n) > \varphi(n-1)$.
	%%
	\item On a 
	\[
		\sum_{n\geq 1} \Pp{\lvert X_{\varphi(n)}-X \rvert \geq \frac{1}{n} } < \infty,
	\]
	donc, par le 1er lemme de Borel--Cantelli, 
	\[
	\Pp{ \limsup_{n\to\infty} \, \left\{ \lvert X_{\varphi(n)}-X \rvert \geq \frac{1}{n} \right\} } = 0.
	\]
	Ainsi, p.s., il n'existe qu'un nombre fini de $n$ tels que $\lvert X_{\varphi(n)}-X \rvert \geq \frac{1}{n}$. Autrement dit, p.s., pour tout $n$ à partir d'un certain rang, $\lvert X_{\varphi(n)}-X \rvert < \frac{1}{n}$.
	Cela montre que, p.s., $\lvert X_{\varphi(n)}-X \rvert \to 0$ quand $n \to \infty$.
	On a bien montré que $(X_n)_{n\geq 1}$ possède une sous-suite convergeant p.s. vers $X$.
\end{enumerate}
\end{comment}


%%%%%%%%%%%%%%%%%%%%%%%%%%%%%%%%%%%%%%%%%%%%%%%%%%%
\partie{Compléments}
%%%%%%%%%%%%%%%%%%%%%%%%%%%%%%%%%%%%%%%%%%%%%%%%%%%


\begin{exo}
	Soit $(Z_n)_{n\geq 1}$ une suite de v.a. réelles avec $Z_n$ de loi exponentielle de paramètre $n$.
	\begin{enumerate}
		\item 
		\begin{enumerate}
			\item Montrer que $Z_n$ converge presque sûrement vers $0$ lorsque $n \to \infty$.
			%%
			\item En déduire que, presque sûrement, à partir d'un certain rang, $Z_n<Z_1$.
		\end{enumerate}
		%%
		\item On suppose ici que les v.a. $(Z_n)_{n \geq 1}$ sont indépendantes. 
		Calculer la somme $\sum_{n \geq 1 } \Pp{Z_n \geq Z_1}$.
		%%
		\item Commenter.
	\end{enumerate}
\end{exo}


\begin{comment}
\begin{enumerate}
\item 
\begin{enumerate}
	\item Soit $\varepsilon>0$. 
	On a $ \Pp{Z_n >\varepsilon}= e^{-n \varepsilon}$. 
	Donc
	\[ 
	\sum_{n \geq 0} \Pp{Z_n>\varepsilon} < \infty.
	\]
	D'après le lemme Borel-Cantelli,  pour tout $\varepsilon>0$, presque sûrement, à partir d'un certain rang on a $Z_n \leq \varepsilon$. 
	Donc presque sûrement, pour entier $k \geq 1$, à partir d'un certain rang $0 \leq Z_n \leq 1/k$. 
	On en déduit que $Z_n$ converge presque sûrement vers $0$.
	%%
	\item Soit $A \coloneqq \{\omega \in \Omega : Z_n(\omega) \to 0 \text{ et } Z_1>0\} $. 
	Soit $\omega \in A$. 
	Alors à partir d'un certain rang $Z_n(\omega) <Z_1(\omega)$. 
	Comme $\Pp{A}=1$, ceci conclut.
\end{enumerate}
%%
\item On a $ \Pp{Z_n>Z_1}=1/(n+1)$ pour $n \geq 2$. Ainsi 
\[
\sum_{n \geq 1 } \Pp{Z_n>Z_1}= \infty.
\]
%%
\item Si on pouvait appliquer le 2nd lemme de Borel--Cantelli, on obtiendrait que 
\[
	\Pp{ \limsup_{n\to\infty}\, \{Z_n>Z_1\} } =1,
\]
ce qui signifie que, p.s., pour une infinité de $n$, $Z_n>Z_1$. Cela contredit ce qui a été montré à la question 1.(b).

Mais heureusement il n'y a pas de contradiction : ici le lemme Borel--Cantelli ne s'applique pas car les événements $\{Z_n>Z_1\}$ ne sont pas indépendants. 
\end{enumerate}
\end{comment}


%%%%%%
\separationexos
%%%%%%


\begin{exo}
	Soit  $(X_n)_{n\geq 1}$ une suite de v.a. réelle et $(\varepsilon_n)_{n\geq 1}$ 
	une suite de nombres positifs telle que 
	$\sum_{n=1}^{\infty}\varepsilon_n<\infty$. Supposons que 
	\[
	\sum_{n=1}^{\infty}
	\P(\abs{X_{n+1}-X_n} > \varepsilon_n)<\infty.
	\] 
	Montrer que $(X_n)_{n\geq 1}$ converge p.s.
\end{exo}

\begin{comment} 
On utilise le 1er lemme de Borel--Cantelli qui nous montre que, p.s., pour tout $n$ à partir d'un certain rang $\abs{X_{n+1}-X_n} \leq \varepsilon_n$.
On considère un tel $\omega$ : il existe $n_0$ (qui dépend de $\omega$) tel que, pour tout $n \geq n_0$, $\abs{X_{n+1}(\omega)-X_n(\omega)} \leq \varepsilon_n$.
On vérifie alors (en utilisant l'hypothèse $\sum_{n=1}^{\infty}\varepsilon_n<\infty$) que $(X_n(\omega))_{n\geq 1}$ est de Cauchy dans $\R$.
On en déduit que la suite $(X_n(\omega))_{n\geq 1}$ converge.
Donc $(X_n)_{n\geq 1}$ converge p.s.
\end{comment}



%%%%%%
\separationexos
%%%%%%



\begin{exo}
	Soit $(p_n)_{n\geq 1}$ une suite de réels dans $[0,1]$.
	Soit $(X_n)_{n\geq 1}$ une suite de v.a. indépendantes telle que $X_n$ ait loi binomiale de paramètre $(n,p_n)$.
	\begin{enumerate}
		\item Trouver une condition nécessaire et suffisante sur la suite $(p_n)_{n\geq 1}$ pour que $(X_n)_{n\geq 1}$ converge en probabilité vers 0.
		%%
		\item Trouver une condition nécessaire et suffisante sur la suite $(p_n)_{n\geq 1}$ pour que $(X_n)_{n\geq 1}$ converge dans $L^1$ vers 0.
		%%
		\item Trouver une condition nécessaire et suffisante sur la suite $(p_n)_{n\geq 1}$ pour que $(X_n)_{n\geq 1}$ converge p.s. vers 0.
	\end{enumerate}
\end{exo}

\begin{comment}
\begin{enumerate}
\item Comme $X_n$ prend des valeurs entières, on remarque que pour $\varepsilon \in {]}0,1{[}$, 
\[
\Pp{\abs{X_n-0} > \varepsilon} 
= \Pp{X_n \neq 0} 
= 1 - \Pp{X_n=0}
= 1 - (1-p_n)^n,
\]
et, pour $\varepsilon \geq 1$, on a $\Pp{\abs{X_n-0} > \varepsilon} 
\leq 1 - (1-p_n)^n$.
On a donc 
\[
X_n \xrightarrow[n\to\infty]{\P} 0 
\quad \Leftrightarrow \quad 
1 - (1-p_n)^n \xrightarrow[n\to\infty]{} 0.
\]
On peut réécrire cela ainsi
\begin{align*}
1 - (1-p_n)^n \xrightarrow[n\to\infty]{} 0
& \quad \Leftrightarrow \quad 
(1-p_n)^n \xrightarrow[n\to\infty]{} 1 \\
& \quad \Leftrightarrow \quad 
n \log(1-p_n) \xrightarrow[n\to\infty]{} 0 \\
& \quad \Leftrightarrow \quad 
n p_n \xrightarrow[n\to\infty]{} 0.
\end{align*}
%%
\item Ici on note tout simplement que 
\[
\Ec{\abs{X_n-0}} = \Ec{X_n} = np_n,
\]
donc
\[
X_n \xrightarrow[n\to\infty]{L^1} 0 
\quad \Leftrightarrow \quad 
n p_n \xrightarrow[n\to\infty]{} 0.
\]
%%	
\item Ici, en procédant comme à l'exercice 4, on peut voir que la condition nécessaire et suffisante est
\[
\sum_{n\geq1} (1 - (1-p_n)^n) < \infty
\quad \Leftrightarrow \quad 
\sum_{n\geq1} n p_n < \infty
\]
\end{enumerate}
\end{comment}


\end{document}
