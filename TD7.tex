\documentclass[a4paper,11pt]{article}
\usepackage[utf8]{inputenc}
\usepackage[T1]{fontenc}
\usepackage{lmodern}

\usepackage{amsthm,amsmath,amsfonts,amssymb,bbm,mathrsfs,stmaryrd}
\usepackage{mathtools}
\usepackage{enumitem}
\usepackage{url}
\usepackage{dsfont}
\usepackage{appendix}
\usepackage{amsthm}
\usepackage[dvipsnames,svgnames]{xcolor}
\usepackage{graphicx}

\usepackage{fancyhdr,lastpage,titlesec,verbatim,ifthen}

\usepackage[colorlinks=true, linkcolor=black, urlcolor=black, citecolor=black]{hyperref}

\usepackage[french]{babel}

\usepackage{caption,tikz,subfigure}

\usepackage[top=2cm, bottom=2cm, left=2cm, right=2cm]{geometry}

%%%%%%%%%%%%%%%%%%%%%%%%%%%%%%%%%%%%%%%%%%%%%%%%%%%%%%%%%%%%%%%%%%%%%%%

%%%%%%%% Taille de la legende des images %%%%%%%%%%%%%%%%%%%%%%%%%%%%%%%
\renewcommand{\captionfont}{\footnotesize}
\renewcommand{\captionlabelfont}{\footnotesize}

%%%%%%%% Numeration des enumerates en romain et chgt de l'espace %%%%%%%
\setitemize[1]{label=$\rhd$, font=\color{NavyBlue},leftmargin=0.8cm}
\setenumerate[1]{font=\color{NavyBlue},leftmargin=0.8cm}
\setenumerate[2]{font=\color{NavyBlue},leftmargin=0.49cm}
%\setlist[enumerate,1]{label=(\roman*), font = \normalfont,itemsep=4pt,topsep=4pt} 
%\setlist[itemize,1]{label=\textbullet, font = \normalfont,itemsep=4pt,topsep=4pt} 

%%%%%%%% Pas d'espacement supplementaire avant \left et apres \right %%%
%%%%%%%% Note : pour les \Big(, utiliser \Bigl( \Bigr) %%%%%%%%%%%%%%%%%
\let\originalleft\left
\let\originalright\right
\renewcommand{\left}{\mathopen{}\mathclose\bgroup\originalleft}
\renewcommand{\right}{\aftergroup\egroup\originalright}

%%%%%%%%%%%%%%%%%%%%%%%%%%%%%%%%%%%%%%%%%%%%%%%%%%%%%%%%%%%%%%%%%%%%%%%

\newcommand{\N}{\mathbb{N}}
\newcommand{\Z}{\mathbb{Z}}
\newcommand{\Q}{\mathbb{Q}}
\newcommand{\R}{\mathbb{R}}
\newcommand{\C}{\mathbb{C}}
\newcommand{\T}{\mathbb{T}}
\renewcommand{\P}{\mathbb{P}}
\newcommand{\E}{\mathbb{E}}

\newcommand{\1}{\mathbbm{1}}

\newcommand{\cA}{\mathcal{A}}
\newcommand{\cB}{\mathcal{B}}
\newcommand{\cC}{\mathcal{C}}
\newcommand{\cE}{\mathcal{E}}
\newcommand{\cF}{\mathcal{F}}
\newcommand{\cG}{\mathcal{G}}
\newcommand{\cH}{\mathcal{H}}
\newcommand{\cI}{\mathcal{I}}
\newcommand{\cJ}{\mathcal{J}}
\newcommand{\cL}{\mathcal{L}}
\newcommand{\cM}{\mathcal{M}}
\newcommand{\cN}{\mathcal{N}}
\newcommand{\cP}{\mathcal{P}}
\newcommand{\cS}{\mathcal{S}}
\newcommand{\cT}{\mathcal{T}}
\newcommand{\cU}{\mathcal{U}}

\newcommand{\Ec}[1]{\mathbb{E} \left[#1\right]}
\newcommand{\Pp}[1]{\mathbb{P} \left(#1\right)}
\newcommand{\Ppsq}[2]{\mathbb{P} \left(#1\middle|#2\right)}

\newcommand{\e}{\varepsilon}

\newcommand{\ii}{\mathrm{i}}
\DeclareMathOperator{\re}{Re}
\DeclareMathOperator{\im}{Im}
\DeclareMathOperator{\Arg}{Arg}

\newcommand{\diff}{\mathop{}\mathopen{}\mathrm{d}}
\DeclareMathOperator{\Var}{Var}
\DeclareMathOperator{\Cov}{Cov}
\newcommand{\supp}{\mathrm{supp}}

\newcommand{\abs}[1]{\left\lvert#1\right\rvert}
\newcommand{\abso}[1]{\lvert#1\rvert}
\newcommand{\norme}[1]{\left\lVert#1\right\rVert}
\newcommand{\ps}[2]{\langle #1,#2 \rangle}

\newcommand{\petito}[1]{o\mathopen{}\left(#1\right)}
\newcommand{\grandO}[1]{O\mathopen{}\left(#1\right)}

\newcommand\relphantom[1]{\mathrel{\phantom{#1}}}

\newcommand{\NB}[1]{{\color{NavyBlue}#1}}
\newcommand{\DSB}[1]{{\color{DarkSlateBlue}#1}}
\newcommand{\emphb}[1]{\emph{{\color{DarkSlateBlue}#1}}}

%%%%%%%% Theorems styles %%%%%%%%%%%%%%%%%%%%%%%%%%%%%%%%%%%%%%%%%%%%%%
\theoremstyle{plain}
\newtheorem{theorem}{Theorem}[section]
\newtheorem{proposition}[theorem]{Proposition}
\newtheorem{lemma}[theorem]{Lemme}
\newtheorem{corollary}[theorem]{Corollaire}
\newtheorem{conjecture}[theorem]{Conjecture}
\newtheorem{definition}[theorem]{Définition}

\theoremstyle{definition}
\newtheorem{remark}[theorem]{Remarque}
\newtheorem{example}[theorem]{Exemple}
\newtheorem{question}[theorem]{Question}

%%%%%%%% Macros spéciales TD %%%%%%%%%%%%%%%%%%%%%%%%%%%%%%%%%%%%%%%%%%

%%%%%%%%%%%% Changer numérotation des pages %%%%%%%%%%%%%%%%%%%%%%%%%%%%
\pagestyle{fancy}
\cfoot{\thepage/\pageref{LastPage}} %%% numéroter page / total de pages
\renewcommand{\headrulewidth}{0pt} %%% empêcher qu'il y ait une ligne horizontale en haut
%%%%%%%%%%%% Ne pas numéroter les pages %%%%%%%%%%%%%%%%%
%\pagestyle{empty}

%%%%%%%%%%%% Supprimer les alineas %%%%%%%%%%%%%%%%%%%%%%%%%%%%%%%%%%%%%
\setlength{\parindent}{0cm} 

%%%%%%%%%%%% Exercice %%%%%%%%%%%%%%%%%%%%%%%%%%%%%%%%%% 
\newcounter{exo}
\newenvironment{exo}[1][vide]
{\refstepcounter{exo}
	{\noindent \textcolor{DarkSlateBlue}{\textbf{Exercice \theexo.}}}
	\ifthenelse{\equal{#1}{vide}}{}{\textcolor{DarkSlateBlue}{(#1)}}
}{}

%%%%%%%%%%%% Partie %%%%%%%%%%%%%%%%%%%%%%%%%%%%%%%%%%%%
\newcounter{partie}
\newcommand\partie[1]{
	\stepcounter{partie}%
	{\bigskip\large\textbf{\DSB{\thepartie.~#1}}\bigskip}
	}

%%%%%%%%%%%% Separateur entre les exos %%%%%%%%%%%%%%%%%
\newcommand{\separationexos}{
	\bigskip
%	{\centering\hfill\DSB{\rule{0.4\linewidth}{1.2pt}}\hfill}\medskip
	}

%%%%%%%%%%%% Corrige %%%%%%%%%%%%%%%%%%%%%%%%%%%%%%%%%%% 
\renewenvironment{comment}{\medskip\noindent \textcolor{BrickRed}{\textbf{Corrigé.}}}{}

%%%%%%%%%%%% Titre %%%%%%%%%%%%%%%%%%%%%%%%%%%%%%%%%%%%%%
\newcommand\titre[1]{\ \vspace{-1cm}
	
	\DSB{\rule{\linewidth}{1.2pt}}
	{\small Probabilités et statistiques continues avancées}
	\hfill {\small Université Paul Sabatier}
	
	{\small KMAXPP03}
	\hfill {\small Licence 3, Printemps 2023}\medskip
	\begin{center}
		{\Large\textbf{\DSB{#1}}}\vspace{-.2cm}
	\end{center}
	\DSB{\rule{\linewidth}{1.2pt}}\medskip
}

%%%%%%%%%%%%%%%%%%%%%%%%%%%%%%%%%%%%%%%%%%%%%%%%%%%%%%%%%%%%%%%%%%%%%%%
\begin{document}
%%%%%%%%%%%%%%%%%%%%%%%%%%%%%%%%%%%%%%%%%%%%%%%%%%%%%%%%%%%%%%%%%%%%%%%

\titre{TD 7 -- Retour sur les convergences $L^p$ et p.s. et sur l'indépendance}


%%%%%%%%%%%%%%%%%%%%%%%%%%%%%%%%%%%%%%%%%%%%%%%%%%%
\partie{Convergence $L^p$}
%%%%%%%%%%%%%%%%%%%%%%%%%%%%%%%%%%%%%%%%%%%%%%%%%%%

\noindent \textcolor{DarkSlateBlue}{\textbf{Rappel.}} 
Soit $p \in [1,\infty[$. Soit $X,X_1,X_2,\dots \in L^p$. 
On dit que $(X_n)_{n\geq 1}$ converge dans $L^p$ vers $X$, que l'on note
\[
X_n \xrightarrow[n\to\infty]{L^p} X,
\]
si $\norme{X_n-X}_p\to 0$ quand $n \to \infty$, où $\norme{X_n-X}_p = \E[\abs{X_n-X}^p]^{1/p}$.


%%%%%%
\separationexos
%%%%%%

\begin{exo}[Stabilité par opérations de la convergence $L^p$] 
	Soit $p,q \in [1,\infty[$.
	Soit $X,X_1,X_2,\dots \in L^p$ et $Y,Y_1,Y_2,\dots \in L^q$. On suppose que
	\[
	X_n \xrightarrow[n\to\infty]{L^p} X 
	\quad \text{et} \quad 
	Y_n \xrightarrow[n\to\infty]{L^q} Y.
	\]
	\begin{enumerate}
		\item Si $p = q$ et $a,b \in \R$, montrer que $(aX_n+bY_n)_{n\geq 1}$ converge dans $L^p$ vers $aX+bY$.
		%%
		\item Si $\frac{1}{p} + \frac{1}{q} \leq 1$, montrer que $(X_nY_n)_{n\geq 1}$ converge dans $L^r$ vers $XY$, avec $r$ tel que $\frac{1}{p} + \frac{1}{q} = \frac{1}{r}$.
		%%
		\item Si $f\colon \R \to \R$ est lipschitzienne, montrer que $(f(X_n))_{n\geq 1}$ converge dans $L^p$ vers $f(X)$.
	\end{enumerate}
\end{exo}

\begin{comment}
\begin{enumerate}
\item C'est une conséquence du fait que la convergence $L^p$ soit associée à une norme. Plus précisément,
\begin{align*}
\norme{(aX_n+bY_n) - (aX+bY)}_p
& = \norme{a(X_n-X)+b(Y_n-Y)}_p \\
& \leq \norme{a(X_n-X)}_p + \norme{b(Y_n-Y)}_p \\
& = a\norme{X_n-X}_p + b\norme{Y_n-Y}_p \\
& \xrightarrow[n\to\infty]{} 0.
\end{align*}
%%
\item Soit $r$ tel que $\frac{1}{p} + \frac{1}{q} = \frac{1}{r}$.
Par inégalité triangulaire,
\begin{align*}
\norme{X_nY_n-XY}_r
& \leq \norme{X_nY_n-XY_n}_r + \norme{XY_n-XY}_r \\
& = \norme{(X_n-X)Y_n}_r + \norme{X(Y_n-Y)}_r.
\end{align*}
On utilise ensuite l'inégalité de Hölder avec $a = p/r$ et $b = q/r$ de sorte que $\frac{1}{a} + \frac{1}{b} = 1$ : 
\begin{align*}
	\norme{(X_n-X)Y_n}_r^r = \Ec{\abs{(X_n-X)Y_n}^r}
	& \leq \Ec{\abs{X_n-X}^{ar}}^{1/a} \Ec{\abs{Y_n}^{br}}^{1/b} \\
	& = \Ec{\abs{X_n-X}^p}^{r/p} \Ec{\abs{Y_n}^q}^{r/q} \\
	& \xrightarrow[n\to\infty]{} 0,
\end{align*}
car $\Ec{\abs{X_n-X}^p} \to 0$ et $\Ec{\abs{Y_n}^q} \to \Ec{\abs{Y}^q} < \infty$ (conséquence de la convergence $L^q$).
Similairement,
\begin{align*}
\norme{X(Y_n-Y)}_r^r 
& \leq \Ec{\abs{X}^p}^{r/p} \Ec{\abs{Y_n-Y}^q}^{r/q}
\xrightarrow[n\to\infty]{} 0.
\end{align*}
On en conclut que 
\begin{align*}
\norme{X_nY_n-XY}_r \xrightarrow[n\to\infty]{} 0.
\end{align*}
%%
\item Si $f$ est $L$-lipschitzienne, alors
\[
	\Ec{\abs{f(X_n) - f(X)}^p} 
	\leq \Ec{(L\abs{X_n - X})^p} 
	= L^p \Ec{\abs{X_n - X}^p} 
	\xrightarrow[n\to\infty]{} 0.
\]
\end{enumerate}
\end{comment}

%%%%%%
\separationexos
%%%%%%

\begin{exo} 
	Soit $m \in \R$ et $v > 0$.
	Soit $X_1,X_2,\dots \in L^2$ des v.a. indépendantes de même moyenne $m$ et de variance majorée par $v$.
	Montrer que
	\[
	\frac{X_1+ \dots + X_n}{n} \xrightarrow[n \to \infty]{L^2} m.
	\]
\end{exo}


%%%%%%%%%%%%%%%%%%%%%%%%%%%%%%%%%%%%%%%%%%%%%%%%%%%
\partie{Convergence p.s.}
%%%%%%%%%%%%%%%%%%%%%%%%%%%%%%%%%%%%%%%%%%%%%%%%%%%

\begin{exo}
	On considère une suite $(Y_n)_{n\geq 1}$ de v.a. réelles positives.
	\begin{enumerate}
		\item Montrer que, si $\forall k \geq 1, Y_k \leq 1/k^2$  p.s.,
		alors $\sum_{k\geq 1} Y_k$ converge p.s.
		%%
		\item Pour $m \geq 1$ et $n \geq 0$, on définit $A_{m,n} = \{ Y_1 + \cdots + Y_m \geq n\}$.
		Parmi les événements 
		\[
		\bigcap_{m\geq1} \bigcup_{n\geq0} A_{m,n}, \quad 
		\bigcap_{n\geq0} \bigcup_{m\geq1} A_{m,n}, \quad 
		\bigcup_{m\geq1} \bigcap_{n\geq0} A_{m,n}, \quad 
		\bigcup_{n\geq0} \bigcap_{m\geq1} A_{m,n}, 
		\]
		lequel correspond à l'événement $\{\sum_{k\geq 1} Y_k \text{ diverge}\}$ ? 
		À quoi correspondent les trois autres ?
		%%
		\item Montrer que si $\sum_{k\geq 1} Y_k$ diverge p.s., alors, pour tout $n \geq 0$,
		\[
		\mathbb{P}(Y_1 + \dots + Y_m \geq n)
		\xrightarrow[m\to\infty]{} 1.
		\]
	\end{enumerate}
\end{exo}



%%%%%%%%%%%%%%%%%%%%%%%%%%%%%%%%%%%%%%%%%%%%%%%%%%%
\partie{Indépendances de suites}
%%%%%%%%%%%%%%%%%%%%%%%%%%%%%%%%%%%%%%%%%%%%%%%%%%%

\noindent \textcolor{DarkSlateBlue}{\textbf{Définition.}}
\begin{itemize}
	\item Soit $A_1,A_2,\ldots \in \cA$.
	On dit que $(A_k)_{k\geq 1}$ est une \emphb{suite d'événements indépendants} si, pour tout $n \geq 1$, $A_1,\dots,A_n$ sont indépendants.
	\item Soit $X_1,X_2,\dots$ des v.a.
	On dit que $(X_k)_{k\geq 1}$ est une \emphb{suite de variables aléatoires indépendantes} si, pour tout $n \geq 1$, $X_1,\dots,X_n$ sont indépendantes.
\end{itemize}

%%%%%%
\separationexos
%%%%%%

\begin{exo} 
	Soit $(A_k)_{k\geq 1}$ une suite d'événements indépendants. Montrer que
		\[
		\P \Biggl( \bigcap_{n\geq 1} A_n \Biggr)
		= \prod_{n\geq 1} \P(A_n).
		\]
\end{exo}

\begin{comment}
Notons que $\prod_{n\geq 1} \P(A_n)$ est toujours bien défini car $\P(A_n) \in [0,1]$, donc la suite $(\prod_{n=1}^m \P(A_n))_{m\geq 1}$ est décroissante et a donc une limite (dans $[0,1]$).
D'autre part, on a
\[
\P \Biggl( \bigcap_{n\geq 1} A_n \Biggr)
= \P \Biggl( \bigcap_{m\geq 1} \downarrow \bigcap_{n=1}^m A_n \Biggr)
= \lim_{m\to \infty} \downarrow \P \Biggl( \bigcap_{n=1}^m A_n \Biggr)
= \lim_{m\to \infty} \downarrow \prod_{n=1}^m \P(A_n)
= \prod_{n\geq 1} \P(A_n),
\]
où on a utilisé dans la 3ème égalité le fait que $A_1,\dots,A_m$ sont indépendants.
\end{comment}


%%%%%%
\separationexos
%%%%%%

\begin{exo} \label{exo indep piles et faces}
	Soit $(X_n)_{n\geq1}$ une suite de variables aléatoires indépendantes de loi de Bernoulli de paramètre $p \in {]}0,1[$.
	Soit $N$ une variable aléatoire de loi de Poisson de paramètre $\lambda >0$, c'est-à-dire vérifiant
	\[
	\forall k \in \N, \quad \Pp{N=k} = \frac{\lambda^k}{k!} e^{-\lambda}.
	\]
	On suppose que $N$ est indépendante de $(X_n)_{n\geq1}$, i.e. pour tout $n \geq $, $N,X_1,\dots,X_n$ sont indépendants.
	On pose 
	\[
	P \coloneqq \sum_{i=1}^N X_i
	\quad \text{et} \quad 
	F \coloneqq N-P=\sum_{i=1}^N(1-X_i),
	\]
	avec $P=F=0$ sur $\{N=0\}$. Les variables aléatoires $P$ et $F$ représentent respectivement le nombre de piles et de faces dans un jeu de pile ou face de paramètre $p$ à $N$ lancers.
	%%
	\begin{enumerate}
		\item Déterminer la loi du couple $(P,N)$.
		%%
		\item En déduire les lois de $P$ et $F$ et montrer que $P$ et $F$ sont indépendantes.
	\end{enumerate}
\end{exo}


\begin{comment}
\begin{enumerate}
\item On a $\P(P=0,N=0) = \P(N=0) = e^{-\lambda}$, et pour $n\geq1$ et $0\leq k\leq n$, 
\begin{align*}
\P(P=k,N=n) 
& = \P\left(N=n,\sum_{i=1}^nX_i=k\right)
= \P(N=n)\P\left(\sum_{i=1}^nX_i=k\right) 
= e^{-\lambda}\frac{\lambda^n}{n!} \binom{n}{k}p^k(1-p)^{n-k}\\ 
& = e^{-\lambda}\frac{(\lambda p)^k}{k!}\frac{(\lambda(1-p))^{n-k}}{(n-k)!}.
\end{align*}
%%
\item On a pour $k,l\geq0$, 
\[
\P(P=k,F=l)
=\P(P=k,N=k+l)
=\left(e^{-\lambda p}\frac{(\lambda p)^k}{k!}\right)
\left(e^{-\lambda(1-p)}\frac{(\lambda(1-p))^l}{l!}\right).
\] 
Donc les variables aléatoires $P$ et $F$ sont indépendantes et de lois respectives les lois de Poisson de paramètres $\lambda p$ et $\lambda(1-p)$. 

\emphb{Remarque.} On utilise en fait ici le petit résultat suivant. Soit $X$ et $Y$ deux v.a. discrètes à valeurs respectivement dans des espaces dénombrables $I$ et $J$ et telles que pour tous $i \in I$ et $j \in J$, on ait $\P(X=i,Y=j)=p_i q_j$. 
Alors $X$ et $Y$ sont indépendantes et on a, pour tout $i \in I$, $\P(X = i) = p_i/c$ où $c \coloneqq \sum_{i \in I} p_i$ et, pour tout $j \in J$, $\P(Y = j) = c q_j $. 

Montrons ce résultat. Soit $j \in J$, on a
\[
\P(Y = j) 
= \sum_{i\in I} \P(X=i,Y=j) 
= \sum_{i\in I} p_i q_j
= q_j c.
\]
Comme $\sum_{j\in J} \P(Y=j) = 1$, on en déduit que $\sum_{j\in J} q_j = 1/c$
Puis de la même manière, on a, pour $i \in I$
\[
\P(X = i) 
= \sum_{j\in J} \P(X=i,Y=j) 
= p_i \sum_{j\in J} q_j
= p_i / c.
\]
Alors, l'hypothèse se réécrit $\P(X=i,Y=j)=\P(X=i) \P(Y=j)$ pour tous $i \in I$ et $j \in J$, ce qui montre l'indépendance entre $X$ et $Y$.
\end{enumerate}
\end{comment}



%%%%%%%%%%%%%%%%%%%%%%%%%%%%%%%%%%%%%%%%%%%%%%%%%%%
\partie{Compléments}
%%%%%%%%%%%%%%%%%%%%%%%%%%%%%%%%%%%%%%%%%%%%%%%%%%%



\begin{exo}
	Deux personnes $A$ et $B$ jouent une suite de parties indépendantes. 
	Lors de chacune d'elles, elles ont respectivement les probabilités $p$ pour $A$ et $1-p$ pour $B$ de gagner. 
	La première personne qui obtient 2 victoires de plus que son adversaire a gagné. 
	Quelle est la probabilité que $A$ gagne?
\end{exo}


%%%%%%
\separationexos
%%%%%%

\begin{exo}[Construction d'une suite de v.a. de Bernoulli indépendantes]
	On considère $\Omega = [0,1[$, muni de $\cA = \cB([0,1[)$ et de $\P$ la mesure de Lebesgue sur $[0,1[$.
	Pour $n \geq 1$ et $\omega \in [0,1{[}$, on pose
	\[
	X_n(\omega) = \lfloor 2^n \omega \rfloor - 2 \lfloor 2^{n-1} \omega \rfloor,
	\]
	où $\lfloor x \rfloor$ désigne la partie entière d'un nombre réel $x$.
	\begin{enumerate}
		\item 
		\begin{enumerate}
			\item Pour tous $n \geq 1$ et $\omega \in [0,1{[}$, montrer que $X_n(\omega) \in \{0,1\}$.
			%%
			\item Montrer par récurrence que, pour tous $n \geq 1$ et $\omega \in [0,1{[}$,
			\[
			0 \leq \omega - \sum_{k=1}^n X_k(\omega) 2^{-k} < 2^{-n}.
			\]
			%%
			\item En déduire que 
			\[
			\omega = \sum_{k=1}^\infty X_k(\omega) 2^{-k}.
			\]
			On dit que les $X_k(\omega)$ sont les coefficients du développement dyadique propre de $\omega$ (i.e. l'unique développement dyadique de $\omega$ qui ne se termine pas par une infinité de 1).
		\end{enumerate}
		%%
		\item 
		\begin{enumerate}
			\item Pour tout $n \geq 1$, montrer que $X_n$ suit la loi de Bernoulli de paramètre $1/2$.
			\item Montrer que $(X_n)_{n\geq 1}$ est une suite de variables aléatoires indépendantes.
		\end{enumerate}
	\end{enumerate}
\end{exo}


%%%%%%
\separationexos
%%%%%%



\begin{exo}
	Soit $(X_n)_{n\geq1}$ une suite de variables aléatoires réelles indépendantes  de même loi et $N$ une variable aléatoire à valeurs dans $\N$ indépendante de la suite $(X_n)_{n\geq1}$. 
	Soit $f\colon \R\to\R_+$ une fonction mesurable. Montrer que
	\[
	\Ec{\sum_{i=1}^N f(X_i)} = \Ec{N} \Ec{f(X_1)},
	\]
	où l'on adopte la convention qu'une somme vide est nulle.
\end{exo}

\begin{comment}
On a 
\begin{align*}
\Ec{\sum_{i=1}^N f(X_i)}
&=\Ec{\sum_{n\geq1}\1_{\{N=n\}}\sum_{i=1}^nf(X_i)}
=\sum_{n\geq1} \Ec{\1_{\{N=n\}}\sum_{i=1}^nf(X_i)} 
=\sum_{n\geq1} \P(N=n) \Ec{\sum_{i=1}^nf(X_i)}\\ 
&=\sum_{n\geq1} \P(N=n) n \Ec{f(X_1)}
=\E[N] \Ec{f(X_1)},
\end{align*} 
où l'on a utilisé que $N$ est indépendante de $(X_1,\dots,X_n)$.
\end{comment}


\end{document}