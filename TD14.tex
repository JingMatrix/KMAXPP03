\documentclass[a4paper,11pt]{article}
\usepackage[utf8]{inputenc}
\usepackage[T1]{fontenc}
\usepackage{lmodern}

\usepackage{amsthm,amsmath,amsfonts,amssymb,bbm,mathrsfs,stmaryrd}
\usepackage{mathtools}
\usepackage{enumitem}
\usepackage{url}
\usepackage{dsfont}
\usepackage{appendix}
\usepackage{amsthm}
\usepackage[dvipsnames,svgnames]{xcolor}
\usepackage{graphicx}

\usepackage{fancyhdr,lastpage,titlesec,verbatim,ifthen}

\usepackage[colorlinks=true, linkcolor=black, urlcolor=black, citecolor=black]{hyperref}

\usepackage[french]{babel}

\usepackage{caption,tikz,subfigure}

\usepackage[top=2cm, bottom=2cm, left=2cm, right=2cm]{geometry}

%%%%%%%%%%%%%%%%%%%%%%%%%%%%%%%%%%%%%%%%%%%%%%%%%%%%%%%%%%%%%%%%%%%%%%%

%%%%%%%% Taille de la legende des images %%%%%%%%%%%%%%%%%%%%%%%%%%%%%%%
\renewcommand{\captionfont}{\footnotesize}
\renewcommand{\captionlabelfont}{\footnotesize}

%%%%%%%% Numeration des enumerates en romain et chgt de l'espace %%%%%%%
\setitemize[1]{label=$\rhd$, font=\color{NavyBlue},leftmargin=0.8cm}
\setenumerate[1]{font=\color{NavyBlue},leftmargin=0.8cm}
\setenumerate[2]{font=\color{NavyBlue},leftmargin=0.49cm}
%\setlist[enumerate,1]{label=(\roman*), font = \normalfont,itemsep=4pt,topsep=4pt} 
%\setlist[itemize,1]{label=\textbullet, font = \normalfont,itemsep=4pt,topsep=4pt} 

%%%%%%%% Pas d'espacement supplementaire avant \left et apres \right %%%
%%%%%%%% Note : pour les \Big(, utiliser \Bigl( \Bigr) %%%%%%%%%%%%%%%%%
\let\originalleft\left
\let\originalright\right
\renewcommand{\left}{\mathopen{}\mathclose\bgroup\originalleft}
\renewcommand{\right}{\aftergroup\egroup\originalright}

%%%%%%%%%%%%%%%%%%%%%%%%%%%%%%%%%%%%%%%%%%%%%%%%%%%%%%%%%%%%%%%%%%%%%%%

\newcommand{\N}{\mathbb{N}}
\newcommand{\Z}{\mathbb{Z}}
\newcommand{\Q}{\mathbb{Q}}
\newcommand{\R}{\mathbb{R}}
\newcommand{\C}{\mathbb{C}}
\newcommand{\T}{\mathbb{T}}
\renewcommand{\P}{\mathbb{P}}
\newcommand{\E}{\mathbb{E}}

\newcommand{\1}{\mathbbm{1}}

\newcommand{\cA}{\mathcal{A}}
\newcommand{\cB}{\mathcal{B}}
\newcommand{\cC}{\mathcal{C}}
\newcommand{\cE}{\mathcal{E}}
\newcommand{\cF}{\mathcal{F}}
\newcommand{\cG}{\mathcal{G}}
\newcommand{\cH}{\mathcal{H}}
\newcommand{\cI}{\mathcal{I}}
\newcommand{\cJ}{\mathcal{J}}
\newcommand{\cL}{\mathcal{L}}
\newcommand{\cM}{\mathcal{M}}
\newcommand{\cN}{\mathcal{N}}
\newcommand{\cP}{\mathcal{P}}
\newcommand{\cS}{\mathcal{S}}
\newcommand{\cT}{\mathcal{T}}
\newcommand{\cU}{\mathcal{U}}

\newcommand{\Ec}[1]{\mathbb{E} \left[#1\right]}
\newcommand{\Pp}[1]{\mathbb{P} \left(#1\right)}
\newcommand{\Ppsq}[2]{\mathbb{P} \left(#1\middle|#2\right)}

\newcommand{\e}{\varepsilon}

\newcommand{\ii}{\mathrm{i}}
\DeclareMathOperator{\re}{Re}
\DeclareMathOperator{\im}{Im}
\DeclareMathOperator{\Arg}{Arg}

\newcommand{\diff}{\mathop{}\mathopen{}\mathrm{d}}
\DeclareMathOperator{\Var}{Var}
\DeclareMathOperator{\Cov}{Cov}
\newcommand{\supp}{\mathrm{supp}}

\newcommand{\abs}[1]{\left\lvert#1\right\rvert}
\newcommand{\abso}[1]{\lvert#1\rvert}
\newcommand{\norme}[1]{\left\lVert#1\right\rVert}
\newcommand{\ps}[2]{\langle #1,#2 \rangle}

\newcommand{\petito}[1]{o\mathopen{}\left(#1\right)}
\newcommand{\grandO}[1]{O\mathopen{}\left(#1\right)}

\newcommand\relphantom[1]{\mathrel{\phantom{#1}}}

\newcommand{\NB}[1]{{\color{NavyBlue}#1}}
\newcommand{\DSB}[1]{{\color{DarkSlateBlue}#1}}

%%%%%%%% Theorems styles %%%%%%%%%%%%%%%%%%%%%%%%%%%%%%%%%%%%%%%%%%%%%%
\theoremstyle{plain}
\newtheorem{theorem}{Theorem}[section]
\newtheorem{proposition}[theorem]{Proposition}
\newtheorem{lemma}[theorem]{Lemme}
\newtheorem{corollary}[theorem]{Corollaire}
\newtheorem{conjecture}[theorem]{Conjecture}
\newtheorem{definition}[theorem]{Définition}

\theoremstyle{definition}
\newtheorem{remark}[theorem]{Remarque}
\newtheorem{example}[theorem]{Exemple}
\newtheorem{question}[theorem]{Question}

%%%%%%%% Macros spéciales TD %%%%%%%%%%%%%%%%%%%%%%%%%%%%%%%%%%%%%%%%%%

%%%%%%%%%%%% Changer numérotation des pages %%%%%%%%%%%%%%%%%%%%%%%%%%%%
\pagestyle{fancy}
\cfoot{\thepage/\pageref{LastPage}} %%% numéroter page / total de pages
\renewcommand{\headrulewidth}{0pt} %%% empêcher qu'il y ait une ligne horizontale en haut
%%%%%%%%%%%% Ne pas numéroter les pages %%%%%%%%%%%%%%%%%
%\pagestyle{empty}

%%%%%%%%%%%% Supprimer les alineas %%%%%%%%%%%%%%%%%%%%%%%%%%%%%%%%%%%%%
\setlength{\parindent}{0cm} 

%%%%%%%%%%%% Exercice %%%%%%%%%%%%%%%%%%%%%%%%%%%%%%%%%% 
\newcounter{exo}
\newenvironment{exo}[1][vide]
{\refstepcounter{exo}
	{\noindent \textcolor{DarkSlateBlue}{\textbf{Exercice \theexo.}}}
	\ifthenelse{\equal{#1}{vide}}{}{\textcolor{DarkSlateBlue}{(#1)}}
}{}

%%%%%%%%%%%% Partie %%%%%%%%%%%%%%%%%%%%%%%%%%%%%%%%%%%%
\newcounter{partie}
\newcommand\partie[1]{
	\stepcounter{partie}%
	{\bigskip\large\textbf{\DSB{\thepartie.~#1}}\bigskip}
	}

%%%%%%%%%%%% Separateur entre les exos %%%%%%%%%%%%%%%%%
\newcommand{\separationexos}{
	\bigskip
%	{\centering\hfill\DSB{\rule{0.4\linewidth}{1.2pt}}\hfill}\medskip
	}

%%%%%%%%%%%% Corrige %%%%%%%%%%%%%%%%%%%%%%%%%%%%%%%%%%% 
%\renewenvironment{comment}{\medskip\noindent \textcolor{BrickRed}{\textbf{Corrigé.}}}{}

%%%%%%%%%%%% Titre %%%%%%%%%%%%%%%%%%%%%%%%%%%%%%%%%%%%%%
\newcommand\titre[1]{\ \vspace{-1cm}
	
	\DSB{\rule{\linewidth}{1.2pt}}
	{\small Probabilités et statistiques continues avancées}
	\hfill {\small Université Paul Sabatier}
	
	{\small KMAXPP03}
	\hfill {\small Licence 3, Printemps 2023}\medskip
	\begin{center}
		{\Large\textbf{\DSB{#1}}}\vspace{-.2cm}
	\end{center}
	\DSB{\rule{\linewidth}{1.2pt}}\medskip
}

%%%%%%%%%%%%%%%%%%%%%%%%%%%%%%%%%%%%%%%%%%%%%%%%%%%%%%%%%%%%%%%%%%%%%%%
\begin{document}
%%%%%%%%%%%%%%%%%%%%%%%%%%%%%%%%%%%%%%%%%%%%%%%%%%%%%%%%%%%%%%%%%%%%%%%

\titre{TD 14 -- Révisions}

%%%%%%%%%%%%%%%%%%%%%%%%%%%%%%%%%%%%%%%%%%%%%%%%%%%
\partie{Toujours plus de convergences}
%%%%%%%%%%%%%%%%%%%%%%%%%%%%%%%%%%%%%%%%%%%%%%%%%%%

\begin{exo}
	Soit $(U_n)_{n\geq 1}$ une suite de v.a. indépendantes de loi uniforme sur $[0,1]$.
	Pour $n \geq 1$, on pose
	\[
	M_n \coloneqq \max \left( \frac{1}{\sqrt{U_1}},\dots,\frac{1}{\sqrt{U_n}} \right).
	\]
	\begin{enumerate}
		\item Calculer la fonction de répartition de $M_n$.
		%%
		\item Soit $p>0$. Déterminer les valeurs de $p$ telles que $M_n$ a un moment d'ordre $p$ fini.
		%%
		\item Montrer que $M_n/\sqrt{n}$ converge en loi quand $n \to \infty$ vers une v.a. dont on déterminera la loi
	\end{enumerate}
\end{exo}

\begin{comment}
\begin{enumerate}
\item Pour $x < 1$, on a $F_{M_n}(x) = 0$. Pour $x \geq 1$, on a
\[
F_{M_n}(x) 
= \Pp{\frac{1}{\sqrt{U_1}} \leq x}^n
= \Pp{U_1 \geq \frac{1}{x^2}}^n
= \left( 1 - \frac{1}{x^2} \right)^n.
\]
%%
\item Comme $M_n \geq 0$, on a 
\[
\Ec{M_n^p} 
= \int_0^\infty \Pp{M_n^p \geq y} \diff y
= \int_0^\infty \Pp{M_n \geq y^{1/p}} \diff y.
\]
Or on a , pour $y \geq 1$,
\[
\Pp{M_n \geq y^{1/p}} 
= 1- F_{M_n}(y^{1/p}) 
= 1 - \left( 1 - \frac{1}{y^{2/p}} \right)^n
\sim_{y\to\infty} \frac{n}{y^{2/p}},
\]
donc $\E[M_n^p] < \infty$ si et seulement si $p < 2$.
%%
\item On passe par les fonctions de répartition.
Pour $x \leq 0$, on a $F_{M_n / \sqrt{n}}(x) = 0$. Pour $x > 0$, on a, à partir d'un certain rang tel que $x \sqrt{n} \geq 1$,
\[
F_{M_n / \sqrt{n}}(x) 
= F_{M_n}(x\sqrt{n}) 
= \left( 1 - \frac{1}{n x^2} \right)^n
\xrightarrow[n\to\infty]{} e^{-1/x^2}.
\]
On pose $F(x) \coloneqq e^{-1/x^2} \1_{x>0}$ pour $x\in\R$. 
Alors $F$ est continue croissante et tend vers 0 en $-\infty$ et en 1 en $+\infty$ donc c'est la fonction de répartition d'une variable aléatoire $Z$, et donc $M_n/\sqrt{n}$ converge en loi vers $Z$.

En outre, on a $F_Z$ est $\cC^1$ sur $\R$, donc la loi de $Z$ est
\[
F_Z'(z) \diff z
= \frac{2}{z^3} e^{-1/z^2} \1_{z>0}  \diff z.
\]
\end{enumerate}
\end{comment}



%%%%%%
\separationexos
%%%%%%

\begin{exo}
	Soit $(A_n)_{n\geq1}$ une suite d'événements indépendants.
	Pour $n\geq 1$, on pose $a_n \coloneqq \P(A_n)$ et
	\[
	b_n \coloneqq \sum_{k=1}^n a_k
	\quad \text{et} \quad
	S_n \coloneqq \sum_{k=1}^n \1_{A_k}.
	\]
	On suppose que $b_n \to \infty$ quand $n\to\infty$.
	\begin{enumerate}
		\item Montrer que $S_n / b_n \to 1$ dans $L^2$ quand $n \to \infty$.
		%%
		\item Pour tout $k \geq 1$, on pose $n_k \coloneqq \inf \{ n \in\N : b_n \geq k^2 \}$.
		Montrer que $k^2 \leq b_{n_k} < k^2 + 1$ et que $(n_k)_{k\in\N}$ est strictement croissante.
		%%
		\item Montrer que $S_{n_k} / b_{n_k} \to 1$ presque sûrement quand $k \to \infty$.
		%%
		\item En déduire que $S_n / b_n \to 1$ presque sûrement quand $n \to \infty$.
	\end{enumerate}
\end{exo}


\begin{comment}
\begin{enumerate} 
\item On a
\[
\Ec{\left( \frac{S_n}{b_n} - 1 \right)^2}
= \Var \left( \frac{S_n}{b_n} \right)
= \sum_{k=1}^n \Var \left( \frac{\1_{A_k}}{b_n} \right)
\leq \frac{1}{b_n^2} \sum_{k=1}^n \Ec{\1_{A_k}^2}
= \frac{1}{b_n^2} \sum_{k=1}^n a_k
= \frac{1}{b_n}
\xrightarrow[n\to\infty]{} 0,
\]
donc cela montre la convergence souhaitée.
%%
\item On note tout d'abord que $n_k$ est bien défini car $b_n \to \infty$. 
Par définition, $b_{n_k} \geq k^2$ et $b_{n_k-1} < k^2$. Or $a_{n_k} \leq 1$, donc on a $b_{n_k} < k^2 +1$.
En particulier, $b_{n_k} < (k+1)^2$ donc $n_k < n_{k+1}$.
%%
\item Soit $\varepsilon > 0$.
On a, par l'inégalité de Bienaymé-Tchebychev
\[
\Pp{\abs{ \frac{S_{n_k}}{b_{n_k}} -1} \geq \varepsilon}
\leq \frac{1}{\varepsilon^2} \Var \left( \frac{S_{n_k}}{b_{n_k}} \right)
\leq \frac{1}{\varepsilon^2} \frac{1}{b_{n_k}}
\leq \frac{1}{\varepsilon^2 k^2},
\]
qui est une suite sommable en $k$.
Par le critère vu en cours, cela implique que $S_{n_k} / b_{n_k} \to 1$ p.s. quand $k \to \infty$.
%%
\item Soit $n \geq 1$. 
Il existe un unique $k \geq 1$ tel que $n_k \leq n < n_{k+1}$.
Comme $(S_n)_{n\geq 1}$, $(b_n)_{n\geq 1}$ et $(n_k)_{k\geq 1}$ sont croissantes, on a
\[
\frac{S_{n_k}}{b_{n_{k+1}}} 
\leq \frac{S_n}{b_n} 
\leq \frac{S_{n_{k+1}}}{b_{n_k}} 
\]
et, comme $k^2 \leq b_{n_k} < k^2 + 1$,
\[
\frac{S_{n_k}}{b_{n_k}} \frac{k^2}{k^2+1}
\leq \frac{S_n}{b_n} 
\leq \frac{S_{n_{k+1}}}{b_{n_{k+1}}} \frac{k^2+1}{k^2}.
\]
Sur l'événement de convergence de $(S_{n_k}/b_{n_k})_{k\geq1}$ vers 1, on a donc $S_n/b_n \to 1$ quand $n \to \infty$ (car alors $k \to \infty$). 
On a donc montré que $S_n / b_n \to 1$ presque sûrement quand $n \to \infty$.
\end{enumerate}
\end{comment}


%%%%%%
\separationexos
%%%%%%


\begin{exo}[Convergence en loi de couples de v.a.]
	Soient $(X_n)_{n\geq1}$ et $(Y_n)_{n\geq1}$ deux suites de v.a. réelles convergeant en loi vers des v.a. $X$ et $Y$.
	On a vu au TD11 qu'en général on n'a pas convergence en loi de $(X_n+Y_n)_{n\geq 1}$ vers $X+Y$. On va voir ci-dessous deux cas où c'est vrai.
	\begin{enumerate}
		\item On suppose ici que $X_n$ et $Y_n$ sont indépendantes pour tout $n\geq1$ et que $X$ et $Y$ sont indépendantes. En utilisant le théorème de Lévy, montrer que $(X_n+Y_n)_{n\geq 1}$ converge en loi vers $X+Y$.
		%%
		\item On suppose que $Y$ est constante p.s. (mais plus l'indépendance). On veut montrer que $(X_n+Y_n)_{n\geq 1}$ converge en loi vers $X+Y$.
		\begin{enumerate}
			\item Montrer qu'il suffit de montrer $\Ec{f(X_n+Y_n)} \to \Ec{f(X+Y)}$ pour toute fonction $f \colon \R \to \R$ lipschitzienne bornée. On considère à présent $f$ une telle fonction.
			%%
			\item Soit $\varepsilon > 0$. Montrer que
			\[
				\abs{\Ec{f(X_n+Y_n)} - \Ec{f(X_n+Y)}}  
				\leq 2 \norme{f}_\infty \P(\abs{Y_n-Y} > \varepsilon)
				+ L \varepsilon,
			\]
			où $L$ est la constante de Lipschitz de $f$. 
			%%
			\item En déduire que $\abs{\Ec{f(X_n+Y_n)} - \Ec{f(X_n+Y)}} \to 0$.
			%%
			\item Montrer que $\abs{\Ec{f(X_n+Y)} - \Ec{f(X+Y)}} \to 0$.
			%%
			\item Conclure.
		\end{enumerate}
	\end{enumerate}
\end{exo}

\begin{comment}
\begin{enumerate}
	\item Pour $\theta \in \R$ et $n \geq 1$, on a
	\[
		\phi_{X_n+Y_n} (\theta) 
		= \Ec{e^{\ii \theta X_n} \cdot e^{\ii \theta Y_n}}
		= \Ec{e^{\ii \theta X_n}} \cdot \Ec{e^{\ii \theta Y_n}}
		= \phi_{X_n} (\theta) \cdot \phi_{Y_n} (\theta),
	\]
	en utilisant l'indépendance de $X_n$ et $Y_n$.
	Comme $(X_n)_{n\geq1}$ et $(Y_n)_{n\geq1}$ converge en loi vers $X$ et $Y$, leurs fonctions caractéristiques convergent et donc
	\[
	\phi_{X_n+Y_n} (\theta) 
	\xrightarrow[n\to\infty]{} \phi_{X} (\theta) \cdot \phi_{Y} (\theta)
	= \phi_{X+Y} (\theta),
	\]
	en utilisant l'indépendance de $X$ et $Y$.
	Donc, par le théorème de Lévy, $(X_n+Y_n)_{n\geq 1}$ converge en loi vers $X+Y$.
	%%
	\item On suppose que $Y$ est constante p.s. (mais plus l'indépendance). On veut montrer que $(X_n+Y_n)_{n\geq 1}$ converge en loi vers $X+Y$.
	\begin{enumerate}
		\item Par le cours, il suffit de montrer $\Ec{f(X_n+Y_n)} \to \Ec{f(X+Y)}$ pour toute fonction $f \colon \R \to \R$ de classe $\cC^\infty$ à support compact. Mais une telle fonction $f$ est bornée et sa dérivée est bornée aussi donc elle est lipschitzienne. Donc il suffit de considérer $f$ bornée et lipschitzienne.
		%%
		\item Soit $\varepsilon > 0$. On distingue selon si $\abs{Y_n-Y} > \varepsilon$ ou non :
		\begin{align*}
		& \abs{\Ec{f(X_n+Y_n)} - \Ec{f(X_n+Y)}} \\
		& \leq \Ec{\abs{f(X_n+Y_n)-f(X_n+Y)}} \\
		& = \Ec{\abs{f(X_n+Y_n)-f(X_n+Y)} \1_{\abs{Y_n-Y} > \varepsilon}}
		+ \Ec{\abs{f(X_n+Y_n)-f(X_n+Y)} \1_{\abs{Y_n-Y} \leq \varepsilon}}.
		\end{align*}
		Dans la première espérance, on borne $\abs{f(X_n+Y_n)-f(X_n+Y)} \leq 2 \norme{f}_\infty$ et dans la 2ème on utilise que $f$ est lipschitzienne pour borner $\abs{f(X_n+Y_n)-f(X_n+Y)} \leq L \abs{Y-Y_n} \leq L\varepsilon$ sur l'événement $\{\abs{Y_n-Y} \leq \varepsilon\}$.
		On a donc le résultat voulu.
		%%
		\item Comme $(Y_n)_{n\geq1}$ converge en loi vers une v.a. constante, il y a aussi convergence en probabilité. Donc $\P(\abs{Y_n-Y} > \varepsilon) \to 0$ quand $n \to \infty$.
		Donc, pour $n$ suffisamment grand, on obtient 
		\[
		\abs{\Ec{f(X_n+Y_n)} - \Ec{f(X_n+Y)}}  
		\leq \varepsilon + L \varepsilon = (L+1)\varepsilon,
		\]
		Comme $L$ est fixé, cela montre que $\abs{\Ec{f(X_n+Y_n)} - \Ec{f(X_n+Y)}} \to 0$.
		%%
		\item La v.a. $Y$ est constante p.s. donc $Y = y$ p.s. pour un certain $y \in \R$.
		De plus, la fonction $x \mapsto f(x+y)$ est continue bornée car $f$ l'est.
		Donc
		\[
		\Ec{f(X_n+Y)} = \Ec{f(X_n+y)}
		\xrightarrow[n\to\infty]{}
		\Ec{f(X+y)} = \Ec{f(X+Y)}.
		\]
		%%
		\item Par inégalité triangulaire,
		\begin{align*}
		& \abs{\Ec{f(X_n+Y_n)} - \Ec{f(X+Y)}} \\
		& \leq \abs{\Ec{f(X_n+Y_n)} - \Ec{f(X_n+Y)}}
		+ \abs{\Ec{f(X_n+Y)} - \Ec{f(X+Y)}} \\
		& \xrightarrow[n\to\infty]{} 0,
		\end{align*}
		par les questions précédentes. C'est ce qu'on voulait montrer !
	\end{enumerate}
\end{enumerate}
\end{comment}

%%%%%%
\separationexos
%%%%%%

\begin{exo}[Lemme de Scheffé]
	Soit $(X_n)_{n\geq 1}$ une suite de v.a. réelles continues. Notons $f_n$ la densité de $X_n$. On suppose que $(f_n)_{n\geq 1}$ converge Lebesgue-presque partout vers une fonction $f$ telle que $\int_\R f(x) \diff x =1$.
	\begin{enumerate}
		\item Montrer que $\abs{f-f_n} = 2(f-f_n)_+ + f_n - f$.
		%%
		\item On note $\lambda$ la mesure de Lebesgue sur $\R$.
		Montrer que 
		\[
		\int_\R \abs{f-f_n} \diff \lambda 
		= 2 \int_\R (f-f_n)_+ \diff \lambda.
		\]
		\item En déduire que $(f_n)_{n\geq 1}$ converge vers $f$ dans $L^1(\R,\lambda)$.
		%%
		\item En déduire que $(X_n)_{n\geq 1}$ converge en loi vers une v.a. de densité $f$.
	\end{enumerate}
\end{exo}

\begin{comment}
\begin{enumerate}
	\item Soit $x \in \R$. 
	On a $\abs{x} = x_+ + x_-$. Mais aussi $x = x_+ - x_-$, donc $x_- = x_+ - x$.
	Ainsi on obtient $\abs{x} = 2 x_+ - x$.
	On applique alors cette formule à $x = f-f_n$.
	%%
	\item On applique la formule précédente (en notant que toutes les fonctions ci-dessous sont intégrables car $f$ et $f_n$ le sont) :
	\[
	\int_\R \abs{f-f_n} \diff \lambda 
	= 2 \int_\R (f-f_n)_+ \diff \lambda + \int_\R f_n \diff \lambda 
	- \int_\R f \diff \lambda.
	\]
	Mais $\int_\R f_n \diff \lambda = 1 = \int_\R f \diff \lambda$ donc on obtient la formule voulue.
	%%
	\item On sait que $(f-f_n)_+ \to 0$ $\lambda$-p.p. et que $\abs{(f-f_n)_+} \leq f$ qui est $\lambda$-intégrable. donc par convergence dominée,
	\[
	\int_\R \abs{f-f_n} \diff \lambda 
	\xrightarrow[n\to\infty]{} 0,
	\]
	ce qui montre que $(f_n)_{n\geq 1}$ converge vers $f$ dans $L^1(\R,\lambda)$.
	%%
	\item Soit $X$ une v.a. de densité $f$ (existe car $f$ d'intégrale 1 et $f$ est positive comme limite de fonction positive).
	Soit $g \colon \R \to \R$ continue bornée.
	Alors 
	\begin{align*}
	\abs{\Ec{g(X_n)} - \Ec{g(X)}}
	= \abs{\int_\R gf_n \diff \lambda - \int_\R gf \diff \lambda}
	\leq \int_\R g\abs{f_n-f} \diff \lambda
	\leq \norme{g}_\infty \int_\R \abs{f-f_n} \diff \lambda 
	\xrightarrow[n\to\infty]{} 0.
	\end{align*}
	Donc $(X_n)_{n\geq 1}$ converge en loi vers $X$.
\end{enumerate}
\end{comment}


%%%%%%%%%%%%%%%%%%%%%%%%%%%%%%%%%%%%%%%%%%%%%%%%%%%
\partie{Compléments}
%%%%%%%%%%%%%%%%%%%%%%%%%%%%%%%%%%%%%%%%%%%%%%%%%%%


\begin{exo}
	On reprend l'exercice 3 du DST1. On observe des v.a. réelles $X_1,\dots,X_n$ i.i.d. dont la loi appartient à la famille $(P_\theta)_{\theta \in [1,4]}$, où $P_\theta$ est définie par
	\[
	P_\theta(\diff x) = \frac{3 \theta^3}{x^4} \1_{[\theta,\infty[}(x) \diff x.
	\]
	On rappelle que si $X_1$ a pour loi $P_\theta$ pour un certain $\theta \in [1,4]$, alors $\E[X_1] = \frac{3\theta}{2}$ et $\Var(X_1) = \frac{3\theta}{4}$.
	
	Soit $\alpha \in {]}0,1{[}$. 
	En utilisant le théorème central limite, construire un intervalle de confiance asymptotique pour $\theta$ de niveau $1-\alpha$. 
\end{exo}

%%%%%%
\separationexos
%%%%%%

\begin{exo}
	Soit $(p_n)_{n\geq1}$ une suite de réels dans ${]}0,1{[}$.
	Pour chaque $n \geq 1$, soit $X_n$ une v.a. de loi Binomiale$(n,p_n)$.
	Dans ce cours, on a déjà vu que :
	\begin{itemize}
		\item Si $p_n = p \in {]}0,1{[}$ est fixé, alors par le TCL
		\[
			\frac{X_n - n p}{\sqrt{n^p(1-p)}}
			\xrightarrow[n\to\infty]{\text{loi}} 
			\cN(0,1). 
		\]
		%% 
		\item Si $np_n \to \lambda \in \R_+^*$, alors 
		\[
			X_n
			\xrightarrow[n\to\infty]{\text{loi}} 
			\mathrm{Poisson}(\lambda). 
		\]
	\end{itemize}
	Ici on suppose que $p_n \to 0$ mais $np_n \to \infty$ quand $n\to\infty$.
	On va montrer que
	\[
	Z_n \coloneqq \frac{X_n - n p}{\sqrt{np_n}}
	\xrightarrow[n\to\infty]{\text{loi}} 
	\cN(0,1). 
	\]
	\begin{enumerate}
		\item Montrer que, pour tout $\theta \in \R$, 
		\[
		\phi_{Z_n}(\theta) 
		= \left( e^{-\ii \theta \sqrt{p_n/n}} \left( 1 + p_n \left(e^{\ii \theta /\sqrt{np_n}} - 1 \right) \right) \right)^n.
		\]
		\item Montrer que, quand $n \to \infty$,
		\[
		e^{-\ii \theta \sqrt{p_n/n}} \left( 1 + p_n \left(e^{\ii \theta /\sqrt{np_n}} - 1 \right) \right)
		= 1 - \frac{\theta^2}{2n} + o \left( \frac{1}{n} \right).
		\]
		\item Conclure.
	\end{enumerate}
\end{exo}


\end{document}
