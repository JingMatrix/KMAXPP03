\documentclass[a4paper,11pt]{article}
\usepackage[utf8]{inputenc}
\usepackage[T1]{fontenc}
\usepackage{lmodern}

\usepackage{amsthm,amsmath,amsfonts,amssymb,bbm,mathrsfs,stmaryrd}
\usepackage{mathtools}
\usepackage{enumitem}
\usepackage{url}
\usepackage{dsfont}
\usepackage{appendix}
\usepackage{amsthm}
\usepackage[dvipsnames,svgnames]{xcolor}
\usepackage{graphicx}

\usepackage{fancyhdr,lastpage,titlesec,verbatim,ifthen}

\usepackage[colorlinks=true, linkcolor=black, urlcolor=black, citecolor=black]{hyperref}

\usepackage[french]{babel}

\usepackage{caption,tikz,subfigure}

\usepackage[top=2cm, bottom=2cm, left=2cm, right=2cm]{geometry}

%%%%%%%%%%%%%%%%%%%%%%%%%%%%%%%%%%%%%%%%%%%%%%%%%%%%%%%%%%%%%%%%%%%%%%%

%%%%%%%% Taille de la legende des images %%%%%%%%%%%%%%%%%%%%%%%%%%%%%%%
\renewcommand{\captionfont}{\footnotesize}
\renewcommand{\captionlabelfont}{\footnotesize}

%%%%%%%% Numeration des enumerates en romain et chgt de l'espace %%%%%%%
\setitemize[1]{label=$\rhd$, font=\color{NavyBlue},leftmargin=0.8cm}
\setenumerate[1]{font=\color{NavyBlue},leftmargin=0.8cm}
\setenumerate[2]{font=\color{NavyBlue},leftmargin=0.49cm}
%\setlist[enumerate,1]{label=(\roman*), font = \normalfont,itemsep=4pt,topsep=4pt} 
%\setlist[itemize,1]{label=\textbullet, font = \normalfont,itemsep=4pt,topsep=4pt} 

%%%%%%%% Pas d'espacement supplementaire avant \left et apres \right %%%
%%%%%%%% Note : pour les \Big(, utiliser \Bigl( \Bigr) %%%%%%%%%%%%%%%%%
\let\originalleft\left
\let\originalright\right
\renewcommand{\left}{\mathopen{}\mathclose\bgroup\originalleft}
\renewcommand{\right}{\aftergroup\egroup\originalright}

%%%%%%%%%%%%%%%%%%%%%%%%%%%%%%%%%%%%%%%%%%%%%%%%%%%%%%%%%%%%%%%%%%%%%%%

\newcommand{\N}{\mathbb{N}}
\newcommand{\Z}{\mathbb{Z}}
\newcommand{\Q}{\mathbb{Q}}
\newcommand{\R}{\mathbb{R}}
\newcommand{\C}{\mathbb{C}}
\newcommand{\T}{\mathbb{T}}
\renewcommand{\P}{\mathbb{P}}
\newcommand{\E}{\mathbb{E}}

\newcommand{\1}{\mathbbm{1}}

\newcommand{\cA}{\mathcal{A}}
\newcommand{\cB}{\mathcal{B}}
\newcommand{\cC}{\mathcal{C}}
\newcommand{\cE}{\mathcal{E}}
\newcommand{\cF}{\mathcal{F}}
\newcommand{\cG}{\mathcal{G}}
\newcommand{\cH}{\mathcal{H}}
\newcommand{\cI}{\mathcal{I}}
\newcommand{\cJ}{\mathcal{J}}
\newcommand{\cL}{\mathcal{L}}
\newcommand{\cM}{\mathcal{M}}
\newcommand{\cN}{\mathcal{N}}
\newcommand{\cP}{\mathcal{P}}
\newcommand{\cS}{\mathcal{S}}
\newcommand{\cT}{\mathcal{T}}
\newcommand{\cU}{\mathcal{U}}

\newcommand{\Ec}[1]{\mathbb{E} \left[#1\right]}
\newcommand{\Pp}[1]{\mathbb{P} \left(#1\right)}
\newcommand{\Ppsq}[2]{\mathbb{P} \left(#1\middle|#2\right)}

\newcommand{\e}{\varepsilon}

\newcommand{\ii}{\mathrm{i}}
\DeclareMathOperator{\re}{Re}
\DeclareMathOperator{\im}{Im}
\DeclareMathOperator{\Arg}{Arg}

\newcommand{\diff}{\mathop{}\mathopen{}\mathrm{d}}
\DeclareMathOperator{\Var}{Var}
\DeclareMathOperator{\Cov}{Cov}
\newcommand{\supp}{\mathrm{supp}}

\newcommand{\abs}[1]{\left\lvert#1\right\rvert}
\newcommand{\abso}[1]{\lvert#1\rvert}
\newcommand{\norme}[1]{\left\lVert#1\right\rVert}
\newcommand{\ps}[2]{\langle #1,#2 \rangle}

\newcommand{\petito}[1]{o\mathopen{}\left(#1\right)}
\newcommand{\grandO}[1]{O\mathopen{}\left(#1\right)}

\newcommand\relphantom[1]{\mathrel{\phantom{#1}}}

\newcommand{\NB}[1]{{\color{NavyBlue}#1}}
\newcommand{\DSB}[1]{{\color{DarkSlateBlue}#1}}

%%%%%%%% Theorems styles %%%%%%%%%%%%%%%%%%%%%%%%%%%%%%%%%%%%%%%%%%%%%%
\theoremstyle{plain}
\newtheorem{theorem}{Theorem}[section]
\newtheorem{proposition}[theorem]{Proposition}
\newtheorem{lemma}[theorem]{Lemme}
\newtheorem{corollary}[theorem]{Corollaire}
\newtheorem{conjecture}[theorem]{Conjecture}
\newtheorem{definition}[theorem]{Définition}

\theoremstyle{definition}
\newtheorem{remark}[theorem]{Remarque}
\newtheorem{example}[theorem]{Exemple}
\newtheorem{question}[theorem]{Question}

%%%%%%%% Macros spéciales TD %%%%%%%%%%%%%%%%%%%%%%%%%%%%%%%%%%%%%%%%%%

%%%%%%%%%%%% Changer numérotation des pages %%%%%%%%%%%%%%%%%%%%%%%%%%%%
\pagestyle{fancy}
\cfoot{\thepage/\pageref{LastPage}} %%% numéroter page / total de pages
\renewcommand{\headrulewidth}{0pt} %%% empêcher qu'il y ait une ligne horizontale en haut
%%%%%%%%%%%% Ne pas numéroter les pages %%%%%%%%%%%%%%%%%
%\pagestyle{empty}

%%%%%%%%%%%% Supprimer les alineas %%%%%%%%%%%%%%%%%%%%%%%%%%%%%%%%%%%%%
\setlength{\parindent}{0cm} 

%%%%%%%%%%%% Exercice %%%%%%%%%%%%%%%%%%%%%%%%%%%%%%%%%% 
\newcounter{exo}
\newenvironment{exo}[1][vide]
{\refstepcounter{exo}
	{\noindent \textcolor{DarkSlateBlue}{\textbf{Exercice \theexo.}}}
	\ifthenelse{\equal{#1}{vide}}{}{\textcolor{DarkSlateBlue}{(#1)}}
}{}

%%%%%%%%%%%% Partie %%%%%%%%%%%%%%%%%%%%%%%%%%%%%%%%%%%%
\newcounter{partie}
\newcommand\partie[1]{
	\stepcounter{partie}%
	{\bigskip\large\textbf{\DSB{\thepartie.~#1}}\bigskip}
	}

%%%%%%%%%%%% Separateur entre les exos %%%%%%%%%%%%%%%%%
\newcommand{\separationexos}{
	\bigskip
%	{\centering\hfill\DSB{\rule{0.4\linewidth}{1.2pt}}\hfill}\medskip
	}

%%%%%%%%%%%% Corrige %%%%%%%%%%%%%%%%%%%%%%%%%%%%%%%%%%% 
%\renewenvironment{comment}{\medskip\noindent \textcolor{BrickRed}{\textbf{Corrigé.}}}{}

%%%%%%%%%%%% Titre %%%%%%%%%%%%%%%%%%%%%%%%%%%%%%%%%%%%%%
\newcommand\titre[1]{\ \vspace{-1cm}
	
	\DSB{\rule{\linewidth}{1.2pt}}
	{\small Probabilités et statistiques continues avancées}
	\hfill {\small Université Paul Sabatier}
	
	{\small KMAXPP03}
	\hfill {\small Licence 3, Printemps 2023}\medskip
	\begin{center}
		{\Large\textbf{\DSB{#1}}}\vspace{-.2cm}
	\end{center}
	\DSB{\rule{\linewidth}{1.2pt}}\medskip
}

%%%%%%%%%%%%%%%%%%%%%%%%%%%%%%%%%%%%%%%%%%%%%%%%%%%%%%%%%%%%%%%%%%%%%%%
\begin{document}
%%%%%%%%%%%%%%%%%%%%%%%%%%%%%%%%%%%%%%%%%%%%%%%%%%%%%%%%%%%%%%%%%%%%%%%

\titre{TD 14 -- Révisions}

%%%%%%%%%%%%%%%%%%%%%%%%%%%%%%%%%%%%%%%%%%%%%%%%%%%
\partie{Toujours plus de convergences}
%%%%%%%%%%%%%%%%%%%%%%%%%%%%%%%%%%%%%%%%%%%%%%%%%%%

\begin{exo}
	Soit $(U_n)_{n\geq 1}$ une suite de v.a. indépendantes de loi uniforme sur $[0,1]$.
	Pour $n \geq 1$, on pose
	\[
	M_n \coloneqq \max \left( \frac{1}{\sqrt{U_1}},\dots,\frac{1}{\sqrt{U_n}} \right).
	\]
	\begin{enumerate}
		\item Calculer la fonction de répartition de $M_n$.
		%%
		\item Soit $p>0$. Déterminer les valeurs de $p$ telles que $M_n$ a un moment d'ordre $p$ fini.
		%%
		\item Montrer que $M_n/\sqrt{n}$ converge en loi quand $n \to \infty$ vers une variable aléatoire dont on précisera la fonction de répartition et la densité.
	\end{enumerate}
\end{exo}

\begin{comment}
\begin{enumerate}
\item Pour $x < 1$, on a $F_{M_n}(x) = 0$. Pour $x \geq 1$, on a
\[
F_{M_n}(x) 
= \Pp{\frac{1}{\sqrt{U_1}} \leq x}^n
= \Pp{U_1 \geq \frac{1}{x^2}}^n
= \left( 1 - \frac{1}{x^2} \right)^n.
\]
%%
\item Comme $M_n \geq 0$, on a 
\[
\Ec{M_n^p} 
= \int_0^\infty \Pp{M_n^p \geq y} \diff y
= \int_0^\infty \Pp{M_n \geq y^{1/p}} \diff y.
\]
Or on a , pour $y \geq 1$,
\[
\Pp{M_n \geq y^{1/p}} 
= 1- F_{M_n}(y^{1/p}) 
= 1 - \left( 1 - \frac{1}{y^{2/p}} \right)^n
\sim_{y\to\infty} \frac{n}{y^{2/p}},
\]
donc $\E[M_n^p] < \infty$ si et seulement si $p < 2$.
%%
\item On passe par les fonctions de répartition.
Pour $x \leq 0$, on a $F_{M_n / \sqrt{n}}(x) = 0$. Pour $x > 0$, on a, à partir d'un certain rang tel que $x \sqrt{n} \geq 1$,
\[
F_{M_n / \sqrt{n}}(x) 
= F_{M_n}(x\sqrt{n}) 
= \left( 1 - \frac{1}{n x^2} \right)^n
\xrightarrow[n\to\infty]{} e^{-1/x^2}.
\]
On pose $F(x) \coloneqq e^{-1/x^2} \1_{x>0}$ pour $x\in\R$. 
Alors $F$ est continue croissante et tend vers 0 en $-\infty$ et en 1 en $+\infty$ donc c'est la fonction de répartition d'une variable aléatoire $Z$, et donc $M_n/\sqrt{n}$ converge en loi vers $Z$.

En outre, on a $F_Z(x) = \int_{-\infty}^x p_Z(z) \diff z$, avec 
\[
p_Z(z) 
\coloneqq F_Z'(z) 
= \frac{2}{z^3} e^{-1/z^2} \1_{z>0},
\]
donc $Z$ admet $p$ pour densité.
\end{enumerate}
\end{comment}



%%%%%%
\separationexos
%%%%%%

\begin{exo}
	Soit $(A_n)_{n\geq1}$ une suite d'événements indépendants.
	Pour $n\geq 1$, on pose $a_n \coloneqq \P(A_n)$ et
	\[
	b_n \coloneqq \sum_{k=1}^n a_k
	\quad \text{et} \quad
	S_n \coloneqq \sum_{k=1}^n \1_{A_k}.
	\]
	On suppose que $b_n \to \infty$ quand $n\to\infty$.
	\begin{enumerate}
		\item Montrer que $S_n / b_n \to 1$ dans $L^2$ quand $n \to \infty$.
		%%
		\item Pour tout $k \geq 1$, on pose $n_k \coloneqq \inf \{ n \in\N : b_n \geq k^2 \}$.
		Montrer que $k^2 \leq b_{n_k} < k^2 + 1$ et que $(n_k)_{k\in\N}$ est strictement croissante.
		%%
		\item Montrer que $S_{n_k} / b_{n_k} \to 1$ presque sûrement quand $k \to \infty$.
		%%
		\item En déduire que $S_n / b_n \to 1$ presque sûrement quand $n \to \infty$.
	\end{enumerate}
\end{exo}


\begin{comment}
\begin{enumerate} 
\item On a
\[
\Ec{\left( \frac{S_n}{b_n} - 1 \right)^2}
= \Var \left( \frac{S_n}{b_n} \right)
= \sum_{k=1}^n \Var \left( \frac{\1_{A_k}}{b_n} \right)
\leq \frac{1}{b_n^2} \sum_{k=1}^n \Ec{\1_{A_k}^2}
= \frac{1}{b_n^2} \sum_{k=1}^n a_k
= \frac{1}{b_n}
\xrightarrow[n\to\infty]{} 0,
\]
donc cela montre la convergence souhaitée.
%%
\item On note tout d'abord que $n_k$ est bien défini car $b_n \to \infty$. 
Par définition, $b_{n_k} \geq k^2$ et $b_{n_k-1} < k^2$. Or $a_{n_k} \leq 1$, donc on a $b_{n_k} < k^2 +1$.
En particulier, $b_{n_k} < (k+1)^2$ donc $n_k < n_{k+1}$.
%%
\item Soit $\varepsilon > 0$.
On a, par l'inégalité de Bienaymé-Tchebychev
\[
\Pp{\abs{ \frac{S_{n_k}}{b_{n_k}} -1} \geq \varepsilon}
\leq \frac{1}{\varepsilon^2} \Var \left( \frac{S_{n_k}}{b_{n_k}} \right)
\leq \frac{1}{\varepsilon^2} \frac{1}{b_{n_k}}
\leq \frac{1}{\varepsilon^2 k^2},
\]
qui est une suite sommable en $k$.
Par Borel--Cantelli, on en déduit que presque sûrement, à partir d'un certain rang, 
\[
\abs{ \frac{S_{n_k}}{b_{n_k}} -1} < \varepsilon
\]
et donc, presque sûrement,
\[
1- \varepsilon
\leq \liminf_{k\to\infty} \frac{S_{n_k}}{b_{n_k}} 
\leq \limsup_{k\to\infty} \frac{S_{n_k}}{b_{n_k}}
\leq 1+ \varepsilon.
\]
On conclut en prenant une suite $(\varepsilon_m)_{m\in\N} \downarrow 0$.
%%
\item Soit $n \geq 1$. 
Il existe un unique $k \geq 1$ tel que $n_k \leq n < n_{k+1}$.
Comme $(S_n)_{n\geq 1}$, $(b_n)_{n\geq 1}$ et $(n_k)_{k\geq 1}$ sont croissantes, on a
\[
\frac{S_{n_k}}{b_{n_{k+1}}} 
\leq \frac{S_n}{b_n} 
\leq \frac{S_{n_{k+1}}}{b_{n_k}} 
\]
et, comme $k^2 \leq b_{n_k} < k^2 + 1$,
\[
\frac{S_{n_k}}{b_{n_k}} \frac{k^2}{k^2+1}
\leq \frac{S_n}{b_n} 
\leq \frac{S_{n_{k+1}}}{b_{n_{k+1}}} \frac{k^2+1}{k^2}.
\]
Sur l'événement de convergence de $(S_{n_k}/b_{n_k})_{k\geq1}$ vers 1, on a donc $S_n/b_n \to 1$ quand $n \to \infty$ (car alors $k \to \infty$). 
On a donc montré que $S_n / b_n \to 1$ presque sûrement quand $n \to \infty$.
\end{enumerate}
\end{comment}


%%%%%%
\separationexos
%%%%%%


\begin{exo}[Convergence en loi de couples de v.a.]
	Soient $(X_n)_{n\geq1}$, $(Y_n)_{n\geq1}$ deux suites de v.a. réelles convergeant en loi vers des v.a. $X$ et $Y$.
	On a vu au TD11 qu'en général on n'a pas convergence en loi de $(X_n+Y_n)_{n\geq 1}$ vers $X+Y$. On va voir ci-dessous deux cas où c'est vrai.
	\begin{enumerate}
		\item On suppose ici que $X_n$ et $Y_n$ sont indépendantes pour tout $n\geq1$ et que $X$ et $Y$ sont indépendantes. En utilisant le théorème de Lévy, montrer que $(X_n+Y_n)_{n\geq 1}$ converge en loi vers $X+Y$.
		%%
		\item On suppose que $Y$ est constante p.s. (mais plus l'indépendance). On veut montrer que $(X_n+Y_n)_{n\geq 1}$ converge en loi vers $X+Y$.
		\begin{enumerate}
			\item Montrer qu'il suffit de montrer $\Ec{f(X_n+Y_n)} \to \Ec{f(X+Y)}$ pour tout fonction $f \colon \R \to \R$ lipschitzienne bornée. On considère à présent $f$ une telle fonction.
			%%
			\item Soit $\varepsilon > 0$. Montrer que
			\[
				\abs{\Ec{f(X_n+Y_n)} - \Ec{f(X_n+Y)}}  
				\leq 2 \norme{f}_\infty \P(\abs{Y_n-Y} > \varepsilon)
				+ L \varepsilon,
			\]
			où $L$ est la constante de Lipschitz de $f$. 
			%%
			\item En déduire que $\abs{\Ec{f(X_n+Y_n)} - \Ec{f(X_n+Y)}} \to 0$.
			%%
			\item Montrer que $\abs{\Ec{f(X_n+Y)} - \Ec{f(X+Y)}} \to 0$.
			%%
			\item Conclure.
		\end{enumerate}
	\end{enumerate}
\end{exo}

%%%%%%
\separationexos
%%%%%%

\begin{exo}
	On reprend l'exercice 3 du DST1. On observe des v.a. réelles $X_1,\dots,X_n$ i.i.d. dont la loi appartient à la famille $(P_\theta)_{\theta \in [1,4]}$, où $P_\theta$ est définie par
	\[
	P_\theta(\diff x) = \frac{3 \theta^3}{x^4} \1_{[\theta,\infty[}(x) \diff x.
	\]
	On rappelle que si $X_1$ a pour loi $P_\theta$ pour un certain $\theta \in [1,4]$, alors $\E[X_1] = \frac{3\theta}{2}$ et $\Var(X_1) = \frac{3\theta}{4}$.
	
	Soit $\alpha \in {]}0,1{[}$. 
	En utilisant le théorème central limite, construire un intervalle de confiance asymptotique pour $\theta$ de niveau $1-\alpha$. 
\end{exo}


%%%%%%%%%%%%%%%%%%%%%%%%%%%%%%%%%%%%%%%%%%%%%%%%%%%
\partie{Compléments}
%%%%%%%%%%%%%%%%%%%%%%%%%%%%%%%%%%%%%%%%%%%%%%%%%%%


\begin{exo}[Lemme de Scheffé]
	Soit $(X_n)_{n\geq 1}$ une suite de v.a. réelles continues. Notons $f_n$ la densité de $X_n$. On suppose que $(f_n)_{n\geq 1}$ converge Lebesgue-presque partout vers une fonction $f$ telle que $\int_\R f(x) \diff x =1$.
	\begin{enumerate}
		\item Montrer que $\abs{f-f_n} = 2(f-f_n)_+ + f_n - f$.
		%%
		\item On note $\lambda$ la mesure de Lebesgue sur $\R$.
		Montrer que 
		\[
			\int_\R \abs{f-f_n} \diff \lambda 
			= 2 \int_\R (f-f_n)_+ \diff \lambda.
		\]
		\item En déduire que $(f_n)_{n\geq 1}$ converge vers $f$ dans $L^1(\R,\lambda)$.
		%%
		\item En déduire que $(X_n)_{n\geq 1}$ converge en loi vers une v.a. de densité $f$.
	\end{enumerate}
\end{exo}


%%%%%%
\separationexos
%%%%%%

\begin{exo}
	Soit $(p_n)_{n\geq1}$ une suite de réels dans ${]}0,1{[}$.
	Pour chaque $n \geq 1$, soit $X_n$ une v.a. de loi Binomiale$(n,p_n)$.
	Dans ce cours, on a déjà vu que :
	\begin{itemize}
		\item Si $p_n = p \in {]}0,1{[}$ est fixé, alors par le TCL
		\[
			\frac{X_n - n p}{\sqrt{n^p(1-p)}}
			\xrightarrow[n\to\infty]{\text{loi}} 
			\cN(0,1). 
		\]
		%% 
		\item Si $np_n \to \lambda \in \R_+^*$, alors 
		\[
			X_n
			\xrightarrow[n\to\infty]{\text{loi}} 
			\mathrm{Poisson}(\lambda). 
		\]
	\end{itemize}
	Ici on suppose que $p_n \to 0$ mais $np_n \to \infty$ quand $n\to\infty$.
	On va montrer que
	\[
	Z_n \coloneqq \frac{X_n - n p}{\sqrt{np_n}}
	\xrightarrow[n\to\infty]{\text{loi}} 
	\cN(0,1). 
	\]
	\begin{enumerate}
		\item Montrer que, pour tout $\theta \in \R$, 
		\[
		\phi_{Z_n}(\theta) 
		= \left( e^{-\ii \theta \sqrt{p_n/n}} \left( 1 + p_n \left(e^{\ii \theta /\sqrt{np_n}} - 1 \right) \right) \right)^n.
		\]
		\item Montrer que, quand $n \to \infty$,
		\[
		e^{-\ii \theta \sqrt{p_n/n}} \left( 1 + p_n \left(e^{\ii \theta /\sqrt{np_n}} - 1 \right) \right)
		= 1 - \frac{\theta^2}{2n} + o \left( \frac{1}{n} \right).
		\]
		\item Conclure.
	\end{enumerate}
\end{exo}


%%%%%%%
%\separationexos
%%%%%%%
%
%\begin{exo}
%	Soit $(X_n)_{n\geq 1}$ une suite de variables aléatoires indépendantes et identiquement ditribuées définies sur $(\Omega, \cA,\P)$.
%	On suppose que $X_1 \in L^2(\P)$, $\E[X_1] = 0$ et $\E[X_1^2] = \sigma^2 < \infty$.
%	On fixe un réel $\alpha > 0$ et, pour $n \geq 1$, on pose
%	\[
%	S_n^{(\alpha)} 
%	\coloneqq \sum_{i=1}^n \frac{X_i}{i^\alpha}.
%	\]
%	\begin{enumerate}
%		\item On suppose que $\alpha > 1/2$. Montrer que $(S_n^{(\alpha)})_{n\geq 1}$ converge dans $L^2$.
%		
%		\emph{Indication.} On pourra utiliser le critère de Cauchy.
%		%%
%		\item On suppose maintenant que $\alpha = 1/2$.
%		\begin{enumerate}
%			\item 
%			%Montrer que, pour tout $\varepsilon>0$, il existe $\eta > 0$ tel que, pour tout $\xi \in [-\eta, \eta]$, on ait
%			%\[
%			%\abs{\ln \phi_{X_1} (\xi) + \frac{\sigma^2 \xi^2}{2}}
%			%\leq \varepsilon \xi^2.
%			%\]
%			Montrer que, pour tout $\varepsilon>0$, il existe $\eta > 0$ tel que, pour tout $\xi \in [-\eta, \eta]$, on ait 
%			\[
%			\abs{ \phi_{X_1} (\xi) \exp\left( \frac{\sigma^2 \xi^2}{2} \right) -1}
%			\leq \varepsilon \xi^2.
%			\]
%			%Montrer que, pour tout $\varepsilon>0$, il existe $\eta > 0$ tel que, pour tout $\xi \in [-\eta, \eta]$, on ait $\abs{z(\xi)} \leq \varepsilon \xi^2$ et
%			%\[
%			%\phi_{X_1} (\xi) \exp\left( \frac{\sigma^2 \xi^2}{2} \right)
%			%= \exp(z(\xi)).
%			%\]
%			%%
%			\item En déduire que la suite
%			\[
%			\left( \frac{S_n^{(1/2)}}{\sqrt{\ln n}} \right)_{n\geq 1}
%			\]
%			converge en loi vers une limite dont on précisera la loi.
%		\end{enumerate}
%	\end{enumerate}
%\end{exo}
%
%\begin{comment}
%\begin{enumerate}
%\item On utilise le critère de Cauchy : soit $1 \leq m \leq n$, on a
%\[
%\Ec{\left( \sum_{i=m}^n \frac{X_i}{i^\alpha} \right)^2}
%= \sum_{i=m}^n \Ec{\left( \frac{X_i}{i^\alpha} \right)^2}
%= \sigma^2 \sum_{i=m}^n \frac{1}{i^{2\alpha}},
%\]
%en utilisant l'indépendance et le fait que les $X_i$ sont centrées.
%Pour $\alpha > 1/2$, la série $\sum_{i\geq 1} i^{-2\alpha}$ converge donc 
%\[
%\sup_{n \geq m \geq N} \Ec{\left( \sum_{i=m}^n \frac{X_i}{i^\alpha} \right)^2}
%\xrightarrow[N\to\infty]{} 0,
%\]
%ce qui montre que $(S_n^{(\alpha)})_{n\geq 1}$ converge dans $L^2$ (car $L^2$ est complet).
%%%
%\item Pour cette question, on utilise une méthode similaire à la démonstration du TCL.
%\begin{enumerate}
%\item 
%%On sait que $X_1$ admet un moment d'ordre 2, donc $\phi_{X_1}$ est 2 fois dérivable en 0 et $\phi_{X_1}'(0) = i \E[X_1] = 0$ et $\phi_{X_1}''(0) = - \E[X_1^2] = -\sigma^2$.
%%Avec un développement de Taylor-Young en 0, on obtient
%%\[
%%\phi_{X_1} (\xi) 
%%= 1 - \frac{\sigma^2 \xi^2}{2} + \petito{\xi^2}
%%= \exp \left( - \frac{\sigma^2 \xi^2}{2} + \petito{\xi^2} \right),
%%\]
%%quand $\xi \to 0$.
%%Ainsi, on a
%%\[
%%\ln \phi_{X_1} (\xi) + \frac{\sigma^2 \xi^2}{2}
%%= \petito{\xi^2},
%%\]
%%quand $\xi \to 0$, ce qui montre le résultat souhaité.
%%
%%\emph{Remarque.} On utilise le développement de $\ln$ en 1, mais on a besoin de l'avoir pour un voisinage de $1$ dans $\C$. Il est facile de définir la fonction $\ln$ sur le disque ouvert de centre $1$ et de rayon 1 par la série entière
%%\[
%%\ln(1+z) = \sum_{n\geq 0} \frac{(-1)^n}{n} z^n.
%%\]
%%Il faut ensuite vérifier que c'est bien la réciproque de $\exp$ définie sur $\ln(D(1,1))$, qui est ouvert contenant 0 donc contenant $-\sigma^2 \xi^2/2$ si $\eta$ est suffisamment petit.
%%Le fait que $\ln(1+z) = z + o(z)$ quand $z \to 0$ découle du fait que la série entière ci-dessus a bien un rayon de convergence $>0$.
%%%
%On sait que $X_1$ admet un moment d'ordre 2, donc $\phi_{X_1}$ est 2 fois dérivable en 0 et $\phi_{X_1}'(0) = i \E[X_1] = 0$ et $\phi_{X_1}''(0) = - \E[X_1^2] = -\sigma^2$.
%Avec un développement de Taylor-Young en 0, on obtient
%\[
%\phi_{X_1} (\xi) 
%= 1 - \frac{\sigma^2 \xi^2}{2} + \petito{\xi^2}
%= \exp \left( - \frac{\sigma^2 \xi^2}{2} \right) \left(1 + \petito{\xi^2}\right),
%\]
%quand $\xi \to 0$. Cela correspond au résultat demandé.
%%%
%\item 
%%Soit $\xi \in \R$, on a
%%\[
%%\phi_{S_n^{(1/2)}/\sqrt{\ln n}} (\xi) 
%%= \Ec{\exp \left( i \sum_{k=1}^n \frac{X_k \xi}{k^{1/2} \sqrt{\ln n}} \right)}
%%= \prod_{k=1}^n \phi_{X_1} \left( \frac{\xi}{\sqrt{k \ln n}} \right).
%%\]
%%Soit $\varepsilon > 0$ et $\eta >0$ donné par la question précédente (entre autre $\eta$ est suffisamment petit pour que $\lvert \phi_{X_1} (\xi) -1 \rvert < 1$ pour $\abs{\xi} \leq \eta$ et donc que $\ln \phi_{X_1} (\xi)$ soit bien défini).
%%
%%On voudrait passer au logarithme, mais on ne sait pas si $\ln (\phi_{S_n^{(1/2)}/\sqrt{\ln n}} (\xi))$ est bien défini.
%%On considère plutôt
%%\begin{align*}
%%\phi_{S_n^{(1/2)}/\sqrt{\ln n}} (\xi) 
%%\exp \left( \sum_{k=1}^n \frac{\sigma^2 \xi^2}{2 k \ln n} \right)
%%& = \prod_{k=1}^n \phi_{X_1} \left( \frac{\xi}{\sqrt{k \ln n}} \right)
%%	\exp \left( \frac{\sigma^2 \xi^2}{2 k \ln n} \right) \\
%%& = \prod_{k=1}^n e^{z_k},
%%\quad \text{avec} \quad
%%z_k \coloneqq \ln \phi_{X_1} \left( \frac{\xi}{\sqrt{k \ln n}} \right) + \frac{\sigma^2 \xi^2}{2 k \ln n},
%%\end{align*}
%%où $z_k$ est bien défini pour $n$ suffisamment grand tel que $\lvert \xi / \sqrt{\ln n} \rvert \leq \eta$ (car alors $\lvert \xi / \sqrt{k\ln n} \rvert \leq \eta$ pour tout $k \geq 1$).
%%Cela nous dit aussi que $\lvert z_k \rvert \leq \varepsilon \xi^2 / (k \ln n)$ et donc
%%\begin{align*}
%%\limsup_{n\to \infty} \abs{\sum_{k=1}^n z_k}
%%\leq \limsup_{n\to \infty} \sum_{k=1}^n \frac{\varepsilon \xi^2}{k \ln n}
%%= \varepsilon \xi^2
%%\end{align*}
%%ce qui nous donne
%%\begin{align*}
%%\limsup_{n\to \infty} \abs{ \phi_{S_n^{(1/2)}/\sqrt{\ln n}} (\xi) 
%%\exp \left( \sum_{k=1}^n \frac{\sigma^2 \xi^2}{2 k \ln n} \right)
%%- 1}
%%& = \limsup_{n\to \infty}  \abs{\exp \left( \sum_{k=1}^n z_k \right) -1} \\
%%& \leq \max \left( e^{\varepsilon \xi^2} - 1, 1-e^{-\varepsilon \xi^2} \right).
%%\end{align*}
%%En faisant tendre $\varepsilon \to 0$, on obtient 
%%\begin{align*}
%%\phi_{S_n^{(1/2)}/\sqrt{\ln n}} (\xi) 
%%\exp \left( \sum_{k=1}^n \frac{\sigma^2 \xi^2}{2 k \ln n} \right)
%%\xrightarrow[n\to\infty]{} 1.
%%\end{align*}
%%Et comme $\exp (\sum_{k=1}^n \frac{\sigma^2 \xi^2}{2 k \ln n}) \to \exp(\sigma^2 \xi^2 / 2)$, on a finalement
%%\begin{align*}
%%\phi_{S_n^{(1/2)}/\sqrt{\ln n}} (\xi) 
%%\xrightarrow[n\to\infty]{} e^{-\sigma^2 \xi^2/2}
%%\end{align*}
%%donc, par le théorème de Lévy faible, $S_n^{(1/2)}/\sqrt{\ln n}$ converge en loi vers $\cN(0,\sigma^2)$.
%
%Soit $\xi \in \R$ fixé dans la suite, on a
%\[
%\phi_{S_n^{(1/2)}/\sqrt{\ln n}} (\xi) 
%= \Ec{\exp \left( i \sum_{k=1}^n \frac{X_k \xi}{k^{1/2} \sqrt{\ln n}} \right)}
%= \prod_{k=1}^n \phi_{X_1} \left( \frac{\xi}{\sqrt{k \ln n}} \right).
%\]
%On écrit
%\begin{align*}
%\phi_{S_n^{(1/2)}/\sqrt{\ln n}} (\xi) 
%\exp \left( \sum_{k=1}^n \frac{\sigma^2 \xi^2}{2 k \ln n} \right)
%& = \prod_{k=1}^n \phi_{X_1} \left( \frac{\xi}{\sqrt{k \ln n}} \right)
%\exp \left( \frac{\sigma^2 \xi^2}{2 k \ln n} \right) \\
%& = \prod_{k=1}^n (1+z_{k,n}),
%\quad \text{avec} \quad
%z_{k,n} \coloneqq \phi_{X_1} \left( \frac{\xi}{\sqrt{k \ln n}} \right)
%\exp \left( \frac{\sigma^2 \xi^2}{2 k \ln n} \right) - 1.
%\end{align*}
%Montrons que $\prod_{k=1}^n (1+z_{k,n}) \to 1$.
%Pour cela, soit $\varepsilon > 0$ et $\eta >0$ donné par la question précédente.
%Pour $n$ suffisamment grand tel que $\lvert \xi / \sqrt{\ln n} \rvert \leq \eta$, on a $\lvert \xi / \sqrt{k\ln n} \rvert \leq \eta$ pour tout $k \geq 1$ et donc $\abs{z_{k,n}} \leq \varepsilon \xi^2 / (k\ln n)$.
%En développant le produit, en appliquant l'inégalité triangulaire puis en refactorisant le produit on a
%\[
%\abs{ \prod_{k=1}^n (1+z_{k,n}) - 1}
%\leq \prod_{k=1}^n \left(1+\abs{z_{k,n}} \right) - 1
%= \exp \left( \sum_{k=1}^n \ln\left(1+\abs{z_{k,n}} \right) \right) - 1.
%\]
%Comme $\ln (1+x) \leq x$ pour $x \geq 0$, on en déduit
%\[
%\abs{ \prod_{k=1}^n (1+z_{k,n}) - 1}
%\leq \exp \left( \sum_{k=1}^n \abs{z_{k,n}} \right) - 1
%\leq \exp \left( \frac{\varepsilon \xi^2}{\ln n} 
%\sum_{k=1}^n \frac{1}{k} \right) - 1.
%\]
%On a donc 
%\begin{align*}
%\limsup_{n\to \infty} \abs{ \prod_{k=1}^n (1+z_{k,n}) - 1}
%\leq e^{\varepsilon \xi^2} - 1,
%\end{align*}
%qui est aussi petit que l'on veut pour $\varepsilon$ petit.
%Donc $\prod_{k=1}^n (1+z_{k,n}) \to 1$ et ainsi 
%\begin{align*}
%\phi_{S_n^{(1/2)}/\sqrt{\ln n}} (\xi) 
%\exp \left( \sum_{k=1}^n \frac{\sigma^2 \xi^2}{2 k \ln n} \right)
%\xrightarrow[n\to\infty]{} 1.
%\end{align*}
%Mais, d'autre part, on a $\exp (\sum_{k=1}^n \frac{\sigma^2 \xi^2}{2 k \ln n}) \to \exp(\sigma^2 \xi^2 / 2)$, donc finalement
%\begin{align*}
%\phi_{S_n^{(1/2)}/\sqrt{\ln n}} (\xi) 
%\xrightarrow[n\to\infty]{} e^{-\sigma^2 \xi^2/2}.
%\end{align*}
%Par le théorème de Lévy faible, cela montre que $S_n^{(1/2)}/\sqrt{\ln n}$ converge en loi vers $\cN(0,\sigma^2)$.
%
%\emph{Remarque}. Dans la solution ci-dessus, on se passe du logarithme complexe, qui n'a pas encore été vu en cours. Son utilisation rendrait le calcul plus naturel, même si elle oblige à vérifier que l'on est bien sur son domaine de définition.
%\end{enumerate}
%\end{enumerate}
%\end{comment}
%
%
%
%%%%%%%
%\separationexos
%%%%%%%
%
%
%
%
%
%\begin{exo}[Théorème de Portmanteau]
%	Soit $X, X_0, X_1, \dots$ des variables al\'eatoires à valeurs dans un espace métrique $(E,d)$ muni de sa tribu borélienne, définies sur $(\Omega,\cA,\P)$.
%	On veut montrer que les propriétés suivantes sont équivalentes :
%	\begin{enumerate}[label=(\roman*)]
%		\item la suite $(X_n)_{n\in \N}$ converge en loi vers $X$ ;
%		%%
%		\item pour tout $f \colon E \to \R$ lipschitzienne bornée, on a $\Ec{f(X_n)} \to \Ec{f(X)}$ quand $n \to \infty$ ;
%		%%
%		\item pour tout fermé $F \subset E$, on a $\limsup_{n\to\infty} \Pp{X_n \in F} \leq \Pp{X \in F}$ ;
%		%%
%		\item pour tout ouvert $G \subset E$, on a $\liminf_{n\to\infty} \Pp{X_n \in G} \geq \Pp{X \in F}$ ;
%		%%
%		\item pour tout borélien $A \subset E$ tel que $\Pp{X \in \partial A} = 0$, on a $\Pp{X_n \in A} \to \Pp{X \in A}$ quand $n \to \infty$.
%	\end{enumerate}
%	Ce résultat est appelé \emph{théorème de Portmanteau}, mais il n'y a pas de mathématicien portant le nom de Portmanteau : il s'agit d'un canular possiblement initié par Billingsley qui, dans l'un de ses livres, attribue le théorème à Jean-Pierre Portmanteau, 
%	en faisant référence à un article de 1915 des Annales de l'Université Felletin ayant pour titre « Espoir pour l'ensemble vide? ».
%	%qui l'aurait publié en 1915 dans les Annales de l'Université Felletin, sous le titre « Espoir pour l'ensemble vide? ».
%	%%
%	\begin{enumerate}
%		\item Montrer les implications faciles : (i) $\Rightarrow$ (ii), (iii) $\Leftrightarrow$ (iv) puis (iii) + (iv) $\Rightarrow$ (v).
%		%%
%		\item Montrer que (ii) $\Rightarrow$ (iii).
%		%%
%		\item En considérant tout d'abord $f \in \cC_b(E)$ positive et en utilisant que $\E[f(X_n)] = \int_0^\infty \Pp{f(X_n) > y} \diff y$, montrer que (v) $\Rightarrow$ (i).
%	\end{enumerate}
%\end{exo}
%
%\begin{comment}
%\begin{enumerate} 
%\item (i) $\Rightarrow$ (ii) : la convergence en loi implique la convergence $\Ec{f(X_n)} \to \Ec{f(X)}$ si $f$ est continue bornée donc en particulier si $f$ est lipschitzienne bornée.
%
%(iii) $\Leftrightarrow$ (iv) : il suffit de passer au complémentaire.
%
%(iii) + (iv) $\Rightarrow$ (v) : on rappelle que $\partial A = \overline{A} \setminus \mathring{A}$ et donc, si $\Pp{X \in \partial A} = 0$, on a
%\[
%\Pp{X \in A} = \Pp{X \in \overline{A}} = \Pp{X \in \mathring{A}}.
%\]
%On a ainsi, par (iv),
%\[
%\liminf_{n\to\infty} \Pp{X_n \in A}
%\geq \liminf_{n\to\infty} \Pp{X_n \in \mathring{A}}
%\geq \Pp{X \in \mathring{A}} 
%= \Pp{X \in A}
%\]
%et, par (iii),
%\[
%\limsup_{n\to\infty} \Pp{X_n \in A}
%\leq \limsup_{n\to\infty} \Pp{X_n \in \overline{A}}
%\leq \Pp{X \in \overline{A}}
%= \Pp{X \in A},
%\]
%ce qui montre que $\Pp{X_n \in A} \to \Pp{X \in A}$.
%%%
%\item Supposons (ii). Soit $F$ un fermé.
%On pose, pour $k \geq 1$,
%\[
%f_k \colon x \in E \mapsto 0 \vee (1 - k d(x,F)),
%\]
%qui est bornée par $1$ et $k$-lipschitzienne.
%Donc, par (ii), on a $\Ec{f_k(X_n)} \to \Ec{f_k(X)}$ pour tout $k \geq 1$.
%Soit $k \geq 1$, on a $f_k \geq \1_F$, donc 
%\[
%\limsup_{n\to\infty} \Pp{X_n \in F} 
%\leq \limsup_{n\to\infty} \Ec{f_k(X_n)}
%= \Ec{f_k(X)}
%\]
%Or $f_k \downarrow \1_F$ quand $k \to \infty$, donc, par convergence dominée, $\Ec{f_k(X)} \to \Pp{X \in F}$.
%donc en prenant $k \to \infty$ dans l'inégalité ci-dessus, on obtient
%\[
%\limsup_{n\to\infty} \Pp{X_n \in F} 
%\leq \Pp{X \in F}.
%\]
%%%
%\item Soit $f \in \cC_b(E)$, on peut supposer $f$ positive (quitte à considérer $f^+$ puis $f^-$).
%On a donc, pour $n \in \N$, 
%\[
%\E[f(X_n)] 
%= \int_0^\infty \Pp{f(X_n) > y} \diff y
%= \int_0^M \Pp{X_n \in \{f> y\}} \diff y,
%\]
%en posant $M \coloneqq \norme{f}_\infty$. On a aussi la même relation avec $X$ au lieu de $X_n$.
%
%Montrons que $\Pp{X \in \partial\{f> y\}} = 0$ pour presque tout $y\in\R$.
%Comme $f$ est continue, $\{f> y\}$ est ouvert et $\{f \geq y\}$ est un fermé contenant $\{f> y\}$ donc $\partial \{f> y\} \subset \{f= y\}$.
%En outre, on a, par Fubini-Tonelli,
%\begin{align*}
%\int_\R \Pp{X \in \{f = y\}} \diff y
%& = \int_\R \int_\R \1_{x \in \{f = y\}} \diff P_X (x) \diff y
%= \int_\R \int_\R \1_{f(x) = y} \diff y \diff P_X (x) \\
%& = \int_\R \lambda(\{f(x)\}) \diff P_X (x) 
%= 0,
%\end{align*}
%où $\lambda$ est la mesure de Lebesgue sur $\R$.
%Ainsi $y \mapsto \Pp{X \in \{f = y\}}$ est une fonction positive d'intégrale nulle, donc elle est nulle presque partout.
%
%Pour presque tout $y\in\R$, on a $\Pp{X \in \partial\{f> y\}} = 0$ donc, par (v), on a
%\[
%\Pp{X_n \in \{f> y\}} \xrightarrow[n\to\infty]{} \Pp{X \in \{f> y\}}.
%\]
%En outre, on peut dominer par $1$ qui est intégrable sur $[0,M]$ et on obtient par convergence dominée
%\[
%\E[f(X_n)] 
%= \int_0^M \Pp{X_n \in \{f> y\}} \diff y
%\xrightarrow[n\to\infty]{} \int_0^M \Pp{X \in \{f> y\}} \diff y
%= \E[f(X)].
%\]
%\end{enumerate}
%\end{comment}


%\begin{exo}[Convergence en loi de couples de v.a.]
%	Soient $(X_n)_{n\geq1}$, $(Y_n)_{n\geq1}$ deux suites de v.a. réelles convergeant en loi vers des v.a. $X$ et $Y$.
%	\begin{enumerate}
%		\item On suppose que les variables $X_n$ et $Y_n$ sont indépendantes pour tout $n\geq1$ et que les variables $X$ et $Y$ sont indépendantes. Montrer que $(X_n,Y_n)\to(X,Y)$ en loi.
%		%%
%		\item On suppose à présent que $Y$ est constante p.s. Montrer que $(X_n,Y_n)\to(X,Y)$ en loi.
%	\end{enumerate}
%\end{exo}
%
%\begin{comment}
%\begin{enumerate}
%\item D'après le théorème de Lévy, il suffit de montrer que $\phi_{(X_n,Y_n)}(t,t')\to \phi_{(X,Y)}(t,t')$ pour tout $(t,t')\in\R^2$. Et l'on a par indépendance, 
%\[
%\phi_{(X_n,Y_n)}(t,t')
%=\phi_{X_n}(t)\phi_{Y_n}(t')\to\phi_{X}(t)\phi_{Y}(t')
%=\phi_{(X,Y)}(t,t').
%\] 
%%%
%\item Il n'est pas vrai en général que $(X_n,Y_n)\to(X,Y)$ en loi. En effet, considérons les variables aléatoires $X_n=Z=Y_n$ pour tout $n\geq1$, avec $Z$ gaussienne centrée. 
%La variable $Z$ étant symétrique, on a $X_n\to -Z$ en loi. Si $(X_n,Y_n)\to(-Z,Z)$ en loi, alors $X_n+Y_n\to -Z+Z$ en loi (car la fonction $(x,y)\mapsto x+y$ est continue), c'est-\`a-dire $2Z=0$ en loi, ce qui est absurde. 
%%%
%\item Il suffit de montrer que $\E[F(X_n,Y_n)]\to\E[F(X,Y)]$ pour une fonction $F$ continue \`a support compact. Soit $a\in\R$ tel que $Y=a$ p.s. 
%On a alors $Y_n\to a$ en probabilité (résultat important \`a savoir prouver). 
%Et 
%\begin{align*}
%|\E[F(X_n,Y_n)]-\E[F(X,a)]|
%& \leq |\E[F(X_n,Y_n)]-\E[F(X_n,a)]| + |\E[F(X_n,a)]-\E[F(X,a)]|.
%\end{align*}
%La fonction $x\in\R\mapsto F(x,a)$ est continue et bornée donc $|\E[F(X_n,a))-\E[F(X,a))|\to0$. De plus, la fonction $F$ est uniformément continue. Pour $\varepsilon>0$, on peut trouver $\delta$ tel que $|F(x,y)-F(x',y')|\leq\varepsilon$ pour $|x-x'|+|y-y'|\leq\delta$. Alors, on a 
%\begin{align*}
%|\E[F(X_n,Y_n)]-\E[F(X_n,a)]|
%& \leq \E[|F(X_n,Y_n)-F(X_n,a)|] \\ 
%& \leq \E[|F(X_n,Y_n)-F(X_n,a)|\1_{\{|Y_n-a|\geq\delta\}}]
%+\E[|F(X_n,Y_n)-F(X_n,a)|\1_{\{|Y_n-a|<\delta\}}] \\ 
%& \leq 2 \norme{F}_\infty \P(|Y_n-a|\geq\delta) + \varepsilon. 
%\end{align*} 
%Ainsi, on a
%\[
%\limsup_{n\to\infty} |\E[F(X_n,Y_n)]-\E[F(X_n,a)]|
%\leq \varepsilon
%\] 
%et ceci étant vrai pour tout $\varepsilon$, on en déduit que $|\E[F(X_n,Y_n)]-\E[F(X_n,a)]| \to 0$, puis le résultat.  
%\end{enumerate}
%\end{comment}




%\begin{exo} 
%	Soit $\lambda >1$ fixé et soit $(X_t)_{t \geq 0}$ une famille de variables aléatoires telle que, pour tout $t \geq 0$, $X_t$ suit une loi géométrique de paramètre $ 1-e^{-t}$, c'est-\`a-dire que
%	\[ \forall k \geq 1, \quad 
%	\Pp{X_t=k}= e^{-t} (1- e^{-t})^{k-1}.
%	\]
%	%%
%	Soit $(U_n)_{n \geq 1}$ une suite de variables aléatoires telle que $ \lambda U_n- \ln(n)$ converge en probabilité vers $-\ln(\mathcal{E})$ lorsque $n \to \infty$, o\`u $\mathcal{E}$ est une variable aléatoire exponentielle de paramètre $1$. 
%	On suppose de plus que les deux familles $(X_t)_{t \geq 0}$ et $(U_n)_{n \geq 1}$ sont indépendantes.
%	
%	Montrer que $X_{U_n}/n^{1/\lambda}$ converge en loi quand $n \to \infty$ vers une variable aléatoire exponentielle, dont le paramètre est aléatoire et vaut $\mathcal{E}^{1/\lambda}$.
%\end{exo}
%
%\begin{comment}
%On va utiliser les fonctions caractéristiques. Pour cela, calculons d'abord la fonction caractéristique de $X_t$:
%\[ 
%\Ec{e^{i u X_t}}=  \frac{1}{1-e^t (1- e^{-iu})}, \quad u \in \R.
%\]
%Par indépendence de $(X_t)_{t \geq 0}$ et $U_n$, on a donc
%\[ 
%\Ec{e^{i u X_{U_n}/ n^{1/  \lambda} }} 
%= \Ec{\frac{1}{1-e^{U_n} (1- e^{- i u / n^{1/ \lambda}})}}.
%\]
%En faisant un développant limité, on voit que, presque s\^urement, 
%\[
%\frac{1}{1-e^{U_n} (1- e^{- i u / n^{1/ \lambda}})} 
%\xrightarrow[n \rightarrow \infty]{} 
%\frac{\mathcal{E}^{1/ \lambda}}{\mathcal{E}^{1/ \lambda} - i u}.\]
%Or
%\[
%\forall s \geq 0, \quad \forall t \in \R, \quad 
%\abs{\frac{1}{1-e^s(1-e^{-it})}} \leq 1.
%\]
%D'après le théorème de convergence dominé, on en déduit que 
%\[ 
%\Ec {e^{i u X_{U_n}/ n ^{1/\lambda} }} 
%\xrightarrow[n\to\infty]{} 
%\Ec{\frac{\mathcal{E}^{1/\lambda}}{\mathcal{E}^{1/ \lambda}-iu}}.
%\]
%Le résultat en découle car, on a $x/(x-iu)= \Ec{e^{iu\textsf{Exp}(x)}}$, o\`u \textsf{Exp}$(x)$ est une variable aléatoire exponentielle de paramètre $x$.
%\end{comment}



\end{document}
