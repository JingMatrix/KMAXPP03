\documentclass[a4paper,11pt]{article}
\usepackage[utf8]{inputenc}
\usepackage[T1]{fontenc}
\usepackage{lmodern}

\usepackage{amsthm,amsmath,amsfonts,amssymb,bbm,mathrsfs,stmaryrd}
\usepackage{mathtools}
\usepackage{enumitem}
\usepackage{url}
\usepackage{dsfont}
\usepackage{appendix}
\usepackage{amsthm}
\usepackage[dvipsnames,svgnames]{xcolor}
\usepackage{graphicx}

\usepackage{fancyhdr,lastpage,titlesec,verbatim,ifthen}

\usepackage[colorlinks=true, linkcolor=black, urlcolor=black, citecolor=black]{hyperref}

\usepackage[french]{babel}

\usepackage{caption,tikz,subfigure}

\usepackage[top=2cm, bottom=2cm, left=2cm, right=2cm]{geometry}

%%%%%%%%%%%%%%%%%%%%%%%%%%%%%%%%%%%%%%%%%%%%%%%%%%%%%%%%%%%%%%%%%%%%%%%

%%%%%%%% Taille de la legende des images %%%%%%%%%%%%%%%%%%%%%%%%%%%%%%%
\renewcommand{\captionfont}{\footnotesize}
\renewcommand{\captionlabelfont}{\footnotesize}

%%%%%%%% Numeration des enumerates en romain et chgt de l'espace %%%%%%%
\setitemize[1]{label=$\rhd$, font=\color{NavyBlue},leftmargin=0.8cm}
\setenumerate[1]{font=\color{NavyBlue},leftmargin=0.8cm}
\setenumerate[2]{font=\color{NavyBlue},leftmargin=0.47cm}
%\setlist[enumerate,1]{label=(\roman*), font = \normalfont,itemsep=4pt,topsep=4pt} 
%\setlist[itemize,1]{label=\textbullet, font = \normalfont,itemsep=4pt,topsep=4pt} 

%%%%%%%% Pas d'espacement supplementaire avant \left et apres \right %%%
%%%%%%%% Note : pour les \Big(, utiliser \Bigl( \Bigr) %%%%%%%%%%%%%%%%%
\let\originalleft\left
\let\originalright\right
\renewcommand{\left}{\mathopen{}\mathclose\bgroup\originalleft}
\renewcommand{\right}{\aftergroup\egroup\originalright}

%%%%%%%%%%%%%%%%%%%%%%%%%%%%%%%%%%%%%%%%%%%%%%%%%%%%%%%%%%%%%%%%%%%%%%%

\newcommand{\N}{\mathbb{N}}
\newcommand{\Z}{\mathbb{Z}}
\newcommand{\Q}{\mathbb{Q}}
\newcommand{\R}{\mathbb{R}}
\newcommand{\C}{\mathbb{C}}
\newcommand{\T}{\mathbb{T}}
\renewcommand{\P}{\mathbb{P}}
\newcommand{\E}{\mathbb{E}}

\newcommand{\1}{\mathbbm{1}}

\newcommand{\cA}{\mathcal{A}}
\newcommand{\cB}{\mathcal{B}}
\newcommand{\cC}{\mathcal{C}}
\newcommand{\cE}{\mathcal{E}}
\newcommand{\cF}{\mathcal{F}}
\newcommand{\cG}{\mathcal{G}}
\newcommand{\cH}{\mathcal{H}}
\newcommand{\cI}{\mathcal{I}}
\newcommand{\cJ}{\mathcal{J}}
\newcommand{\cL}{\mathcal{L}}
\newcommand{\cM}{\mathcal{M}}
\newcommand{\cN}{\mathcal{N}}
\newcommand{\cP}{\mathcal{P}}
\newcommand{\cT}{\mathcal{T}}
\newcommand{\cU}{\mathcal{U}}

\newcommand{\Ec}[1]{\mathbb{E} \left[#1\right]}
\newcommand{\Pp}[1]{\mathbb{P} \left(#1\right)}
\newcommand{\Ppsq}[2]{\mathbb{P} \left(#1\middle|#2\right)}

\newcommand{\e}{\varepsilon}

\newcommand{\ii}{\mathrm{i}}
\DeclareMathOperator{\re}{Re}
\DeclareMathOperator{\im}{Im}
\DeclareMathOperator{\Arg}{Arg}

\newcommand{\diff}{\mathop{}\mathopen{}\mathrm{d}}
\DeclareMathOperator{\Var}{Var}
\DeclareMathOperator{\Cov}{Cov}
\newcommand{\supp}{\mathrm{supp}}

\newcommand{\abs}[1]{\left\lvert#1\right\rvert}
\newcommand{\abso}[1]{\lvert#1\rvert}
\newcommand{\norme}[1]{\left\lVert#1\right\rVert}
\newcommand{\ps}[2]{\langle #1,#2 \rangle}

\newcommand{\petito}[1]{o\mathopen{}\left(#1\right)}
\newcommand{\grandO}[1]{O\mathopen{}\left(#1\right)}

\newcommand\relphantom[1]{\mathrel{\phantom{#1}}}

\newcommand{\NB}[1]{{\color{NavyBlue}#1}}
\newcommand{\DSB}[1]{{\color{DarkSlateBlue}#1}}

%%%%%%%% Theorems styles %%%%%%%%%%%%%%%%%%%%%%%%%%%%%%%%%%%%%%%%%%%%%%
\theoremstyle{plain}
\newtheorem{theorem}{Theorem}[section]
\newtheorem{proposition}[theorem]{Proposition}
\newtheorem{lemma}[theorem]{Lemme}
\newtheorem{corollary}[theorem]{Corollaire}
\newtheorem{conjecture}[theorem]{Conjecture}
\newtheorem{definition}[theorem]{Définition}

\theoremstyle{definition}
\newtheorem{remark}[theorem]{Remarque}
\newtheorem{example}[theorem]{Exemple}
\newtheorem{question}[theorem]{Question}

%%%%%%%% Macros spéciales TD %%%%%%%%%%%%%%%%%%%%%%%%%%%%%%%%%%%%%%%%%%

%%%%%%%%%%%% Changer numérotation des pages %%%%%%%%%%%%%%%%%%%%%%%%%%%%
\pagestyle{fancy}
\cfoot{\thepage/\pageref{LastPage}} %%% numéroter page / total de pages
\renewcommand{\headrulewidth}{0pt} %%% empêcher qu'il y ait une ligne horizontale en haut
%%%%%%%%%%%% Ne pas numéroter les pages %%%%%%%%%%%%%%%%%
%\pagestyle{empty}

%%%%%%%%%%%% Supprimer les alineas %%%%%%%%%%%%%%%%%%%%%%%%%%%%%%%%%%%%%
\setlength{\parindent}{0cm}

%%%%%%%%%%%% Exercice %%%%%%%%%%%%%%%%%%%%%%%%%%%%%%%%%% 
\newcounter{exo}
\newenvironment{exo}[1][vide]
{\refstepcounter{exo}
	{\noindent \textcolor{DarkSlateBlue}{\textbf{Exercice \theexo.}}}
	\ifthenelse{\equal{#1}{vide}}{}{\textcolor{DarkSlateBlue}{(#1)}}
}{}

%%%%%%%%%%%% Partie %%%%%%%%%%%%%%%%%%%%%%%%%%%%%%%%%%%%
\newcounter{partie}
\newcommand\partie[1]{
	\stepcounter{partie}%
	{\bigskip\large\textbf{\DSB{\thepartie.~#1}}\bigskip}
}

%%%%%%%%%%%% Separateur entre les exos %%%%%%%%%%%%%%%%%
\newcommand{\separationexos}{
	\bigskip
	%	{\centering\hfill\DSB{\rule{0.4\linewidth}{1.2pt}}\hfill}\medskip
}

%%%%%%%%%%%% Corrige %%%%%%%%%%%%%%%%%%%%%%%%%%%%%%%%%%% 
\renewenvironment{comment}{\medskip\noindent \textcolor{BrickRed}{\textbf{Corrigé.}}}{}

%%%%%%%%%%%% Titre %%%%%%%%%%%%%%%%%%%%%%%%%%%%%%%%%%%%%%
\newcommand\titre[1]{\ \vspace{-1cm}

	\DSB{\rule{\linewidth}{1.2pt}}
	{\small Probabilités et statistiques continues avancées}
	\hfill {\small Université Paul Sabatier}

	{\small KMAXPP03}
	\hfill {\small Licence 3, Printemps 2023}\medskip
	\begin{center}
		{\Large\textbf{\DSB{#1}}}\vspace{-.2cm}
	\end{center}
	\DSB{\rule{\linewidth}{1.2pt}}\medskip
}

%%%%%%%%%%%%%%%%%%%%%%%%%%%%%%%%%%%%%%%%%%%%%%%%%%%%%%%%%%%%%%%%%%%%%%%
\begin{document}
%%%%%%%%%%%%%%%%%%%%%%%%%%%%%%%%%%%%%%%%%%%%%%%%%%%%%%%%%%%%%%%%%%%%%%%

\titre{TD 4 -- Vecteurs aléatoires et indépendance}

%%%%%%%%%%%%%%%%%%%%%%%%%%%%%%%%%%%%%%%%%%%%%%%%%%%
\partie{Vecteurs aléatoires, loi jointe, lois marginales}
%%%%%%%%%%%%%%%%%%%%%%%%%%%%%%%%%%%%%%%%%%%%%%%%%%%

\begin{exo} Soit $(X,Y)$ une v.a. de loi $\1_{[0,1]^2}(x,y) \diff x \diff y$.
	Déterminer la loi de $(XY,X/Y)$.

	\emph{Remarque.} On peut voir $(X,Y)$ comme un point tiré uniformément sur le carré $[0,1]^2$, mais aussi $X$ et $Y$ comme des v.a. indépendantes de loi uniforme sur $[0,1]$.
\end{exo}

\begin{comment}
Notons $P=XY$ et $Q=X/Y$.
Soit $f \colon \R^2 \to \R_+$ mesurables. On a, par le théorème de transfert,
\begin{align*}
	\Ec{f(P,Q)}
	= \Ec{f(XY,X/Y)}
	 & = \int_{\R^2} f(xy,x/y) \1_{[0,1]^2}(x,y) \diff x \diff y        \\
	 & = \int_{(\R_+^*)^2} f(xy,x/y) \1_{]0,1[^2}(x,y) \diff x \diff y,
\end{align*}
où on a utilisé que $\1_{[0,1]^2} = \1_{]0,1[^2}$ Lebesgue-p.p. pour ensuite restreindre l'intégrale à $x,y > 0$.
On veut à présent faire le changement de variables $(p,q) = (xy,x/y)$. Tentons de l'inverser : pour $x,y >0$, on a
\[
	\begin{cases}
		p = xy \\
		q = x/y
	\end{cases}
	\quad \Leftrightarrow \quad
	\begin{cases}
		pq = x^2 \\
		p/q = y^2
	\end{cases}
	\quad \Leftrightarrow \quad
	\begin{cases}
		x = \sqrt{pq} \\
		y = \sqrt{p/q}.
	\end{cases}
\]
Cela nous pousse à introduire
\[
	\varphi \colon (p,q) \in (\R_+^*)^2 \longmapsto \left( \sqrt{pq},{\textstyle\sqrt{p/q}} \right) \in (\R_+^*)^2
\]
de sorte qu'une fois le changement de variable avec $\varphi$ effectué on obtienne $f(p,q)$ dans l'intégrale.
Vérifions que $\varphi$ est un $\cC^1$-difféomorphisme :
\begin{itemize}
	\item $\varphi$ est bijective car son inverse est $(x,y) \in (\R_+^*)^2 \mapsto (xy,x/y) \in (\R_+^*)^2$.
	\item $\varphi$ a des dérivées partielles qui sont continues sur $(\R_+^*)^2$ et donc elle est $\cC^1$ : en effet, sa matrice jacobienne est
	      \[
		      \left(
		      \begin{array}{ccc}
				      \frac{\partial \varphi_1}{\partial p}(p,q) & \frac{\partial \varphi_1}{\partial q}(p,q) \\
				      \frac{\partial \varphi_2}{\partial p}(p,q) & \frac{\partial \varphi_2}{\partial q}(p,q)
			      \end{array}
		      \right)
		      = \left(
		      \begin{array}{ccc}
				      \frac{\sqrt{q}}{2\sqrt{p}} & \frac{\sqrt{p}}{2\sqrt{q}} \\
				      \frac{1}{2 \sqrt{pq}}      & -\frac{\sqrt{p}}{2q^{3/2}}
			      \end{array}
		      \right).
	      \]
	\item Le Jacobien de $\varphi$ est
	      \[
		      \mathrm{Jac}_\varphi(p,q) = - \frac{1}{4q} - \frac{1}{4q} = -\frac{1}{2q} \neq 0.
	      \]
\end{itemize}
Cela montre que $\varphi$ est un $\cC^1$-difféomorphisme, donc par formule de changement de variables
\begin{align*}
	\Ec{f(P,Q)}
	 & = \int_{(\R_+^*)^2} f(p,q) \1_{]0,1[^2}(\sqrt{pq},{\textstyle\sqrt{p/q}}) \cdot \abs{\mathrm{Jac}_\varphi(p,q)} \diff p \diff q.
\end{align*}
Notons que pour $p,q>0$,
\[
	(\sqrt{pq},{\textstyle\sqrt{p/q}}) \in {]}0,1{[}^2
	\quad \Leftrightarrow \quad
	\begin{cases}
		\sqrt{pq} < 1 \\
		\sqrt{p/q} < 1
	\end{cases}
	\quad \Leftrightarrow \quad
	\begin{cases}
		pq < 1 \\
		p/q < 1
	\end{cases}
	\quad \Leftrightarrow \quad
	\begin{cases}
		pq < 1 \\
		p < q.
	\end{cases}
\]
Donc
\begin{align*}
	\Ec{f(P,Q)}
	= \int_{(\R_+^*)^2} f(p,q) \1_{\{pq<1,\, p < q\}} \frac{1}{2q} \diff p \diff q
	= \int_{\R^2} f(p,q) \1_{\{pq<1,\, 0 < p < q\}} \frac{1}{2q} \diff p \diff q.
\end{align*}
Donc $(P,Q)$ est une v.a. continue de densité
\[
	(p,q) \in \R^2 \longmapsto \frac{1}{2q} \1_{\{pq<1,\, 0 < p < q\}}.
\]\newpage
\end{comment}

%%%%%%
\separationexos
%%%%%%

\begin{exo} On tire un point $(X,Y)$ uniformément au hasard dans le disque unité de $\R^2$.
	\begin{enumerate}
		\item Quelle est la loi jointe de $(X,Y)$ ?
		\item Déterminer les lois marginales de $(X,Y)$.
		\item Soit $S$ la surface du disque centré en $(0,0)$ et dont $(X,Y)$ est sur le bord. Déterminer la loi de~$S$.
	\end{enumerate}
\end{exo}

\begin{comment}
\begin{enumerate}
	\item La loi de $(X,Y)$ est la mesure de Lebesgue restreinte au disque unité $\{ (x,y) \in \R^2 : x^2+y^2 \leq 1 \}$, que l'on normalise pour avoir masse totale 1.
	      Pour cela il faut la diviser par l'aire du disque unité, c'est-à-dire par $\pi$.
	      Donc on a $\diff P_{(X,Y)}(x,y) = \frac{1}{\pi} \1_{\{ x^2+y^2 \leq 1 \}} \diff x \diff y$.
	\item Comme $(X,Y)$ est continue, on sait par le cours que $X$ et $Y$ sont continues et leurs densités sont données par
	      \[
		      p_X(x) = \int_\R \frac{1}{\pi} \1_{\{ x^2+y^2 \leq 1 \}} \diff y
		      \qquad \text{et} \qquad
		      p_Y(y) = \int_\R \frac{1}{\pi} \1_{\{ x^2+y^2 \leq 1 \}} \diff x.
	      \]
	      Puis on remarque que $p_X(x) = 0$ si $\abs{x} > 1$ et, pour $\abs{x} \leq 1$,
	      \[
		      p_X(x) = \int_\R \frac{1}{\pi} \1_{\{ \abs{y} \leq \sqrt{1-x^2} \}} \diff y
		      = \frac{2 \sqrt{1-x^2}}{\pi}.
	      \]
	      Donc $X$ a pour loi $\frac{2}{\pi} \sqrt{1-x^2} \1_{[-1,1]}(x) \diff x$ et $Y$ a le même loi.
	\item Le rayon de ce disque est $\sqrt{X^2+Y^2}$ donc $S=\pi(X^2+Y^2)$.
	      Pour $f \colon \R \to \R_+$ mesurable, on a
	      \[
		      \Ec{f(S)} = \Ec{f(\pi(X^2+Y^2))}
		      = \int_{\R^2} f\left(\pi(x^2+y^2)\right) \frac{1}{\pi} \1_{\{ x^2+y^2 \leq 1 \}} \diff x \diff y
	      \]
	      Ici, on veut faire un changement de variable, mais il y a le choix : on veut introduire la variable $s = \pi(x^2+y^2)$, mais il y a une autre variable de libre.

	      On peut par exemple choisir de laisser $x$ inchangé, i.e. faire le changement $(x,s) = (x,\pi(x^2+y^2))$.
	      Cela revient à faire un changement de variable en dimension 1.
	      En effet par Fubini-Tonelli, on a
	      \[
		      \Ec{f(S)}
		      =  \frac{1}{\pi} \int_{-1}^1 \left( \int_{-1}^1 f\left(\pi(x^2+y^2)\right) \1_{\{ x^2+y^2 \leq 1 \}} \diff y \right) \diff x,
	      \]
	      donc on peut faire le changement $s = \pi(x^2+y^2)$ dans l'intégrale centrale, où $x$ est considéré comme fixé.
	      Cependant ce changement n'est pas bijectif, mais comme la fonction intégrée est paire en $y$ et en $x$, on a
	      \begin{align*}
		      \Ec{f(S)}
		       & = \frac{4}{\pi} \int_0^1 \left( \int_0^1 f\left(\pi(x^2+y^2)\right) \1_{\{ x^2+y^2 \leq 1 \}} \diff y \right) \diff x                          \\
		       & = \frac{4}{\pi} \int_0^1 \left( \int_{\pi x^2}^{\pi} f(s) \1_{\{ s/\pi \leq 1 \}} \frac{1}{2 \pi \sqrt{(s/\pi) - x^2}} \diff s \right) \diff x
	      \end{align*}
	      où on a utilisé $s = \pi(x^2+y^2) \Leftrightarrow y = \sqrt{(s/\pi) - x^2}$.
	      Puis on rééchange les intégrales, en utilisant que $s \geq \pi x^2 \Leftrightarrow x \leq \sqrt{s/\pi}$, et on intègre en $x$
	      \begin{align*}
		      \Ec{f(S)}
		       & = \frac{2}{\pi^2} \int_0^{\pi} f(s) \left( \int_{0}^{\sqrt{s/\pi}} \frac{1}{\sqrt{(s/\pi) - x^2}} \diff x \right) \diff s       \\
		       & = \frac{2}{\pi^2} \int_0^{\pi} f(s) \left[ \sin^{-1} \left( \frac{x}{\sqrt{s/\pi}} \right) \right]_{x=0}^{\sqrt{s/\pi}} \diff s \\
		       & = \frac{2}{\pi^2} \int_0^{\pi} f(s) \left( \frac{\pi}{2} - 0 \right) \diff s                                                    \\
		       & = \int_0^{\pi} f(s) \frac{1}{\pi} \diff s.
	      \end{align*}
	      Donc $S$ a pour loi $\frac{1}{\pi} \1_{[0,\pi]}(s) \diff s$, c'est donc une v.a. uniforme sur $[0,\pi]$.
\end{enumerate}
\end{comment}

%%%%%%
\separationexos
%%%%%%

\begin{exo}[Variables aléatoires entièrement corrélées]
	Soit $X,Y \in L^2(\Omega,\cA,\P)$.
	\begin{enumerate}
		\item Montrer que $\sigma_X = 0$ si et seulement si il existe $a \in \R$ tel que $X = a$ p.s.
		\item Posons $Z = \sigma_X^2 Y - \Cov(X,Y) X$.
		      Montrer que $\sigma_Z^2 = \sigma_X^2 ( \sigma_X^2 \sigma_Y^2 - \Cov(X,Y)^2)$.
		\item En déduire que $\abs{\Cov(X,Y)} = \sigma_X \sigma_Y$ si et seulement si
		      \[
			      \exists a,c \in \R, Y = cX+ a \text{ p.s.}
			      \qquad \text{ou} \qquad
			      \exists a \in \R, X = a \text{ p.s.}
		      \]
	\end{enumerate}
\end{exo}

\begin{comment}
\begin{enumerate}
	\item $(\Rightarrow)$ Si $\sigma_X = 0$, alors $\E[(X-\E[X])^2] = 0$, mais comme $(X-\E[X])^2 \geq 0$, cela implique que $(X-\E[X])^2 = 0$ $\P$-presque partout (c'est-à-dire p.s.).
	      Donc $X = \E[X]$ p.s. et $a = \E[X]$ est une constante.

	      $(\Leftarrow)$ S'il existe $a \in \R$ tel que $X = a$ p.s., alors $\E[X] = \E[a] = a$ et donc
	      \[
		      \sigma_X^2 = \Ec{(X-\E[X])^2}= \E[(a-a)^2] = 0.
	      \]
	      %%
	\item Notons $C = \Cov(X,Y)$ pour simplifier les formule. On a
	      \begin{align*}
		      \sigma_Z^2
		       & = \Ec{(Z-\E[Z])^2}                                       \\
		       & = \Ec{(\sigma_X^2 (Y-\E[Y]) - C (X-\E[X]))^2}            \\
		       & = \sigma_X^4 \Ec{(Y-\E[Y])^2} + C^2 \Ec{(X-\E[X])^2}
		      - 2 \sigma_X^2 C \Ec{(Y-\E[Y]) (X-\E[X])}                   \\
		       & = \sigma_X^4 \sigma_Y^2 + C^2 \sigma_X^2
		      - 2 \sigma_X^2 C^2                                          \\
		       & = \sigma_X^2 \left( \sigma_X^2 \sigma_Y^2 - C^2 \right).
	      \end{align*}
	      %%
	\item $(\Rightarrow)$ Si $\abs{\Cov(X,Y)} = \sigma_X \sigma_Y$, alors $\sigma_Z^2 = 0$, donc il existe $b \in \R$ tel que $Z=b$ p.s. par la question~1. Donc $\sigma_X^2 Y = \Cov(X,Y) X + b$ p.s.
	      Alors on a deux cas :
	      \begin{itemize}
		      \item Soit $\sigma_X^2 > 0$ et on a $Y = cX+a$ p.s. avec $a=b/\sigma_X^2$ et $c = \Cov(X,Y)/\sigma_X^2$.
		            %%
		      \item Soit $\sigma_X^2 = 0$ et alors $X$ est constante p.s. par la question 1.
	      \end{itemize}

	      $(\Leftarrow)$ On distingue les deux cas
	      \begin{itemize}
		      \item Si $Y = cX+a$ p.s. alors
		            \[
			            \Cov(X,Y)
			            = \Ec{(X-\E[X])(cX+a-\E[cX+a]) }
			            = c \Ec{(X-\E[X])^2}
			            = c\sigma_X^2
		            \]
		            et
		            \[
			            \sigma_Y^2
			            = \Ec{(cX+a-\E[cX+a])^2}
			            = c^2 \Ec{(X-\E[X])^2}
			            = c^2 \sigma_X^2
		            \]
		            donc $\abs{\Cov(X,Y)} = \sigma_X \sigma_Y$.
		            %%
		      \item Si $X = a$ p.s. alors $X - \E[X] = 0$ p.s. donc $\sigma_X = 0$ et $\Cov(X,Y) = 0$ donc $\abs{\Cov(X,Y)} = \sigma_X \sigma_Y$.
	      \end{itemize}
\end{enumerate}
\end{comment}

%%%%%%%%%%%%%%%%%%%%%%%%%%%%%%%%%%%%%%%%%%%%%%%%%%%
\partie{Indépendance}
%%%%%%%%%%%%%%%%%%%%%%%%%%%%%%%%%%%%%%%%%%%%%%%%%%%

\begin{exo}
	Soit $(\Omega, \cA,\P)$ un espace de probabilité et $A_1,\dots,A_n \in \cA$.
	Montrer que les propositions suivantes sont équivalentes :
	\begin{enumerate}[label=(\roman*)]
		\item les événements $A_1,\dots,A_n$ sont indépendants;
		\item pour tous $C_1 \in \{A_1, A_1^c\},\dots,C_n \in \{A_n, A_n^c\}$, les événements $C_1,\dots,C_n$ sont indépendants;
		\item les variables aléatoires $\1_{A_1},\dots\1_{A_n}$ sont indépendantes.
	\end{enumerate}
\end{exo}

\begin{comment}

\end{comment}

%%%%%%
\separationexos
%%%%%%

\begin{exo}
	Soit $(\Omega,\cA,\P)$ un espace de probabilité.
	\begin{enumerate}
		\item Soit $A \in \cA$.
		      Montrer que $\P(A) \in \{0,1\}$ si et seulement si, pour tout $B \in \cA$, $A$ et $B$ sont indépendants.
		      %%
		\item Soit $A,B,C \in \cA$. Supposons $A,B,C$ indépendants.
		      Montrer que $A$ est indépendant de $B \cup C$ et de $B \setminus C.$
	\end{enumerate}
\end{exo}

%%%%%%
\separationexos
%%%%%%

\begin{exo}
	Considérons l'expérience consistant à lancer deux dés à 6 faces équilibrés. Soit $X$ la somme des dés, $Y$ la différence absolue entre les dés,
	et $Z = \1_A$ où $A$ est l'événement ``le 1er dé donne un résultat pair''.
	\begin{enumerate}
		\item Modéliser l'expérience et spécifier $X$, $Y$ et $Z$ dans ce cadre.
		\item Pour chaque paire de v.a. parmi $X$, $Y$ et $Z$, justifier si elles sont indépendantes ou non.

		      \emph{Indication.} Pour montrer l'indépendance, on pourra utiliser que des v.a. discrètes $X_1$ et $X_2$ sont indépendantes ssi, pour tous $x_1,x_2 \in \R$, $\P(X_1=x_1,X_2=x_2) = \P(X_1=x_1)\P(X_2=x_2)$. Ce résultat, déjà vu en probabilités discrètes, sera rappelé au prochain cours.
	\end{enumerate}
\end{exo}

\begin{comment}
%i.e. $Y(i,j) = \abs{i-j}$ pour $(i,j) \in \Omega$, 
%On a $\P(Y=1) = \P(B) = 1/2$.
%	
%\begin{align*}
%\P(X=0,Y=1) = \P(\{(2,2),(4,4),(6,6)\}) = \frac{3}{36} = \frac{6}{36} \cdot \frac{1}{2} = \P(X=0) \cdot \P(Y=1).
%\end{align*}
%
%\begin{align*}
%\P(X=1,Y=1) = \P(\{(2,1),(2,3),(4,3),(4,5),(6,5)\}) = \frac{5}{36} = \frac{10}{36} \cdot \frac{1}{2} = \P(X=1) \cdot \P(Y=1).
%\end{align*}
%
%\begin{align*}
%\P(X=2,Y=1) = \P(\{(2,4),(4,2),(4,6),(6,4)\}) = \frac{4}{36} = \frac{8}{36} \cdot \frac{1}{2} = \P(X=2) \cdot \P(Y=1).
%\end{align*}
\end{comment}

%%%%%%%%%%%%%%%%%%%%%%%%%%%%%%%%%%%%%%%%%%%%%%%%%%%
\partie{Compléments}
%%%%%%%%%%%%%%%%%%%%%%%%%%%%%%%%%%%%%%%%%%%%%%%%%%%

\begin{exo}
	Soit $(X,Y)$ une v.a. à valeurs dans $\R^2$. On a vu en cours que si $(X,Y)$ est continue alors $X$ et $Y$ sont continues. Que dire de la réciproque ?
\end{exo}

%%%%%%
\separationexos
%%%%%%

\begin{exo}
	On tire trois élèves $E_1$, $E_2$ et $E_3$ dans la promotion et on s'interroge sur leur date de naissance.
	On considère les événements
	\begin{align*}
		A & = \text{``$E_1$ et $E_2$ ont le même anniversaire''}, \\
		B & = \text{``$E_1$ et $E_3$ ont le même anniversaire''}, \\
		C & = \text{``$E_3$ et $E_2$ ont le même anniversaire''}.
	\end{align*}
	Modéliser l'expérience (on négligera les années bissextiles) et montrer que les événements $A,B$ et $C$ sont deux à deux indépendants mais pas indépendants dans leur ensemble.
\end{exo}

%%%%%%
\separationexos
%%%%%%

\begin{exo} Soit $(X,Y)$ une v.a. de loi $\1_{[0,1]^2}(x,y) \diff x \diff y$.
	Déterminer la loi de $(X+Y,XY)$.
\end{exo}

%\begin{comment}
%Notons $S=X+Y$ et $P=XY$.
%Soit $f \colon \R^2 \to \R_+$ mesurables. On a, par le théorème de transfert,
%\[
%\Ec{f(S,P)} 
%= \Ec{f(X+Y,XY)} 
%= \int_{\R^2} f(x+y,xy) \1_{[0,1]^2}(x,y) \diff x \diff y.
%\]
%On veut faire le changement de variables $(s,p) = (x+y,xy)$. Mais $(x,y) \mapsto (x+y,xy)$ n'est pas injective sur ${]}0,1{[}^2$ car $(x,y)$ et $(y,x)$ ont la même image.
%\end{comment}

%%%%%%
\separationexos
%%%%%%

\begin{exo}
	On considère $\Omega = [0,1]$ muni de $\cA = \cB([0,1])$ et $\P$ la mesure de Lebesgue sur $[0,1]$. Pour tout $n \geq 1$, on définit
	\[
		A_n = \bigcup_{k=1}^{2^{n-1}} {\biggl]} \frac{2k-2}{2^n},\frac{2k-1}{2^n} \biggr].
	\]
	Montrer que $A_1,\dots,A_n$ sont indépendants.
\end{exo}

%%%%%%
\separationexos
%%%%%%

\begin{exo}
	Soient $n \in \N^*$ et $p_1,p_2, p_3 \in [0,1]$ tels que
	$p_1+p_2+p_3=1$.
	On considère $(X,Y)$ à valeurs dans $\llbracket 0,n\rrbracket^2$, de loi déterminée par
	\[
		\forall i,j \in \llbracket 0,n\rrbracket, \quad
		\mathbb{P}(X=i,Y=j)
		= \frac{n!}{i! j! (n-i-j)!} p_1^i p_2^j p_3^{n-i-j} \1_{i+j \leq n},
	\]
	\begin{enumerate}
		\item Que modélise $(X,Y)$ ?
		\item Déterminer les lois marginales de $(X,Y)$.
	\end{enumerate}
\end{exo}

%%%%%%
\separationexos
%%%%%%

\begin{exo}
	Soit $D$ le quart supérieur droit du disque unité de $\R^2$.
	Soit $(X,Y)$ de loi $cxy \1_D(x,y) \diff x \diff y$, où $c$ est une constante.
	\begin{enumerate}
		\item Déterminer $c$.
		\item Déterminer les marginales de $(X,Y)$.
		\item Soit $(R,\Theta)$ les coordonnées polaires de $(X,Y)$. Déterminer la loi de $(R,\Theta)$.
	\end{enumerate}
\end{exo}

%%%%%%
\separationexos
%%%%%%

\begin{exo}
	Soit $(\Omega, \cA,\P)$ un espace de probabilité. Soit $X$ une v.a. réelle.
	\begin{enumerate}
		\item Soit $A \in \cA$. Montrer que si $A$ et $A$ sont indépendants alors $\P(A) \in \{0,1\}$.
		      %%
		\item Soit $\mathcal{C}$ une sous-tribu de $\cA$ telle que pour tout $C \in \mathcal{C}$, $\P(C) \in \{0,1\}$.
		      Montrer que si $X$ est $\mathcal{C}$-mesurable, alors $X$ est constante p.s.

		      \emph{Indication.} On pourra utiliser la fonction de répartition de $X$ et considérer $x_0 \coloneqq \inf \{x \in \R : F_X(x)=1\}$.
		      %%
		\item Soit $f \colon \R \rightarrow \R$ mesurable et $Y= f(X)$.
		      Montrer que si $X$ et $Y$ sont indépendantes, alors $Y$ est constante p.s.

		      \emph{Indication.} On pourra montrer que la tribu $\cC = \sigma(Y)$ et la v.a. $Y$ satisfont les hypothèses de la question 2. Rappelons du TD2 que $\sigma(Y) = \{ Y^{-1}(B) : B\in\cB(\R) \}$ est la tribu engendrée par $Y$, qui est la plus petite tribu sur $\Omega$ rendant $Y$ mesurable.
	\end{enumerate}
\end{exo}

\begin{comment}
\begin{enumerate}
\item On a $\P(A \cap A)= \P(A)^2= \P(A)$. D'où $\P(A)=0$ ou $\P(A)=1$.
%%
\item Pour tout $x \in \R$, $\{X \leq x\} \in \mathcal{C}$ et donc $F(x) = \Pr{X \leq x}=0$ ou $1$.
Comme $\lim_{x \rightarrow \infty} F(x)=1$ et $\lim_{x \rightarrow -\infty} F(x)=0$, on a $x_0 \coloneqq \inf  \{x \in \R : F(x)=1\} \in (- \infty, \infty)$.
Comme $F$ est croissante, si $a<x_0<b$, on a $F(a)=0$ et $F(b)=1$.
En particulier, $ \P( x_0-1/n < X \leq x_0+1/n)=1$ pour $n \geq 1$.
Or:
\[
	\{x_0\}= \bigcap_{n \geq 1} \left] x_0- \frac{1}{n}, x_{0}+ \frac{1}{n} \right].
\]
Cette intersection \'etant d\'ecroissante et $\P$ \'etant finie, on a donc
\[
	\P(X=x_0)
	= \lim_{n \to \infty} \P \left( \left] x_0- \frac{1}{n}, x_0+ \frac{1}{n} \right] \right)
	=1.
\]
Ainsi, $X=x_0$ p.s.
%%
\item Notons $\mathcal{C}= \sigma(Y)$.
Soit $C \in \cC$. On a $C = Y^{-1}(B)$ pour $B\in\cB(\R)$.
Donc
\[
	C = \{Y \in B\} = \{f(X) \in B\} = \{ X \in f^{-1}(B) \}.
\]
Mais par indépendance de $X$ et $Y$, les événement $\{Y \in B\}$ et $\{ X \in f^{-1}(B) \}$ sont indépendants.
Donc $C$ est ind\'ependant de lui-m\^eme, et donc $\P(C) \in \{0,1\}$ par la question 1.
Donc la tribu $\cC$ satisfait : pour tout $C \in \mathcal{C}$, $\P(C) \in \{0,1\}$.
De plus $Y$ est $\cC$-mesurable (car $\sigma(Y)$ est la plus petite tribu sur $\Omega$ rendant $Y$ mesurable).
Donc $Y$ est constante p.s. par la question 2.
\end {enumerate}
\end{comment}

\end{document}
