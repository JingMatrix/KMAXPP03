\documentclass[a4paper,11pt]{article}
\usepackage[utf8]{inputenc}
\usepackage[T1]{fontenc}
\usepackage{lmodern}

\usepackage{amsthm,amsmath,amsfonts,amssymb,bbm,mathrsfs,stmaryrd}
\usepackage{mathtools}
\usepackage{enumitem}
\usepackage{url}
\usepackage{dsfont}
\usepackage{appendix}
\usepackage{amsthm}
\usepackage[dvipsnames,svgnames]{xcolor}
\usepackage{graphicx}

\usepackage{fancyhdr,lastpage,titlesec,verbatim,ifthen}

\usepackage[colorlinks=true, linkcolor=black, urlcolor=black, citecolor=black]{hyperref}

\usepackage[french]{babel}

\usepackage{caption,tikz,subfigure}

\usepackage[top=2cm, bottom=2cm, left=2cm, right=2cm]{geometry}

%%%%%%%%%%%%%%%%%%%%%%%%%%%%%%%%%%%%%%%%%%%%%%%%%%%%%%%%%%%%%%%%%%%%%%%

%%%%%%%% Taille de la legende des images %%%%%%%%%%%%%%%%%%%%%%%%%%%%%%%
\renewcommand{\captionfont}{\footnotesize}
\renewcommand{\captionlabelfont}{\footnotesize}

%%%%%%%% Numeration des enumerates en romain et chgt de l'espace %%%%%%%
\setitemize[1]{label=$\rhd$, font=\color{NavyBlue},leftmargin=0.8cm}
\setenumerate[1]{font=\color{NavyBlue},leftmargin=0.8cm}
\setenumerate[2]{font=\color{NavyBlue},leftmargin=0.49cm}
%\setlist[enumerate,1]{label=(\roman*), font = \normalfont,itemsep=4pt,topsep=4pt} 
%\setlist[itemize,1]{label=\textbullet, font = \normalfont,itemsep=4pt,topsep=4pt} 

%%%%%%%% Pas d'espacement supplementaire avant \left et apres \right %%%
%%%%%%%% Note : pour les \Big(, utiliser \Bigl( \Bigr) %%%%%%%%%%%%%%%%%
\let\originalleft\left
\let\originalright\right
\renewcommand{\left}{\mathopen{}\mathclose\bgroup\originalleft}
\renewcommand{\right}{\aftergroup\egroup\originalright}

%%%%%%%%%%%%%%%%%%%%%%%%%%%%%%%%%%%%%%%%%%%%%%%%%%%%%%%%%%%%%%%%%%%%%%%

\newcommand{\N}{\mathbb{N}}
\newcommand{\Z}{\mathbb{Z}}
\newcommand{\Q}{\mathbb{Q}}
\newcommand{\R}{\mathbb{R}}
\newcommand{\C}{\mathbb{C}}
\newcommand{\T}{\mathbb{T}}
\renewcommand{\P}{\mathbb{P}}
\newcommand{\E}{\mathbb{E}}

\newcommand{\1}{\mathbbm{1}}

\newcommand{\cA}{\mathcal{A}}
\newcommand{\cB}{\mathcal{B}}
\newcommand{\cC}{\mathcal{C}}
\newcommand{\cE}{\mathcal{E}}
\newcommand{\cF}{\mathcal{F}}
\newcommand{\cH}{\mathcal{H}}
\newcommand{\cI}{\mathcal{I}}
\newcommand{\cJ}{\mathcal{J}}
\newcommand{\cL}{\mathcal{L}}
\newcommand{\cM}{\mathcal{M}}
\newcommand{\cN}{\mathcal{N}}
\newcommand{\cP}{\mathcal{P}}
\newcommand{\cS}{\mathcal{S}}
\newcommand{\cT}{\mathcal{T}}

\newcommand{\Ec}[1]{\mathbb{E} \left[#1\right]}
\newcommand{\Pp}[1]{\mathbb{P} \left(#1\right)}
\newcommand{\Ppsq}[2]{\mathbb{P} \left(#1\middle|#2\right)}

\newcommand{\e}{\varepsilon}

\newcommand{\ii}{\mathrm{i}}
\DeclareMathOperator{\re}{Re}
\DeclareMathOperator{\im}{Im}
\DeclareMathOperator{\Arg}{Arg}

\newcommand{\diff}{\mathop{}\mathopen{}\mathrm{d}}
\DeclareMathOperator{\Var}{Var}
\DeclareMathOperator{\Cov}{Cov}
\newcommand{\supp}{\mathrm{supp}}

\newcommand{\abs}[1]{\left\lvert#1\right\rvert}
\newcommand{\abso}[1]{\lvert#1\rvert}
\newcommand{\norme}[1]{\left\lVert#1\right\rVert}
\newcommand{\ps}[2]{\langle #1,#2 \rangle}

\newcommand{\petito}[1]{o\mathopen{}\left(#1\right)}
\newcommand{\grandO}[1]{O\mathopen{}\left(#1\right)}

\newcommand\relphantom[1]{\mathrel{\phantom{#1}}}

\newcommand{\NB}[1]{{\color{NavyBlue}#1}}
\newcommand{\DSB}[1]{{\color{DarkSlateBlue}#1}}

%%%%%%%% Theorems styles %%%%%%%%%%%%%%%%%%%%%%%%%%%%%%%%%%%%%%%%%%%%%%
\theoremstyle{plain}
\newtheorem{theorem}{Theorem}[section]
\newtheorem{proposition}[theorem]{Proposition}
\newtheorem{lemma}[theorem]{Lemme}
\newtheorem{corollary}[theorem]{Corollaire}
\newtheorem{conjecture}[theorem]{Conjecture}
\newtheorem{definition}[theorem]{Définition}

\theoremstyle{definition}
\newtheorem{remark}[theorem]{Remarque}
\newtheorem{example}[theorem]{Exemple}
\newtheorem{question}[theorem]{Question}

%%%%%%%% Macros spéciales TD %%%%%%%%%%%%%%%%%%%%%%%%%%%%%%%%%%%%%%%%%%

%%%%%%%%%%%% Changer numérotation des pages %%%%%%%%%%%%%%%%%%%%%%%%%%%%
\pagestyle{fancy}
\cfoot{\thepage/\pageref{LastPage}} %%% numéroter page / total de pages
\renewcommand{\headrulewidth}{0pt} %%% empêcher qu'il y ait une ligne horizontale en haut
%%%%%%%%%%%% Ne pas numéroter les pages %%%%%%%%%%%%%%%%%
%\pagestyle{empty}

%%%%%%%%%%%% Supprimer les alineas %%%%%%%%%%%%%%%%%%%%%%%%%%%%%%%%%%%%%
\setlength{\parindent}{0cm}

%%%%%%%%%%%% Exercice %%%%%%%%%%%%%%%%%%%%%%%%%%%%%%%%%% 
\newcounter{exo}
\newenvironment{exo}[1][vide]
{\refstepcounter{exo}
	{\noindent \textcolor{DarkSlateBlue}{\textbf{Exercice \theexo.}}}
	\ifthenelse{\equal{#1}{vide}}{}{\textcolor{DarkSlateBlue}{(#1)}}
}{}

%%%%%%%%%%%% Partie %%%%%%%%%%%%%%%%%%%%%%%%%%%%%%%%%%%%
\newcounter{partie}
\newcommand\partie[1]{
	\stepcounter{partie}%
	{\bigskip\large\textbf{\DSB{\thepartie.~#1}}\bigskip}
}

%%%%%%%%%%%% Separateur entre les exos %%%%%%%%%%%%%%%%%
\newcommand{\separationexos}{
	\bigskip
	\bigskip
	%	{\centering\hfill\DSB{\rule{0.4\linewidth}{1.2pt}}\hfill}\medskip
}
\newcommand{\petiteseparationexos}{
	\bigskip
	%	{\centering\hfill\DSB{\rule{0.4\linewidth}{1.2pt}}\hfill}\medskip
}

%%%%%%%%%%%% Corrige %%%%%%%%%%%%%%%%%%%%%%%%%%%%%%%%%%% 
\renewenvironment{comment}{\medskip\noindent \textcolor{BrickRed}{\textbf{Corrigé.}}}{}

%%%%%%%%%%%% Titre %%%%%%%%%%%%%%%%%%%%%%%%%%%%%%%%%%%%%%
\newcommand\titre[1]{\ \vspace{-1cm}

	\DSB{\rule{\linewidth}{1.2pt}}
	{\small Probabilités et statistiques continues avancées}
	\hfill {\small Université Paul Sabatier}

	{\small KMAXPP03}
	\hfill {\small Licence 3, Printemps 2023}\medskip
	\begin{center}
		{\Large\textbf{\DSB{#1}}}\vspace{-.2cm}
	\end{center}
	\DSB{\rule{\linewidth}{1.2pt}}\medskip
}

%%%%%%%%%%%%%%%%%%%%%%%%%%%%%%%%%%%%%%%%%%%%%%%%%%%%%%%%%%%%%%%%%%%%%%%
\begin{document}
%%%%%%%%%%%%%%%%%%%%%%%%%%%%%%%%%%%%%%%%%%%%%%%%%%%%%%%%%%%%%%%%%%%%%%%

\titre{Devoir sur table 2 \\ \medskip {\large Contrôle continu du 17 avril 2023, 10h00--11h30}}

\medskip 

\noindent \textcolor{DarkSlateBlue}{\textbf{Instructions:}} 
Tout appareil électronique est interdit. Le seul matériel autorisé est une feuille A4 recto-verso manuscrite.

\medskip 

\noindent \textcolor{DarkSlateBlue}{\textbf{Total:}} 20 points

%%%%%%
\separationexos
%%%%%%

\begin{exo}[3 points] 
	Soit $(X_n)_{n\geq1}$ une suite de v.a. positives convergeant p.s. vers $X$. Montrer que
	\[
		\frac{2 X_n}{X_n+1} \xrightarrow[n\to\infty]{L^1} \frac{2 X}{X+1}.
	\]	
\end{exo}


%%%%%%
\petiteseparationexos
%%%%%%


\begin{exo}[7 points] 
	Soit $(U_n)_{n\geq 1}$ une suite de v.a. i.i.d. de loi uniforme sur $[0,1]$.
	\begin{enumerate}
		\item \DSB{(2 pts)} Montrer que $(U_n^n)_{n\geq 1}$ converge dans $L^1$ vers 0.
		%%
		\item \DSB{(1 pt)} Montrer que $(U_n^n)_{n\geq 1}$ converge en probabilité vers 0.
		%%
		\item 
		\begin{enumerate}
			\item \DSB{(2 pts)} Soit $a \in {]}0,1{[}$. Montrer que
			\[
				\sum_{n\geq 1} \Pp{U_n^n \geq a} = \infty.
			\]
			%%
			\item \DSB{(2 pts)} En déduire que, p.s., $(U_n^n)_{n\geq 1}$ ne converge pas  vers 0.
		\end{enumerate}
	\end{enumerate}
\end{exo}

%%%%%%
%\separationexos
%%%%%%%
%
%\begin{exo}[2 points] 
%	Soit $X$ une v.a. réelle. Soit $(u_n)_{n\geq 1}$ une suite de réels tendant vers l'infini.
%	Montrer que $(X/u_n)_{n\geq 1}$ converge en probabilité vers 0.
%\end{exo}

%%%%%%
\separationexos
%%%%%%

\begin{exo}[7 points] 
	Soit $(X_n)_{n\geq 1}$ une suite de v.a. positives indépendantes dans $L^2$. On suppose qu'il existe $m,v >0$ tels que
	\[
		\forall n \geq 1, \quad 
		\E[X_n] = m
		\quad \text{et} \quad
		\Var(X_n) \leq v.
	\]
	On pose $S_n = X_1 + \dots +X_n$. On veut montrer que $(S_n/n)_{n\geq 1}$ converge p.s. vers $m$.
	\begin{enumerate}
		\item \DSB{(1 pt)} Quelle hypothèse manque-t-il pour appliquer la loi forte des grands nombres vue en cours ?
		%%
		\item \DSB{(2 pts)} Pour tout $\varepsilon > 0$, montrer que 
		\[
			\Pp{\abs{\frac{S_n}{n} - m} > \varepsilon} \leq \frac{v}{n\varepsilon^2}.
		\]
		%%
		\item \DSB{(2 pts)} En déduire que
		\[
			\frac{S_{n^2}}{n^2} \xrightarrow[n\to\infty]{\text{p.s.}} m.
		\]
		%%
		\item \DSB{(2 pts)} En déduire que
		\[
		\frac{S_{n}}{n} \xrightarrow[n\to\infty]{\text{p.s.}} m.
		\]
	\end{enumerate}
\end{exo}

%%%%%%
\petiteseparationexos
%%%%%%

\begin{exo}[3 points] 
	Soit $(X_n)_{n\geq1}$ et $(Y_n)_{n\geq 1}$ des suites de v.a. réelles. 
	On dit que $(X_n)_{n\geq1}$ est tendue si
	\[
		\forall \delta > 0, \quad 
		\exists M> 0, \quad 
		\forall n \geq 1, \quad
		\P(\abs{X_n} \geq M) \leq \delta.
	\]
	Supposons $(X_n)_{n\geq 1}$ tendue et que $(Y_n)_{n\geq 1}$ converge en probabilité vers 0.
	Montrer que $(X_n Y_n)_{n\geq 1}$ converge en probabilité vers 0.
\end{exo}


\end{document}
