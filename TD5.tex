\documentclass[a4paper,11pt]{article}
\usepackage[utf8]{inputenc}
\usepackage[T1]{fontenc}
\usepackage{lmodern}

\usepackage{amsthm,amsmath,amsfonts,amssymb,bbm,mathrsfs,stmaryrd}
\usepackage{mathtools}
\usepackage{enumitem}
\usepackage{url}
\usepackage{dsfont}
\usepackage{appendix}
\usepackage{amsthm}
\usepackage[dvipsnames,svgnames]{xcolor}
\usepackage{graphicx}

\usepackage{fancyhdr,lastpage,titlesec,verbatim,ifthen}

\usepackage[colorlinks=true, linkcolor=black, urlcolor=black, citecolor=black]{hyperref}

\usepackage[french]{babel}

\usepackage{caption,tikz,subfigure}

\usepackage[top=2cm, bottom=2cm, left=2cm, right=2cm]{geometry}

%%%%%%%%%%%%%%%%%%%%%%%%%%%%%%%%%%%%%%%%%%%%%%%%%%%%%%%%%%%%%%%%%%%%%%%

%%%%%%%% Taille de la legende des images %%%%%%%%%%%%%%%%%%%%%%%%%%%%%%%
\renewcommand{\captionfont}{\footnotesize}
\renewcommand{\captionlabelfont}{\footnotesize}

%%%%%%%% Numeration des enumerates en romain et chgt de l'espace %%%%%%%
\setitemize[1]{label=$\rhd$, font=\color{NavyBlue},leftmargin=0.8cm}
\setenumerate[1]{font=\color{NavyBlue},leftmargin=0.8cm}
\setenumerate[2]{font=\color{NavyBlue},leftmargin=0.47cm}
%\setlist[enumerate,1]{label=(\roman*), font = \normalfont,itemsep=4pt,topsep=4pt} 
%\setlist[itemize,1]{label=\textbullet, font = \normalfont,itemsep=4pt,topsep=4pt} 

%%%%%%%% Pas d'espacement supplementaire avant \left et apres \right %%%
%%%%%%%% Note : pour les \Big(, utiliser \Bigl( \Bigr) %%%%%%%%%%%%%%%%%
\let\originalleft\left
\let\originalright\right
\renewcommand{\left}{\mathopen{}\mathclose\bgroup\originalleft}
\renewcommand{\right}{\aftergroup\egroup\originalright}

%%%%%%%%%%%%%%%%%%%%%%%%%%%%%%%%%%%%%%%%%%%%%%%%%%%%%%%%%%%%%%%%%%%%%%%

\newcommand{\N}{\mathbb{N}}
\newcommand{\Z}{\mathbb{Z}}
\newcommand{\Q}{\mathbb{Q}}
\newcommand{\R}{\mathbb{R}}
\newcommand{\C}{\mathbb{C}}
\newcommand{\T}{\mathbb{T}}
\renewcommand{\P}{\mathbb{P}}
\newcommand{\E}{\mathbb{E}}

\newcommand{\1}{\mathbbm{1}}

\newcommand{\cA}{\mathcal{A}}
\newcommand{\cB}{\mathcal{B}}
\newcommand{\cC}{\mathcal{C}}
\newcommand{\cE}{\mathcal{E}}
\newcommand{\cF}{\mathcal{F}}
\newcommand{\cG}{\mathcal{G}}
\newcommand{\cH}{\mathcal{H}}
\newcommand{\cI}{\mathcal{I}}
\newcommand{\cJ}{\mathcal{J}}
\newcommand{\cL}{\mathcal{L}}
\newcommand{\cM}{\mathcal{M}}
\newcommand{\cN}{\mathcal{N}}
\newcommand{\cP}{\mathcal{P}}
\newcommand{\cS}{\mathcal{S}}
\newcommand{\cT}{\mathcal{T}}
\newcommand{\cU}{\mathcal{U}}

\newcommand{\Ec}[1]{\mathbb{E} \left[#1\right]}
\newcommand{\Pp}[1]{\mathbb{P} \left(#1\right)}
\newcommand{\Ppsq}[2]{\mathbb{P} \left(#1\middle|#2\right)}

\newcommand{\e}{\varepsilon}

\newcommand{\ii}{\mathrm{i}}
\DeclareMathOperator{\re}{Re}
\DeclareMathOperator{\im}{Im}
\DeclareMathOperator{\Arg}{Arg}

\newcommand{\diff}{\mathop{}\mathopen{}\mathrm{d}}
\DeclareMathOperator{\Var}{Var}
\DeclareMathOperator{\Cov}{Cov}
\newcommand{\supp}{\mathrm{supp}}

\newcommand{\abs}[1]{\left\lvert#1\right\rvert}
\newcommand{\abso}[1]{\lvert#1\rvert}
\newcommand{\norme}[1]{\left\lVert#1\right\rVert}
\newcommand{\ps}[2]{\langle #1,#2 \rangle}

\newcommand{\petito}[1]{o\mathopen{}\left(#1\right)}
\newcommand{\grandO}[1]{O\mathopen{}\left(#1\right)}

\newcommand\relphantom[1]{\mathrel{\phantom{#1}}}

\newcommand{\NB}[1]{{\color{NavyBlue}#1}}
\newcommand{\DSB}[1]{{\color{DarkSlateBlue}#1}}

%%%%%%%% Theorems styles %%%%%%%%%%%%%%%%%%%%%%%%%%%%%%%%%%%%%%%%%%%%%%
\theoremstyle{plain}
\newtheorem{theorem}{Theorem}[section]
\newtheorem{proposition}[theorem]{Proposition}
\newtheorem{lemma}[theorem]{Lemme}
\newtheorem{corollary}[theorem]{Corollaire}
\newtheorem{conjecture}[theorem]{Conjecture}
\newtheorem{definition}[theorem]{Définition}

\theoremstyle{definition}
\newtheorem{remark}[theorem]{Remarque}
\newtheorem{example}[theorem]{Exemple}
\newtheorem{question}[theorem]{Question}

%%%%%%%% Macros spéciales TD %%%%%%%%%%%%%%%%%%%%%%%%%%%%%%%%%%%%%%%%%%

%%%%%%%%%%%% Changer numérotation des pages %%%%%%%%%%%%%%%%%%%%%%%%%%%%
\pagestyle{fancy}
\cfoot{\thepage/\pageref{LastPage}} %%% numéroter page / total de pages
\renewcommand{\headrulewidth}{0pt} %%% empêcher qu'il y ait une ligne horizontale en haut
%%%%%%%%%%%% Ne pas numéroter les pages %%%%%%%%%%%%%%%%%
%\pagestyle{empty}

%%%%%%%%%%%% Supprimer les alineas %%%%%%%%%%%%%%%%%%%%%%%%%%%%%%%%%%%%%
\setlength{\parindent}{0cm} 

%%%%%%%%%%%% Exercice %%%%%%%%%%%%%%%%%%%%%%%%%%%%%%%%%% 
\newcounter{exo}
\newenvironment{exo}[1][vide]
{\refstepcounter{exo}
	{\noindent \textcolor{DarkSlateBlue}{\textbf{Exercice \theexo.}}}
	\ifthenelse{\equal{#1}{vide}}{}{\textcolor{DarkSlateBlue}{(#1)}}
}{}

%%%%%%%%%%%% Partie %%%%%%%%%%%%%%%%%%%%%%%%%%%%%%%%%%%%
\newcounter{partie}
\newcommand\partie[1]{
	\stepcounter{partie}%
	{\bigskip\large\textbf{\DSB{\thepartie.~#1}}\bigskip}
	}

%%%%%%%%%%%% Separateur entre les exos %%%%%%%%%%%%%%%%%
\newcommand{\separationexos}{
	\bigskip
%	{\centering\hfill\DSB{\rule{0.4\linewidth}{1.2pt}}\hfill}\medskip
	}

%%%%%%%%%%%% Corrige %%%%%%%%%%%%%%%%%%%%%%%%%%%%%%%%%%% 
\renewenvironment{comment}{\medskip\noindent \textcolor{BrickRed}{\textbf{Corrigé.}}}{}

%%%%%%%%%%%% Titre %%%%%%%%%%%%%%%%%%%%%%%%%%%%%%%%%%%%%%
\newcommand\titre[1]{\ \vspace{-1cm}
	
	\DSB{\rule{\linewidth}{1.2pt}}
	{\small Probabilités et statistiques continues avancées}
	\hfill {\small Université Paul Sabatier}
	
	{\small KMAXPP03}
	\hfill {\small Licence 3, Printemps 2023}\medskip
	\begin{center}
		{\Large\textbf{\DSB{#1}}}\vspace{-.2cm}
	\end{center}
	\DSB{\rule{\linewidth}{1.2pt}}\medskip
}

%%%%%%%%%%%%%%%%%%%%%%%%%%%%%%%%%%%%%%%%%%%%%%%%%%%%%%%%%%%%%%%%%%%%%%%
\begin{document}
%%%%%%%%%%%%%%%%%%%%%%%%%%%%%%%%%%%%%%%%%%%%%%%%%%%%%%%%%%%%%%%%%%%%%%%

\titre{TD 5 -- Indépendance de variables aléatoires}


%%%%%%%%%%%%%%%%%%%%%%%%%%%%%%%%%%%%%%%%%%%%%%%%%%%
\partie{Indépendance de variables aléatoires}
%%%%%%%%%%%%%%%%%%%%%%%%%%%%%%%%%%%%%%%%%%%%%%%%%%%


\begin{exo}[Espérances successives]
	Supposons que $X$ et $Y$ sont indépendantes. Soit $F \colon \R \times \R \rightarrow \R_+$ une fonction mesurable. 
	Montrer que $\Ec{F(X,Y)}= \Ec{g(Y)}$, o\`u $g \colon \R \rightarrow \R$ est la fonction d\'efinie par $g(y)=\Ec{F(X,y)}$ pour $y \in \R$.
\end{exo}


\begin{comment}
Comme $X$ et $Y$ sont indépendantes, $P_{(X,Y)}= P_{X}\otimes P_{Y}$. Par le théorème de transfert, on a 
\begin{align*}
\Ec{F(X,Y)} 
& = \int_{\R \times \R} F(x,y) \diff P_{(X,Y)}(x,y) \\
& = \int_{\R \times \R} F(x,y) \diff P_{X}(x) \diff P_{Y}(y) \\
& = \int_{\R} \left( \int_{\R} F(x,y)P_{X}(\diff x)\right) \diff P_{Y}(y) \\
& = \int_{\R} g(y) \diff P_{Y}(y) \\
& = \Ec{g(Y)},
\end{align*}
où l'on a utilisé Fubini--Tonelli, puis la définition de $g$ et finalement le théorème de transfert.
\end{comment}

%%%%%%
\separationexos
%%%%%%


\begin{exo}[Covariance et indépendance] Soit $X$ et $Y$ deux v.a. réelles.
	\begin{enumerate}
        \item Pour $a_1,a_2,b_1,b_2 \in \R$, montrer que 
        $\Cov(a_1X+a_2,b_1Y+b_2) = a_1 b_1 \Cov(X,Y)$.
        %%
		\item Supposons que $X$ et $Y$ ne peuvent prendre chacune que deux valeurs distinctes.
		Montrer que $X$ et $Y$ sont indépendantes si et seulement si $\Cov(X,Y) = 0$.
		
		\emph{Indication.} On pourra montrer qu'il existe $a_1,b_1 \in \R^*$ et $a_2,b_2 \in \R$ tels que $a_1X+a_2$ et $b_1Y+b_2$ sont à valeurs dans $\{0,1\}$, puis vérifier que $X$ et $Y$ sont indépendantes ssi $a_1X+a_2$ et $b_1Y+b_2$ le sont.
		%%
		\item Soit $X$ une v.a. de loi $\cN(0,1)$. Montrer que $\Cov(X,X^2) = 0$. Qu'en conclure?
	\end{enumerate}
\end{exo}

\begin{comment}
\begin{enumerate}
    \item On a 
    \begin{align*}
    \Cov(a_1X+a_2,b_1Y+b_2) 
    & = \E[(a_1X+a_2-\E[a_1X+a_2])(b_1Y+b_2-\E[b_1Y+b_2])] \\
    & = \E[(a_1X-\E[a_1X])(b_1Y-\E[b_1Y])] \\
    & = a_1 a_2 \E[(X-\E[X])(Y-\E[Y])] \\
    & = a_1 a_2 \Cov(X,Y).
    \end{align*}
    
	\item Le fait que si $X$ et $Y$ sont indépendantes, alors $\Cov(X,Y) = 0$ est vrai en toute généralité et a été vu en cours.

    
	Montrons donc la réciproque dans le cas où $X \in \{x_1,x_2\}$ et $Y \in \{y_1,y_2\}$ avec $x_1 \neq x_2$ et $y_1 \neq y_2$.
    Supposons que $\Cov(X,Y) = 0$, on veut montrer que $X$ et $Y$ sont indépendantes.
    On cherche $a_1,a_2 \in \R$ tels que $a_1X+a_2 \in \{0,1\}$. Pour cela il suffit que
    \[
    \begin{cases}
    a_1x_1+a_2 = 0 \\
    a_1x_2+a_2 = 1
    \end{cases}
    \quad \Leftrightarrow \quad 
    \begin{cases}
    a_2 = -a_1x_1 \\
    a_1(x_2-x_1) = 1
    \end{cases}
    \quad \Leftrightarrow \quad 
    \begin{cases}
    a_2 = -x_1(x_2-x_1)^{-1} \\
    a_1 = (x_2-x_1)^{-1}
    \end{cases}
    \]
    où l'on utilise que $x_2-x_1 \neq 0$. En outre, on a $a_1 \neq 0$.
    De même, il existe $b_1 \in \R^*$, $b_2\in \R$ tels que $b_1 Y + b_2 \in \{0,1\}$.
    On pose $X' = a_1X+a_2$ et $Y' = b_1 Y + b_2$.

    Montrons que $X'$ et $Y'$ sont indépendantes. Notons d'abord que, par la question 1, $\Cov(X',Y') = a_1 b_1 \Cov(X,Y) = 0$.
    Donc 
    \[
    \E[X'Y'] = \E[X'] \E[Y'].
    \]
    Posons $A = \{X'=1\}$ et $B = \{Y' = 1\}$. Alors $X' = \1_A$ (car $X' \in \{0,1\}$) et $Y' = \1_B$.
    Donc $\E[X'] = \P(A)$, $\E[Y'] = \P(B)$ et $\E[X'Y'] = \E[\1_{A\cap B}] = \P(A \cap B)$. On a donc montré que 
    \[
    \P(A \cap B) = \P(A)\P(B)
    \]
    et ainsi $A$ et $B$ sont indépendants. Par l'exercice 4 du TD4 cela montre que $X'=\1_A$ et $Y'=\1_B$ sont indépendants.
    Il en découle que $X = (X'-a_2)/a_1$ et $Y = (Y'-b_2)/b_1$ sont indépendants (en tant que fonctions mesurables de v.a. indépendantes).

    
%    \emph{Indication.} Si $x_1,x_2$ sont les valeurs possibles de $X$ et $y_1,y_2$ les valeurs possibles de $Y$, on montrera que $\E[(X-x_i)(Y-y_j)] = \E[X-x_i] \E[Y-y_j]$ pour tous $i,j \in \{1,2\}$.
%   On a donc $\E[XY] = \E[X]\E[Y]$.
%	Ainsi, pour tous $i,j \in \{1,2\}$,
%	\begin{align*}
%	\E[(X-x_i)(Y-y_j)] 
%	& = \E[XY] - x_i \E[Y] -y_j \E[X] + x_i x_j \\
%	& = \E[X]\E[Y] - x_i \E[Y] -y_j \E[X] + x_i x_j \\
%	& = (\E[X]-x_i)(\E[Y] -y_j) \\
%	& = \E[X-x_i] \E[Y-y_j].
%	\end{align*}
%	Comme $X-x_1 = (x_2-x_1) \1_{\{X=x_2\}}$, on a $\E[X-x_1] = (x_2-x_1) \P(X=x_2)$. De même $\E[Y-y_1] = (y_2-y_1) \P(Y=y_2)$.
%	De plus, $(X-x_1)(Y-y_1) = (x_2-x_1)(y_2-y_1) \1_{\{X=x_2,Y=y_2\}}$,
%	donc
%	\[
%	\E[(X-x_1)(Y-y_1)] = (x_2-x_1)(y_2-y_1) \P(X=x_2,Y=y_2).
%	\]
%	Ainsi, la formule montrée précédemment implique que
%	\[
%	\P(X=x_2,Y=y_2) = \P(X=x_2) \P(Y=y_2).
%	\]
%	Comme $x_1$ et $x_2$ jouent le même rôle on peut remplacer $x_2$ par $x_1$ dans la formule ci-dessus. De même on peut remplacer $y_2$ par $y_1$.
%	Ainsi on a
%	\[
%	\forall i,j \in \{1,2\}, \P(X=x_i,Y=y_j) = \P(X=x_i) \P(Y=y_j),
%	\]
%	ce qui montre que $X$ et $Y$ sont indépendantes.
	%%
    \item On a
	\[
	\Cov(X,X^2) = \Ec{X \cdot X^2} - \Ec{X} \cdot \Ec{X^2} = 0 - 0 \cdot 1 = 0,
	\]
	où l'on a utilisé
	\[
	\Ec{X^3} = \int_\R x^3 \frac{1}{\sqrt{2\pi}} e^{-x^2/2} \diff x = 0
	\]
	par imparité de la fonction intégrée.
	
	On en conclut qu'en général avoir une covariance nulle n'implique pas l'indépendance. En effet, $X$ et $X^2$ ne sont pas indépendantes : si c'était le cas, on aurait $X^2$ et $X^2$ indépendantes (en appliquant la fonction carré à $X$) et donc
	\[
		\P(X^2 < 1,X^2< 1) = \P(X^2 < 1) \P(X^2< 1) = \P(X^2 < 1)^2,
	\]
	ce qui implique $\P(X^2 < 1) \in \{0,1\}$ ce qui n'est pas vrai:
	\[
	0 < \P(X^2< 1) = \int_{-1}^1 \frac{1}{\sqrt{2\pi}} e^{-x^2/2} \diff x < \int_\R \frac{1}{\sqrt{2\pi}} e^{-x^2/2} \diff x = 1.
	\]
\end{enumerate}
\end{comment}


%%%%%%%%%%%%%%%%%%%%%%%%%%%%%%%%%%%%%%%%%%%%%%%%%%%
\partie{Indépendance et calcul de lois}
%%%%%%%%%%%%%%%%%%%%%%%%%%%%%%%%%%%%%%%%%%%%%%%%%%%

\begin{exo}[Méthode de Box-Muller de simulation des gaussiennes]
	\begin{enumerate}
		\item Soit $U$ une v.a. uniforme sur $[0,1]$. Pour $\lambda > 0$, déterminer la loi de $-\frac{1}{\lambda} \ln U$.
		\item Soit $\Theta$ une v.a. uniforme sur $[0,2\pi]$ et $Z$ une v.a. de loi exponentielle de paramètre 1/2. 
        On suppose que $Z$ et $\theta$ sont indépendantes.
        Déterminer la loi de $(\sqrt{Z} \cos \Theta, \sqrt{Z} \sin \Theta)$.
		\item En déduire une méthode pour simuler deux v.a. indépendantes de loi gaussienne standard à partir de deux v.a. $U_1$ et $U_2$ indépendantes de loi uniforme sur $[0,1]$.
	\end{enumerate}
\end{exo}

\begin{comment}
\begin{enumerate}
\item Soit par changement de variable soit par calcul de fonction de répartition, on trouve que $-\frac{1}{\lambda} \ln U$ suit une loi exponentielle de paramètre $\lambda$.
%%
\item Soit $f:\R^2\to\R_+$ une fonction bor\'elienne. On a 
\begin{align*}
\Ec{f\left(\sqrt{Z}\cos\Theta,\sqrt{Z}\sin\Theta\right)}
&=\frac{1}{4\pi}\int_0^{2\pi} \int_0^\infty 
f(\sqrt{z}\cos\theta,\sqrt{z}\sin\theta)e^{-z/2}\diff z\diff \theta \\ 
&=\frac{1}{2\pi}\int_0^{2\pi} \int_0^\infty f(r\cos\theta,r\sin\theta)e^{-r^2/2}r\diff z\diff \theta,
\end{align*}
en utilisant le changement de variable $z \mapsto r=z^2$.
Puis d'apr\`es la formule du passage en coordonn\'ees polaires, on a 
\[
\frac{1}{2\pi}\int_0^\infty\int_0^{2\pi}f(r\cos\theta,r\sin\theta)e^{-r^2/2}r\diff r\diff \theta
=\frac{1}{2\pi}\int_{\R^2}f(x,y)e^{-(x^2+y^2)/2}\diff x\diff y.
\] 
La loi de $(\sqrt{Z}\cos\Theta,\sqrt{Z}\sin\Theta)$ est donc $(2\pi)^{-1}e^{-(x^2+y^2)/2}\diff x\diff y$.
Les variables $\sqrt{Z}\cos\Theta$ et $\sqrt{Z}\sin\Theta$ sont ind\'ependantes et de loi $\cN(0,1)$. 
%%
\item Par la question 1., $Z = - 2 \ln U_1$ suit une loi exponentielle de paramètre $1/2$. De plus, $\Theta = 2\pi U_2$ suit la loi uniforme sur $[0,2\pi]$, et $Z$ et $\Theta$ sont indépendantes (en tant que fonctions mesurables de v.a. indépendantes).
Donc
\[
	X = \sqrt{- 2 \ln U_1} \cos (2\pi U_2)
	\qquad \text{et} \qquad 
	Y = \sqrt{- 2 \ln U_1} \sin (2\pi U_2)
\]
sont des v.a. indépendantes de loi gaussienne standard par la question 2.
\end{enumerate}
\end{comment}

%%%%%%
\separationexos
%%%%%%

\begin{exo}[Somme de v.a. gaussiennes] 
	Rappelons que la fonction caractéristique de la loi gaussienne $\cN(m,\sigma^2)$ est $\theta \in \R \mapsto \exp(\ii m \theta + \frac{1}{2} \sigma^2 \theta^2)$.
	 
	Soit $X_1,\dots,X_n$ des v.a. indépendantes telles que $X_k \sim \cN(m_k,\sigma_k^2)$. En calculant sa fonction caractéristique, déterminer la loi de $X_1 + \dots + X_n$.
\end{exo}

\begin{comment}
	Voir la fin du Chapitre 2 dans les notes de cours.
\end{comment}


%%%%%%
\separationexos
%%%%%%

\begin{exo}[Minimum de v.a. exponentielles] 
	\begin{enumerate}
		\item Calculer la fonction de répartition de la loi exponentielle de paramètre $\lambda > 0$.
		\item Soit $X_1,\dots,X_n$ des v.a. indépendantes de lois exponentielles de paramètres $\lambda_1,\dots,\lambda_n$. Déterminer la loi de $\min(X_1,\dots,X_n)$.
	\end{enumerate}
\end{exo}

\begin{comment}
\begin{enumerate}
\item Soit $X$ de loi exponentielle de paramètre $\lambda > 0$. On a, pour $a \geq 0$,
\[
F_X(a) = \int_0^a \lambda e^{-\lambda x} \diff x = 1 - e^{-\lambda a}
\]
et, pour $a < 0$, $F_X(a) = 0$. Donc $F_X(a) = (1 - e^{-\lambda a}) \1_{a \geq 0}$.
%%
\item Soit $Y = \min(X_1,\dots,X_n)$. Pour $a \geq 0$,
\begin{align*}
F_Y(a) 
& = \P(\min(X_1,\dots,X_n) \leq a) \\
& = 1-\P(\min(X_1,\dots,X_n)>a) \\
& = 1-\P(X_1>a,\dots,X_n>a) \\
& = 1-\P(X_1>a) \cdots \P(X_n>a) \\
& = 1- e^{-\lambda_1 a} \cdots e^{-\lambda_n a} \\
& = 1 - e^{-(\lambda_1+\dots+\lambda_n) a}
\end{align*}
et, pour $a < 0$, $F_Y(a) = 0$ car $Y > 0$ p.s.
Donc $F_Y$ est la fonction de répartition de la loi exponentielle de paramètre $\lambda_1+\dots+\lambda_n$. Donc $Y$ suit la loi exponentielle de paramètre $\lambda_1+\dots+\lambda_n$.
\end{enumerate}
\end{comment}


\partie{Compléments}


\begin{exo} 
	Soit $X$ et $Y$ des v.a. réelles. pour chacune des lois jointes suivantes, déterminer si $X$ et $Y$ sont indépendantes. Si elles le sont, trouver leurs lois marginales.
	\begin{enumerate}
		\item $\displaystyle \diff P_{(X,Y)}(x,y) = \frac{1}{24\pi} (2+6x^2+y^2+3x^2y^2)e^{-(x^2+y^2)/2} \diff x \diff y$;
		\item $\displaystyle \diff P_{(X,Y)}(x,y) = e^{-y} \1_{0<x<y} \diff x \diff y$;
		\item $\displaystyle P_{(X,Y)} = \sum_{k,\ell \in \N} \frac{1}{2} e^{-2}(e-1)^3 (k+\ell) e^{-(k+\ell)} \delta_{(k,\ell)}$;
		\item $\displaystyle P_{(X,Y)} = \sum_{k,\ell \in \N} \frac{1}{2} e^{-2} \frac{2^{k-\ell}}{k!} \delta_{(k,\ell)}$.
	\end{enumerate}
\end{exo}

\begin{comment}
    \begin{enumerate}
        \item On remarque que $(X,Y)$ est continue et sa densité se factorise ainsi :
        \[
            p_{(X,Y)}(x,y) 
            = \frac{1}{24\pi} (1+3x^2) (2+y^2) e^{-(x^2+y^2)/2}
            = \left( \frac{1}{24\pi} (1+3x^2) e^{-x^2/2} \right) 
            \left( (2+y^2)e^{-y^2/2} \right).
        \]
        Par le résultat vu en cours, cela montre que $X$ et $Y$ sont indépendantes. Pour trouver les lois marginales il faut trouver comment répartir le facteur constant. Pour cela on calcule
        \[
        \int_\R (1+3x^2)e^{-x^2/2} \diff x 
        = \int_\R e^{-x^2/2} \diff x + 3 \int_\R x^2 e^{-x^2/2} \diff x
        = \sqrt{2 \pi} + 3 \sqrt{2\pi} = 4 \sqrt{2 \pi},
        \]
        où l'on a utilisé pour la deuxième intégrale que si $Z \sim \cN(0,1)$ alors $\E[Z^2]=1$.
        On en déduit que la factorisation normalisée est
        \[
            p_{(X,Y)}(x,y) 
            = \left( \frac{1}{4\sqrt{2 \pi}} (1+3x^2)e^{-x^2/2} \right) 
            \left( \frac{1}{3\sqrt{2 \pi}}(2+y^2)e^{-y^2/2} \right).
        \]
        Donc les densités de $X$ et $Y$ sont
        \[
            p_X(x) 
            = \frac{1}{4\sqrt{2 \pi}} (1+3x^2) e^{-x^2/2} 
            \qquad \text{et} \qquad 
            p_Y(y) = \frac{1}{3\sqrt{2 \pi}}(2+y^2) e^{-y^2/2}.
        \]
        %%
        \item Il ne semble pas possible de factoriser la densité de $(X,Y)$. Montrons que $X$ et $Y$ ne sont pas indépendantes.
    
        \emph{Méthode 1.} On remarque que
        \[
            \P(X>1,Y<1) = 0,
        \]
        car $X<Y$ p.s.
        Mais
        \[
            \P(X>1) = \int_{\R^2} \1_{x>1} \diff P_{(X,Y)}(x,y)
            = \int_1^\infty \left( \int_x^\infty e^{-y} \diff y \right) \diff x
            > 0
        \]
        et 
        \[
            \P(Y<1) = \int_{\R^2} \1_{x>1} \diff P_{(X,Y)}(x,y)
            = \int_0^1  \left( \int_0^y \diff x \right) \diff y
            > 0.
        \]
        Donc $\P(X>1,Y<1) \neq \P(X>1)\P(Y<1)$.

        \emph{Méthode 2.} On peut calculer les densités marginales $p_X$ et $p_Y$ et vérifier que $p_{(X,Y)}(x,y) \neq p_X(x) p_Y(y)$ sur une partie de $\R^2$ de mesure de Lesbesgue non-nulle.
        %%
        \item Si $X$ et $Y$ étaient indépendantes alors on aurait 
        \[
        \forall k,\ell \in \N, \qquad \P(X=k,Y=\ell) = \P(X=k)\P(Y=\ell).
        \]
        Mais on a 
        \[
        \P(X=k,Y=\ell) = P_{(X,Y)}(\{(k,\ell)\}) = \frac{1}{2} e^{-2}(e-1)^3 (k+\ell) e^{-(k+\ell)},
        \]
        qui ne semble pas possible de factoriser comme le produit de quelque chose qui ne dépend que de $k$, fois quelque chose qui ne dépend que de $\ell$. Montrons donc que $X$ et $Y$ ne sont pas indépendantes.

        On va par exemple vérifier que 
        \[
        \P(X=0,Y=0) \neq \P(X=0) \P(Y=0).
        \]
        En effet, on a
        \[
        \P(X=0,Y=0) = \frac{1}{2} e^{-2}(e-1)^3 (0+0) e^{-(0+0)} = 0,
        \]
        mais 
        \[
        \P(X=0) = \sum_{\ell \in \N} \P(X=0,Y=\ell)
        = \sum_{\ell \in \N} \frac{1}{2} e^{-2}(e-1)^3 \ell e^{-\ell} 
        > 0
        \]
        et de même $\P(Y=0)>0$.
    \end{enumerate}
\end{comment}



%%%%%%
\separationexos
%%%%%%


\begin{exo} 
	Deux personnes ont rendez-vous à 14h00 mais elles sont peu ponctuelles: les instants d'arrivées $X$ et $Y$ sont deux v.a. indépendantes uniformément réparties dans $[14,15]$. 
	Calculer la loi de la durée d'attente du premier.
\end{exo}

%%%%%%
\separationexos
%%%%%%

\begin{exo}
	Trouver un exemple de v.a. réelles $X,Y,Z$ où $X$ est indépendante de $Y$, de $Z$ mais pas de $(Y,Z)$.
\end{exo}


\begin{comment}
On prend $X$ et $Y$ indépendantes de loi de Bernoulli de paramètre $1/2$ et $Z \coloneqq \1_{X=Y}$. Alors on a
\[
\Pp{Z=1,X=1} = \Pp{Y=1,X=1} = \Pp{Y=1} \Pp{X=1} = \frac{1}{2} \Pp{X=1} = \Pp{Z=1} \Pp{X=1},
\]
donc $X$ et $Z$ sont indépendantes (ce sont des indicatrices donc ce calcul suffit !).
Mais on a
\[
\Pp{(Y,Z)=(1,1),X=1} = \Pp{(Y,Z)=(1,1)} \neq \Pp{(Y,Z)=(1,1)} \Pp{X=1},
\]
car $(Y,Z)=(1,1) \Rightarrow X=1$.
\end{comment}

%%%%%%
\separationexos
%%%%%%

\begin{exo}
	Soit $\mu$ et $\nu$ deux mesures de probabilité sur $(\R,\cB(\R))$. Construire explicitement un espace de probabilité $(\Omega,\cA,\P)$ et deux v.a. réelles indépendantes $X$ et $Y$ telles que $P_X = \mu$ et $P_Y = \nu$.
\end{exo}


\begin{comment}
Il suffit de prendre $(\Omega,\cA,\P) = (\R^2,\cB(\R^2),\mu\otimes\nu)$, $X \colon (x,y) \in \Omega \mapsto x$ et $Y \colon (x,y) \in \Omega \mapsto y$.
\end{comment}


%%%%%%
\separationexos
%%%%%%

\begin{exo}
	Soient $X,Y,Z$ des v.a. réelles indépendantes de même loi.
	\begin{enumerate}
		\item Supposons que leur loi commune n'a pas d'atome. Calculer $\P(X<Y<Z)$.
		\item Montrer que le résultat de la question précédente n'est plus forcément vrai si leur loi peut avoir un atome.
	\end{enumerate}
\end{exo}



%%%%%%
\separationexos
%%%%%%



\begin{exo}[Convolution de mesures finies]
	Soit $\mu,\nu$ des mesures finies sur $(\R,\cB(\R))$. 
	La \emph{convolution} de $\mu$ et de $\nu$, not\'ee $\mu *\nu$, est %la mesure image de $\mu \otimes\nu$ par l'application $(x,y)\in \R^n \times \R^n \mapsto x+y$, c'est-\`a-dire que 
	la mesure sur $(\R,\cB(\R))$ définie par : pour toute fonction mesurable positive $f$ sur $\R$, 
	\[
	\int_{\R} f \diff (\mu*\nu) \coloneqq \int_{\R}\int_{\R} f(x+y) \diff\mu(x)\diff\nu(y).
	\]
	\begin{enumerate}
		\item Montrer que $\mu*\nu$ est bien définie et que c'est une mesure finie, dont on explicitera la masse totale.
		%%
		\item Soit $\mu,\nu,\rho$ trois mesures finies sur $(\R,\cB(\R))$. 
		Montrer les points suivants 
		\begin{enumerate}
			\item $\mu* \delta_0 = \mu$;
			%%
			\item $\mu*\nu = \nu *\mu$;
			%%
			\item $(\mu*\nu)*\rho = \mu*(\nu*\rho)$.
		\end{enumerate}
		%%
		\item \DSB{(Convolution et indépendance)}~
		Soit $X$ et $Y$ des v.a. réelles indépendantes. Montrer que la loi de $X+Y$ est $P_X * P_Y$.
	\end{enumerate}
\end{exo}

%%%%%%
\separationexos
%%%%%%

\begin{exo}[Statistique d'ordre]
    Sur un espace de probabilit\'e $(\Omega,\cA,\P)$, on se donne $X_1,\dots,X_n$ des v.a. indépendantes de loi uniforme sur $[0,1]$. 
    Notons $\cS_n$ le groupe des permutations de $\llbracket 1,n \rrbracket$.
	%%
	\begin{enumerate}
		\item Construire, à l'aide des événements $A_\sigma=\{X_{\sigma_1}<\dots<X_{\sigma_n}\}$ pour $\sigma\in \cS_n$, $n$ variables al\'eatoires $Y_1,\dots,Y_n$ sur $(\Omega,\cA,\P)$ telles que, pour presque tout $\omega\in\Omega$,
		\[
		Y_1(\omega) \leq \dots \leq Y_n(\omega)
		\quad \text{et} \quad
		\{Y_1(\omega),\dots,Y_n(\omega)\}=\{X_1(\omega),\dots,X_n(\omega)\}.
		\]
		%%
		\item D\'eterminer les lois des variables al\'eatoires $(Y_1,\dots,Y_n)$ et $(Y_1/Y_2,\dots,Y_{n-1}/Y_n)$. 
	\end{enumerate}
\end{exo}

\begin{comment}
\begin{enumerate}
\item Soit $i\in\{1,\ldots,n\}$. On peut d\'efinir pour presque tout $\omega\in\Omega$, 
\[
Y_i(\omega) 
= \sum_{\sigma\in\cS_n} X_{\sigma_i}(\omega) 
\1_{\{X_{\sigma_1}(\omega)<\dots<X_{\sigma_n}(\omega)\}}.
\] 
En effet, la mesure de Lebesgue des hyperplans de $\R^n$ o\`u deux coordonn\'ees sont \'egales est nulle. Ainsi, les variables al\'eatoires $Y_1,\ldots,Y_n$ sont bien d\'efinies.  
%%
\item Soit $f\colon [0,1]^n\to\R_+$ une fonction bor\'elienne. 
On a 
\begin{align*}
\E[f(Y_1,\dots,Y_n)]
&= \sum_{\sigma\in\cS_n} \int_{\{0\leq x_{\sigma_1}<\dots<x_{\sigma_n}\leq1\}} f(x_{\sigma_1},\dots,x_{\sigma_n}) \diff x_1 \cdots dx_n\\ 
&= n! \int_{\{0\leq x_1<\dots<x_n\leq1\}} f(x_1,\dots,x_n) \diff x_1\cdots dx_n.
\end{align*}
La loi de $(Y_1,\dots,Y_n)$ est donc 
\[
n!\1_{\{0\leq x_1<\dots<x_n\leq1\}} \diff x_1\cdots dx_n.
\]  
Soit maintenant $g \colon ]0,1[^{n-1}\to\R_+$ une fonction bor\'elienne. 
On a
\[
\E\left[ g\left(\frac{Y_1}{Y_2},\dots,\frac{Y_{n-1}}{Y_n}\right) \right]
= n! \int_{\{0<x_1<\dots<x_n<1\}} 
g\left(\frac{x_1}{x_2},\dots,\frac{x_{n-1}}{x_n}\right)\diff x_1 \cdots dx_n.
\]
Et $\phi \colon (x_1,\dots,x_n)\in\{0<x_1<\dots<x_n<1\} \mapsto \left(\frac{x_1}{x_2},\dots,\frac{x_{n-1}}{x_n},x_n\right) \in ]0,1[^n$ est un $C^1$-diff\'eomorphisme de jacobien \'egal \`a $(x_2\cdots x_n)^{-1}$. Ainsi, d'apr\`es la formule de changements de variables puis le th\'eor\`eme de Fubini-Tonelli, on a 
\begin{align*}
\E\left[ g\left(\frac{Y_1}{Y_2},\dots,\frac{Y_{n-1}}{Y_n}\right) \right]
&= n! \int_{]0,1[^n} g(u_1,\dots,u_{n-1}) u_2u_3^2\cdots u_{n-1}^{n-2}u_n^{n-1}
\diff u_1 \cdots \diff u_n\\ 
&= (n-1)! \int_{]0,1[^{n-1}} g(u_1,\dots,u_{n-1}) u_2u_3^2\cdots u_{n-1}^{n-2}
\diff u_1 \cdots \diff u_{n-1}.
\end{align*}
Donc la loi de $(Y_1/Y_2,\dots,Y_{n-1}/Y_n)$ est 
\[
(n-1)! u_2 u_3^2 \cdots u_{n-1}^{n-2} \1_{]0,1[^{n-1}}(u_1,\dots,u_{n-1})
\diff u_1 \cdots \diff u_{n-1}.
\] 
\emph{Remarque.} Les variables aléatoires $Y_1/Y_2,\dots,Y_{n-1}/Y_n$ sont ind\'ependantes, et chaque $Y_i/Y_{i+1}$ a pour loi $iy^{i-1}\1_{]0,1[}(y) \diff y$.
\end{enumerate}
\end{comment}


%\begin{exo}
%	Soient $X$ et $Y$ deux variables aléatoires indépendantes de même
%	loi uniforme sur $[0,1]$. On note $m = \min (X,Y)$ et $M = \max (X,Y)$.
%	\begin{enumerate}
%		\item Déterminer la loi du vecteur aléatoire $(m,M)$.
%		\item En déduire la loi de $m$ et celle de $M$.
%		\item Les deux variables $m$ et $M$ sont-elles indépendantes ? Justifiez
%		précisément. 
%	\end{enumerate}
%\end{exo}


%\begin{exo}[Lois de variables aléatoires vectorielles]
%	\begin{enumerate}
%		\item  Soit $(X,Y)$ un couple aléatoire de densité
%		$$f(x,y)=ke^{-\frac{x^2 + y^2}{2}}\1_{{\mathbb R}_+^2 \cup {\mathbb R}_-^2}(x,y).$$
%		\begin{itemize}
%			\item[a)]Déterminer $k$.
%			\item[b)]Calculer les lois marginales de $X$ et de $Y$. Les v.a.r. $X$ et $Y$ sont-elles indépendantes ?
%			\item[c)]Calculer $\text{Cov}(X,Y)$.	
%			\item[d)]Déterminer les lois de $Z=X+Y$ et de $U=X-Y$ en fonction de
%			$$\Psi(x)=\frac{1}{\sqrt{2\pi}}\int_{-\infty}^x e^{-\frac{t^2}{2}} dt .$$
%		\end{itemize}
%		\item Soit $(X,Y)$ un couple aléatoire de densité
%		$$f(x,y)=a\min (1,\frac{1}{x^2},\frac{1}{y^6}).$$
%		\begin{itemize}
%		\item[a)]Déterminer $a$, puis calculer les lois marginales de $X$ et $Y$. $X$ et $Y$ sont-elles indépendantes ?
%		\item[b)] $X$, $Y$, $(X,Y)$, $\min (|X|,|Y|)$ et $\min (X,Y)$ sont-elles intégrables ?
%		\end{itemize}
%		\item Soit $X,Y,Z$ des variables aléatoires réelles indépendantes de même loi $\text{Exp}(1)$.
%		On pose $S=X+Y+Z$.
%		\begin{itemize}
%			\item[a)]Déterminer la loi du vecteur aléatoire $\displaystyle (\frac{X}{S},\frac{Y}{S},S)$.
%			\item[b)]En déduire la loi de $S$, puis que la loi du vecteur $\displaystyle (\frac{X}{S},\frac{Y}{S})$
%			est uniforme sur un domaine qu'on aura précisé.
%			\item[c)]Que peut-on dire des variables $\displaystyle (\frac{X}{S},\frac{Y}{S})$ et $S$ ?
%		\end{itemize}
%	\end{enumerate}
%\end{exo}



%\begin{exo}
%	Soit $U$ et $V$ deux v.a. indépendantes de loi uniforme sur $]-1/2;1/2[.$
%	\begin{enumerate}
%		\item Montrer que $X=U+V$ admet une densité que l'on calculera.
%		\item Montrer que la fonction caractéristique de $X$ vaut 
%		$$\varphi_X(u)= \frac{2(1- \cos u)}{u^2}.$$
%	\end{enumerate}
%\end{exo}


%\begin{exo}
%	Deux joueurs $A$ et $B$ jouent une suite de parties indépendantes. Lors de chacune d'elles, ils ont respectivement les probabilités $p$ pour $A$ et $q=1-p$ pour $B$ de gagner. Le vainqueur final est celui des deux joueurs qui le premier obtient 2 victoires de plus que son adversaire. Quelle est la probabilité que $A$ soit vainqueur.
%\end{exo}


%\begin{exo}
%	{\bf Inégalité de Hoeffding}
%	\begin{enumerate} 
%		\item Préliminaire
%		Soit $Y$ une v.a. centrée telle qu'il existe $c<d $ vérifiant ${\mathbb P}[ c \leq Y\leq d]=1.$
%		\begin{enumerate}
%			\item Montrer que $c\leq 0 \leq d.$
%			\item Montrer que pour tout $s \in {\mathbb R}^+,$ en utilisant la convexité de $x \mapsto e^{sx}$ 
%			$${\mathbb E}[e^{sY} ]\leq \frac{d}{d-c} e^{sc}+ \frac{-c}{d-c} e^{sd}= f(s).$$
%			\item On pose $u= s(d-c),$ $\psi(u) =\ln f (s),$
%			en utilisant un développement de Taylor Lagrange montrer que $\psi(u)\leq \frac{u^2}{8}.$
%			\item En déduire que ${\mathbb E}[ e^{sY}] \leq e^{\frac{s^2(d-c)^2}{8}}.$
%		\end{enumerate}
%		\item Soient $(X_i)_{i=1}^n$ v.a. indépendantes telles que pour tout $k=1,...,n$ il existe $a_k \leq b_k$ deux nombres réels tels que 
%		$$a_k \leq X_k \leq b_k.$$
%		On note $S_n=X_1+...+X_n.$
%		\begin{enumerate}
%			\item Montrer que ${\mathbb P}[ S_n- {\mathbb E}[ S_n] \geq t] \leq {\mathbb E}[ e^{s( S_n- {\mathbb E}[ S_n] )-st}].$
%			\item Montrer que ${\mathbb P}[ S_n- {\mathbb E}[ S_n] \geq t] \leq e^{-st + s^2\sum_{k=1}^n \frac{(b_k-a_k)^2}{8}}.$
%			\item En optimisant en $s$ montrer que
%			$${\mathbb P}[ S_n- {\mathbb E}[ S_n] \geq t]\leq e^{ -\frac{2t^2}{\sum_{k=1}^n (b_k-a_k)^2}}.$$
%			\item Par la même méthode en déduire une majoration de ${\mathbb P}[ S_n- {\mathbb E}[ S_n] \leq -t].$
%			\item En déduire une majoration de ${\mathbb P}[\left| S_n- {\mathbb E}[ S_n] \right|\geq t].$
%			\item Comparer avec la majoration obtenue avec l'inégalité de Bienaymé-Tchebitchev
%		\end{enumerate}
%	\end{enumerate}
%\end{exo}
%
%\begin{exo}
%	
%	Soient $U$ et $X=(X_1,...,X_n)$ deux v.a. indépendantes, $U$ à valeurs dans $\{1,....,n\}$ et $X$ ayant une densité par rapport à la mesure de lebesgue sur ${\mathbb R}^n.$ On note $X_U(\omega)= X_{U(\omega)}(\omega).$
%	\begin{enumerate}
%		\item Montrer que $X_U$ est une v.a.
%		\item Montrer que $X_U$ admet une densité par rapport à la mesure de Lebesgue.
%		\item Etudier le cas où $U$ suit une loi uniforme sur $\{1,...,n\}$ et les $X_i$ sont indépedantes de même loi.
%	\end{enumerate}
%\end{exo}

%\begin{exo}\textbf{(Introduction à la convergence p.s.)}
%On considère une suite $(Y_n)_{n\geq 0}$ de variables aléatoires sur
%un espace probabilisé $(\Omega,\mathcal A,\mathbb{P})$ à valeurs dans
%$({\mathbb R}^+,\cB ({\mathbb R}^+))$.
%
%\begin{enumerate}
%
%\item Montrer que si $\displaystyle \forall k \in {\mathbb N}^{\ast}, Y_k \leq \frac{1}{k^2}$   $\mathbb{P}$ p.s.,
%alors $\displaystyle \sum_{k=1}^{+\infty} Y_k$ converge $\mathbb{P}$ p.s.
%
%\item On définit les ensembles $A_{m,n} = \{Y_0 + Y_1 + \cdots + Y_m \geq n\}$.\\
%
%Parmi les quatre évènements $\underset {n=0}{\overset
%{+\infty}{\bigcap}}\underset {m=0}{\overset
%{+\infty}{\bigcup}}A_{m,n}$, $\underset {m=0}{\overset
%{+\infty}{\bigcap}}\underset {n=0}{\overset
%{+\infty}{\bigcup}}A_{m,n}$, $\underset {n=0}{\overset
%{+\infty}{\bigcup}}\underset {m=0}{\overset
%{+\infty}{\bigcap}}A_{m,n}$, $\underset {m=0}{\overset
%{+\infty}{\bigcup}}\underset {n=0}{\overset
%{+\infty}{\bigcap}}A_{m,n}$, lequel correspond à l'évènement $\{\sum
%Y_k$  diverge $\}$ ? A quoi correspondent les trois autres ?
%
%\item Montrer que si $\sum Y_k$ diverge $\mathbb{P}$ p.s., alors
%$\forall n \in \mathbb N$, $\displaystyle \lim_{m\rightarrow
%+\infty} \mathbb{P}(\{Y_0 + Y_1 + \cdots + Y_m \geq n\})= 1$.
%\end{enumerate}
%\end{exo}


%\begin{exo}\textbf{(Intervalle de confiance d'une proportion)}\\
%	Pour modéliser le résultat d'un sondage ayant deux réponses (oui ou non) ou le lancé d'une 
%	pièce par exemple, on considère sur $(\Omega, \cA)$, une famille $(\P_p)_{p \in [0,1]}$ de mesures de probabilité
%	et une suite $(X_n)_n$ de v.a. telle que, pour chaque $p \in [0,1]$, sous la probabilité $\P_p$,
%	$(X_n)_n$ est i.i.d. de loi $P_p$ de Bernoulli $B(1,p)$. Remarque : $(\Omega, \cA,(\P_p)_{p \in [0,1]})$
%	est dit \emph{un modèle statistique}.
%	
%	$(X_1, \dots, X_n)$ est alors un $n$-échantillon de loi $B(1,p)$, avec $p$ inconnu, à estimer.
%	\begin{enumerate}
%		\item \label{majoration} Majorer $\P_p( |\bar{X}_n - p | \geq \varepsilon) $ à l'aide de l'inégalité de Bienaymé-Tchebychev.
%		\item En utilisant en outre $\max_{p \in [0,1]} p(1-p) = 1/4$, montrer que, pour tout $\alpha \in ]0,1[$,
%		$$
%		\P_p \left(|\bar{X}_n - p | \geq \frac{1}{2 \sqrt{n \alpha}} \right) \leq \alpha.
%		$$
%		L'intervalle aléatoire $I_n \equiv \left]\bar{X}_n - \frac{1}{2 \sqrt{n \alpha}},  \bar{X}_n + \frac{1}{2 \sqrt{n \alpha}}\right[$
%		satisfait alors 
%		$$
%		\P_p( p \in I_n ) \geq 1-\alpha.
%		$$
%		On dit que c'est un intervalle de confiance pour $p$ de risque $\alpha$. 
%		\item \textbf{Question supplémentaire.} On va construire ici un intervalle de confiance plus précis, sans utiliser la majoration $p(1-p) \leq 1/4$. 
%		Montrer que $ |\bar{X}_n - p | \geq \sqrt{\frac{p(1-p)}{n \alpha}}$ équivaut à $(n \alpha + 1) \, p^2 + (-2 \, n \alpha \, \bar{X}_n -1) \, p 
%		+ n \alpha \bar{X}_n^2 \geq 0$ ou encore à $p \notin I_n = ]a_n, b_n[$ où $a_n$ et $b_n$ sont les racines (aléatoires)
%		du trinôme précédent. En déduire que $I_n$ est un intervalle de confiance de $p$ de risque $\alpha$ et l'expliciter. 
%	\end{enumerate}
%\end{exo}


%\begin{exo} \label{exo indep piles et faces}
%	Soit $(X_n)_{n\geq1}$ une suite de variables al\'eatoires ind\'ependantes de loi de Bernoulli de paramètre $p \in \,]0,1[$.
%	Soit $N$ une variable aléatoire de loi de Poisson de param\`etre $\lambda >0$, c'est-à-dire vérifiant
%	\[
%	\forall k \in \N, \quad \Pp{N=k} = \frac{\lambda^k}{k!} e^{-\lambda}.
%	\]
%	On suppose que $N$ est indépendante de $(X_n)_{n\geq1}$.
%	On pose 
%	\[
%	P \coloneqq \sum_{i=1}^N X_i
%	\quad \text{et} \quad 
%	F \coloneqq N-P=\sum_{i=1}^N(1-X_i),
%	\]
%	avec $P=F=0$ sur $\{N=0\}$. Les variables al\'eatoires $P$ et $F$ repr\'esentent respectivement le nombre de piles et de faces dans un jeu de pile ou face de param\`etre $p$ \`a $N$ lancers.
%	%%
%	\begin{enumerate}
%		\item D\'eterminer la loi du couple $(P,N)$.
%		%%
%		\item En d\'eduire les lois de $P$ et $F$ et montrer que $P$ et $F$ sont ind\'ependantes.
%	\end{enumerate}
%\end{exo}
%
%
%\begin{comment}
%\begin{enumerate}
%\item On a $\P(P=0,N=0) = \P(N=0) = e^{-\lambda}$, et pour $n\geq1$ et $0\leq k\leq n$, 
%\begin{align*}
%\P(P=k,N=n) 
%& = \P\left(N=n,\sum_{i=1}^nX_i=k\right)
%= \P(N=n)\P\left(\sum_{i=1}^nX_i=k\right) 
%= e^{-\lambda}\frac{\lambda^n}{n!} \binom{n}{k}p^k(1-p)^{n-k}\\ 
%& = e^{-\lambda}\frac{(\lambda p)^k}{k!}\frac{(\lambda(1-p))^{n-k}}{(n-k)!}.
%\end{align*}
%%%
%\item On a pour $k,l\geq0$, 
%\[
%\P(P=k,F=l)
%=\P(P=k,N=k+l)
%=\left(e^{-\lambda p}\frac{(\lambda p)^k}{k!}\right)
%\left(e^{-\lambda(1-p)}\frac{(\lambda(1-p))^l}{l!}\right).
%\] 
%Donc les variables al\'eatoires $P$ et $F$ sont ind\'ependantes et de lois respectives les lois de Poisson de param\`etres $\lambda p$ et $\lambda(1-p)$. 
%
%\emph{Remarque.} On utilise en fait ici le petit résultat suivant. Soit $X$ et $Y$ deux variables aléatoires à valeurs respectivement dans des espaces dénombrables $I$ et $J$ telles que pour tous $i \in I$ et $j \in J$, on ait $\P(X=i,Y=j)=p_i q_j$. 
%Alors $X$ et $Y$ sont indépendantes et on a, pour tout $i \in I$, $\P(X = i) = p_i/c$ où $c \coloneqq \sum_{i \in I} p_i$ et, pour tout $j \in J$, $\P(Y = j) = c q_j $. 
%
%Montrons ce résultat. Soit $j \in J$, on a
%\[
%\P(Y = j) 
%= \sum_{i\in I} \P(X=i,Y=j) 
%= \sum_{i\in I} p_i q_j
%= q_j c.
%\]
%Comme $\sum_{j\in J} \P(Y=j) = 1$, on en déduit que $\sum_{j\in J} q_j = 1/c$
%Puis de la même manière, on a, pour $i \in I$
%\[
%\P(X = i) 
%= \sum_{j\in J} \P(X=i,Y=j) 
%= p_i \sum_{j\in J} q_j
%= p_i / c.
%\]
%Alors, l'hypothèse se réécrit $\P(X=i,Y=j)=\P(X=i) \P(Y=j)$ pour tous $i \in I$ et $j \in J$, ce qui montre l'indépendance entre $X$ et $Y$ (car on est sur un espace dénombrable donc les $P_{(X,Y)} (\{(i,j)\})$ caractérisent $P_{(X,Y)}$).
%\end{enumerate}
%\end{comment}
%
%
%\begin{exo}
%	Soit $(X_n)_{n\geq1}$ une suite de variables al\'eatoires r\'eelles ind\'ependantes  de même loi et $N$ une variable al\'eatoire \`a valeurs dans $\N$ ind\'ependante de la suite $(X_n)_{n\geq1}$. 
%	Soit $f\colon \R\to\R_+$ une fonction mesurable. Montrer que
%	\[
%	\Ec{\sum_{i=1}^N f(X_i)} = \Ec{N} \Ec{f(X_1)},
%	\]
%	où l'on adopte la convention qu'une somme vide est nulle.
%\end{exo}
%
%\begin{comment}
%On a 
%\begin{align*}
%\Ec{\sum_{i=1}^N f(X_i)}
%&=\Ec{\sum_{n\geq1}\1_{\{N=n\}}\sum_{i=1}^nf(X_i)}
%=\sum_{n\geq1} \Ec{\1_{\{N=n\}}\sum_{i=1}^nf(X_i)} 
%=\sum_{n\geq1} \P(N=n) \Ec{\sum_{i=1}^nf(X_i)}\\ 
%&=\sum_{n\geq1} \P(N=n) n \Ec{f(X_1)}
%=\E[N] \Ec{f(X_1)},
%\end{align*} 
%où l'on a utilisé que $N$ est indépendante de $(X_1,\dots,X_n)$.
%\end{comment}
\end{document}