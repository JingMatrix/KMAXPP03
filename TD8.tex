\documentclass[a4paper,11pt]{article}
\usepackage[utf8]{inputenc}
\usepackage[T1]{fontenc}
\usepackage{lmodern}

\usepackage{amsthm,amsmath,amsfonts,amssymb,bbm,mathrsfs,stmaryrd}
\usepackage{mathtools}
\usepackage{enumitem}
\usepackage{url}
\usepackage{dsfont}
\usepackage{appendix}
\usepackage{amsthm}
\usepackage[dvipsnames,svgnames]{xcolor}
\usepackage{graphicx}

\usepackage{fancyhdr,lastpage,titlesec,verbatim,ifthen}

\usepackage[colorlinks=true, linkcolor=black, urlcolor=black, citecolor=black]{hyperref}

\usepackage[french]{babel}

\usepackage{caption,tikz,subfigure}

\usepackage[top=2cm, bottom=2cm, left=2cm, right=2cm]{geometry}

%%%%%%%%%%%%%%%%%%%%%%%%%%%%%%%%%%%%%%%%%%%%%%%%%%%%%%%%%%%%%%%%%%%%%%%

%%%%%%%% Taille de la legende des images %%%%%%%%%%%%%%%%%%%%%%%%%%%%%%%
\renewcommand{\captionfont}{\footnotesize}
\renewcommand{\captionlabelfont}{\footnotesize}

%%%%%%%% Numeration des enumerates en romain et chgt de l'espace %%%%%%%
\setitemize[1]{label=$\rhd$, font=\color{NavyBlue},leftmargin=0.8cm}
\setenumerate[1]{font=\color{NavyBlue},leftmargin=0.8cm}
\setenumerate[2]{font=\color{NavyBlue},leftmargin=0.49cm}
%\setlist[enumerate,1]{label=(\roman*), font = \normalfont,itemsep=4pt,topsep=4pt} 
%\setlist[itemize,1]{label=\textbullet, font = \normalfont,itemsep=4pt,topsep=4pt} 

%%%%%%%% Pas d'espacement supplementaire avant \left et apres \right %%%
%%%%%%%% Note : pour les \Big(, utiliser \Bigl( \Bigr) %%%%%%%%%%%%%%%%%
\let\originalleft\left
\let\originalright\right
\renewcommand{\left}{\mathopen{}\mathclose\bgroup\originalleft}
\renewcommand{\right}{\aftergroup\egroup\originalright}

%%%%%%%%%%%%%%%%%%%%%%%%%%%%%%%%%%%%%%%%%%%%%%%%%%%%%%%%%%%%%%%%%%%%%%%

\newcommand{\N}{\mathbb{N}}
\newcommand{\Z}{\mathbb{Z}}
\newcommand{\Q}{\mathbb{Q}}
\newcommand{\R}{\mathbb{R}}
\newcommand{\C}{\mathbb{C}}
\newcommand{\T}{\mathbb{T}}
\renewcommand{\P}{\mathbb{P}}
\newcommand{\E}{\mathbb{E}}

\newcommand{\1}{\mathbbm{1}}

\newcommand{\cA}{\mathcal{A}}
\newcommand{\cB}{\mathcal{B}}
\newcommand{\cC}{\mathcal{C}}
\newcommand{\cE}{\mathcal{E}}
\newcommand{\cF}{\mathcal{F}}
\newcommand{\cG}{\mathcal{G}}
\newcommand{\cH}{\mathcal{H}}
\newcommand{\cI}{\mathcal{I}}
\newcommand{\cJ}{\mathcal{J}}
\newcommand{\cL}{\mathcal{L}}
\newcommand{\cM}{\mathcal{M}}
\newcommand{\cN}{\mathcal{N}}
\newcommand{\cP}{\mathcal{P}}
\newcommand{\cS}{\mathcal{S}}
\newcommand{\cT}{\mathcal{T}}
\newcommand{\cU}{\mathcal{U}}

\newcommand{\Ec}[1]{\mathbb{E} \left[#1\right]}
\newcommand{\Pp}[1]{\mathbb{P} \left(#1\right)}
\newcommand{\Ppsq}[2]{\mathbb{P} \left(#1\middle|#2\right)}

\newcommand{\e}{\varepsilon}

\newcommand{\ii}{\mathrm{i}}
\DeclareMathOperator{\re}{Re}
\DeclareMathOperator{\im}{Im}
\DeclareMathOperator{\Arg}{Arg}

\newcommand{\diff}{\mathop{}\mathopen{}\mathrm{d}}
\DeclareMathOperator{\Var}{Var}
\DeclareMathOperator{\Cov}{Cov}
\newcommand{\supp}{\mathrm{supp}}

\newcommand{\abs}[1]{\left\lvert#1\right\rvert}
\newcommand{\abso}[1]{\lvert#1\rvert}
\newcommand{\norme}[1]{\left\lVert#1\right\rVert}
\newcommand{\ps}[2]{\langle #1,#2 \rangle}

\newcommand{\petito}[1]{o\mathopen{}\left(#1\right)}
\newcommand{\grandO}[1]{O\mathopen{}\left(#1\right)}

\newcommand\relphantom[1]{\mathrel{\phantom{#1}}}

\newcommand{\NB}[1]{{\color{NavyBlue}#1}}
\newcommand{\DSB}[1]{{\color{DarkSlateBlue}#1}}

%%%%%%%% Theorems styles %%%%%%%%%%%%%%%%%%%%%%%%%%%%%%%%%%%%%%%%%%%%%%
\theoremstyle{plain}
\newtheorem{theorem}{Theorem}[section]
\newtheorem{proposition}[theorem]{Proposition}
\newtheorem{lemma}[theorem]{Lemme}
\newtheorem{corollary}[theorem]{Corollaire}
\newtheorem{conjecture}[theorem]{Conjecture}
\newtheorem{definition}[theorem]{Définition}

\theoremstyle{definition}
\newtheorem{remark}[theorem]{Remarque}
\newtheorem{example}[theorem]{Exemple}
\newtheorem{question}[theorem]{Question}

%%%%%%%% Macros spéciales TD %%%%%%%%%%%%%%%%%%%%%%%%%%%%%%%%%%%%%%%%%%

%%%%%%%%%%%% Changer numérotation des pages %%%%%%%%%%%%%%%%%%%%%%%%%%%%
\pagestyle{fancy}
\cfoot{\thepage/\pageref{LastPage}} %%% numéroter page / total de pages
\renewcommand{\headrulewidth}{0pt} %%% empêcher qu'il y ait une ligne horizontale en haut
%%%%%%%%%%%% Ne pas numéroter les pages %%%%%%%%%%%%%%%%%
%\pagestyle{empty}

%%%%%%%%%%%% Supprimer les alineas %%%%%%%%%%%%%%%%%%%%%%%%%%%%%%%%%%%%%
\setlength{\parindent}{0cm} 

%%%%%%%%%%%% Exercice %%%%%%%%%%%%%%%%%%%%%%%%%%%%%%%%%% 
\newcounter{exo}
\newenvironment{exo}[1][vide]
{\refstepcounter{exo}
	{\noindent \textcolor{DarkSlateBlue}{\textbf{Exercice \theexo.}}}
	\ifthenelse{\equal{#1}{vide}}{}{\textcolor{DarkSlateBlue}{(#1)}}
}{}

%%%%%%%%%%%% Partie %%%%%%%%%%%%%%%%%%%%%%%%%%%%%%%%%%%%
\newcounter{partie}
\newcommand\partie[1]{
	\stepcounter{partie}%
	{\bigskip\large\textbf{\DSB{\thepartie.~#1}}\bigskip}
	}

%%%%%%%%%%%% Separateur entre les exos %%%%%%%%%%%%%%%%%
\newcommand{\separationexos}{
	\bigskip
%	{\centering\hfill\DSB{\rule{0.4\linewidth}{1.2pt}}\hfill}\medskip
	}

%%%%%%%%%%%% Corrige %%%%%%%%%%%%%%%%%%%%%%%%%%%%%%%%%%% 
%\renewenvironment{comment}{\medskip\noindent \textcolor{BrickRed}{\textbf{Corrigé.}}}{}

%%%%%%%%%%%% Titre %%%%%%%%%%%%%%%%%%%%%%%%%%%%%%%%%%%%%%
\newcommand\titre[1]{\ \vspace{-1cm}
	
	\DSB{\rule{\linewidth}{1.2pt}}
	{\small Probabilités et statistiques continues avancées}
	\hfill {\small Université Paul Sabatier}
	
	{\small KMAXPP03}
	\hfill {\small Licence 3, Printemps 2023}\medskip
	\begin{center}
		{\Large\textbf{\DSB{#1}}}\vspace{-.2cm}
	\end{center}
	\DSB{\rule{\linewidth}{1.2pt}}\medskip
}

%%%%%%%%%%%%%%%%%%%%%%%%%%%%%%%%%%%%%%%%%%%%%%%%%%%%%%%%%%%%%%%%%%%%%%%
\begin{document}
%%%%%%%%%%%%%%%%%%%%%%%%%%%%%%%%%%%%%%%%%%%%%%%%%%%%%%%%%%%%%%%%%%%%%%%

\titre{TD 8 -- Convergence de variables aléatoires}

%%%%%%%%%%%%%%%%%%%%%%%%%%%%%%%%%%%%%%%%%%%%%%%%%%%
\partie{Convergence dans $L^p$}
%%%%%%%%%%%%%%%%%%%%%%%%%%%%%%%%%%%%%%%%%%%%%%%%%%%



\begin{exo} 
	Soit $(X_n)_{n\geq 1}$ une suite de v.a. réelles convergeant p.s. vers une v.a. réelle $X$.
	Soit $f \colon \R \to \R$ continue bornée.
	Montrer que, pour tout $p \in [1,\infty[$,
	\[
	f(X_n) \xrightarrow[n\to\infty]{L^p} f(X).
	\]
\end{exo}


\begin{comment}
	Comme $f$ est continue, on a
	\[
	f(X_n) \xrightarrow[n\to\infty]{\text{p.s.}} f(X).
	\]
	et donc
	\[
	\abs{f(X_n)-f(X)}^p \xrightarrow[n\to\infty]{\text{p.s.}} 0.
	\]
	En outre, $f$ est bornée, donc il existe $M > 0$ tel que, pour tout $x \in \R$, $\abs{f(x)} \leq M$.
	Alors on a 
	\[
	\abs{f(X_n)-f(X)}^p \leq \left( \abs{f(X_n)} + \abs{f(X)} \right)^p
	\leq (2M)^p,
	\]
	et $\E[(2M)^p] = (2M)^p < \infty$ donc cela nous fournit une domination.
	Par convergence dominée, on a donc
	\[
	\Ec{\abs{f(X_n)-f(X)}^p} \xrightarrow[n\to\infty]{} 0,
	\]
	ce qui montre la convergence $L^p$.
\end{comment}


%%%%%%
\separationexos
%%%%%%


\begin{exo} 
	Soit $X \in L^1$. Montrer que $(n\sin(X/n))_{n\geq 1}$ converge dans $L^1$ vers une limite à identifier.
\end{exo}


\begin{comment}
On a $n\sin(X/n) \to X$ p.s. donc $n\sin(X/n) - X \to 0$.
D'autre part, $\abs{n\sin(X/n)} \leq \abs{X}$ donc $\abs{n\sin(X/n)-X} \leq 2\abs{X}$ qui est intégrable. Donc, par convergence dominée, on a
\[
\Ec{\abs{n\sin(X/n)-X}} \xrightarrow[n\to\infty]{} 0.
\]
\end{comment}

%%%%%%
\separationexos
%%%%%%

\begin{exo} 
	Soit $(X_k)_{k\geq 1}$ une suite de v.a. indépendantes dans $L^2$. On dit que la série $\sum_{k\geq1} X_k$ converge dans $L^2$ si ses sommes partielles $\sum_{k=1}^n X_k$ convergent dans $L^2$.
	\begin{enumerate}
		\item Supposons que les séries $\sum_{k\geq 1} \E[X_k]$ et $\sum_{k\geq 1} \Var(X_k)$ convergent.
		\begin{enumerate}
			\item Pour $n \geq 1$, soit $Z_n = \sum_{k=1}^n (X_k-\E[X_k])$. Montrer que $(Z_n)_{n\geq1}$ est de Cauchy dans $L^2$.
			\item En déduire que $\sum_{k\geq1} X_k$ converge dans $L^2$.
		\end{enumerate}
		\item Supposons que $\sum_{k\geq1} X_k$ converge dans $L^2$.
		Montrer que $\sum_{k\geq 1} \E[X_k]$ et $\sum_{k\geq 1} \Var(X_k)$ convergent.
	\end{enumerate}
\end{exo}

%%%%%%%%%%%%%%%%%%%%%%%%%%%%%%%%%%%%%%%%%%%%%%%%%%%
\partie{Convergence en probabilité}
%%%%%%%%%%%%%%%%%%%%%%%%%%%%%%%%%%%%%%%%%%%%%%%%%%%


\begin{exo} Soit $(X_n)_{n\geq 1}$ une suite de v.a. i.i.d. 
	\begin{enumerate}
		\item Montrer que \[ \frac{X_n}{n} \xrightarrow[n\to\infty]{\P} 0.\]
		\item Soit $p \in[1,\infty[$. Trouver une condition nécessaire et suffisante (en terme de moments de $X_1$) pour qu'il y ait convergence dans $L^p$.
	\end{enumerate}
\end{exo}


%%%%%%
\separationexos
%%%%%%


\begin{exo}[Stabilité par opérations]
	Soit $(X_n)_{n\geq 1}$ et $(Y_n)_{n\geq 1}$ des suites de v.a. réelles convergeant en probabilité vers $X$ et $Y$ respectivement.
	\begin{enumerate}
		\item 
		\begin{enumerate}
			\item Montrer que \[X_n+Y_n \xrightarrow[n\to\infty]{\P} X+Y.\]
			\item En déduire que \[ X_n-Y_n \xrightarrow[n\to\infty]{\P} X-Y.\]
		\end{enumerate}
		%%
		\item On veut montrer la stabilité par produit. Soit $\varepsilon>0$. 
		\begin{enumerate}
			\item Montrer que 
			\[
			\Pp{ \abs{X_nY_n - XY} \geq \varepsilon }
			\leq \Pp{ \abs{X_n-X} \cdot \abs{Y_n} \geq \frac{\varepsilon}{2} }
			+ \Pp{ \abs{X} \cdot \abs{Y_n-Y} \geq \frac{\varepsilon}{2} }
			\]
			\item Soit $\delta > 0$. Montrer qu'il existe $M > 0$ tel que
			$\Pp{\abs{X} \geq M} \leq \delta $ et $\Pp{\abs{Y} \geq M} \leq \delta$.
			\item Montrer que, pour $n$ suffisamment grand, $\Pp{\abs{Y_n} \geq M+1} \leq 2 \delta$.
			\item En déduire que 
			\[
			X_nY_n \xrightarrow[n\to\infty]{\P} XY.
			\]
		\end{enumerate}
		%%
		\item Supposons $\P(Y = 0)= 0$. On veut montrer la stabilité par quotient.
		\begin{enumerate}
			\item Soit $\delta > 0$. Montrer qu'il existe $\eta > 0$ tel que $\Pp{\abs{Y} \leq \eta} \leq \delta$.
			\item Montrer que, pour $n$ suffisamment grand, $\Pp{\abs{Y_n} \geq \eta/2} \leq 2 \delta$.
			\item En déduire que 
			\[ 
			\frac{1}{Y_n} \xrightarrow[n\to\infty]{\P} \frac{1}{Y}.
			\]
			\item Conclure que 
			\[ 
			\frac{X_n}{Y_n} \xrightarrow[n\to\infty]{\P} \frac{X}{Y}.
			\]
		\end{enumerate}
		%%
		\item Soit $f\colon \R \to \R$ est continue. En utilisant la question 2.(a) et le fait qu'une fonction continue sur un compact est uniformément continue, montrer que 
		\[
		f(X_n) \xrightarrow[n\to\infty]{\P} f(X).
		\]
	\end{enumerate}
\end{exo}




%%%%%%%%%%%%%%%%%%%%%%%%%%%%%%%%%%%%%%%%%%%%%%%%%%%
\partie{Compléments}
%%%%%%%%%%%%%%%%%%%%%%%%%%%%%%%%%%%%%%%%%%%%%%%%%%%




\begin{exo} 
	Soit $(X_n)_{n\geq 1}$ une suite de v.a. réelles convergeant en probabilité vers une v.a. réelle $X$. On suppose qu'il existe $M > 0$ tel que $\abs{X} \leq M$ p.s. et, pour tout $n \geq 1$, $\abs{X_n} \leq M$ p.s.
	Montrer que, pour tout $p \in [1,\infty[$,
	\[
	X_n \xrightarrow[n\to\infty]{L^p} X.
	\]
\end{exo}


%%%%%%
\separationexos
%%%%%%

%On note $L^0(\Omega,\cA,\P)$, abrégé en $L^0$, l'espace des v.a. réelles quotienté par la relation d'équivalence ``être égal p.s.'' (c'est-à-dire où l'on considère deux variables aléatoires égales p.s. comme identiques).
%Cet espace correspond au cas $p=0$ des espaces $L^p$.
%Le résultat suivant fournit une métrique sur $L^0$ qui correspond à la convergence en probabilité.
%Notons que le choix de cette métrique n'est pas unique.
%
%\begin{theorem}
%	Pour $X$ et $Y$ des v.a. réelles, on définit 
%	\[
%		d(X,Y) = \Ec{\abs{X-Y} \wedge 1}.
%	\]
%	Alors, $d$ est une distance sur $L^0$ qui métrise la convergence en probabilité, c'est-à-dire que, pour $X,X_1,X_2,\dots \in L^0$,
%	\[
%	X_n \xrightarrow[n\to\infty]{\P} X
%	\quad \Leftrightarrow \quad 
%	d(X_n,X) \xrightarrow[n\to\infty]{} 0.
%	\]
%\end{theorem}
%
%\begin{proof}
%	
%\end{proof}

\begin{exo}
	On note $L^0(\Omega,\cA,\P)$, abrégé en $L^0$, l'espace des v.a. réelles quotienté par la relation d'équivalence ``être égal p.s.''
	Pour $X$ et $Y$ des variables aléatoires réelles, on définit
	\[
	d(X,Y) \coloneqq \Ec{\min(1,\abs{X-Y})}.
	\]
	\begin{enumerate}
		\item Montrer que $d$ est une distance sur $L^0$.
		%%
		\item Montrer que $d$ métrise la convergence en probabilité, c'est-à-dire que, pour $X,X_1,X_2,\dots \in L^0$,
		\[
			X_n \xrightarrow[n\to\infty]{\P} X
			\quad \Leftrightarrow \quad 
			d(X_n,X) \xrightarrow[n\to\infty]{} 0.
		\]
		%%
		\item En utilisant cette nouvelle caractérisation de la convergence en probabilité, montrer de nouveau que les convergences p.s. et $L^1$ impliquent la convergence en probabilité.
		%%
%		\item Montrer que $L^0$ muni de $d$ est complet.
	\end{enumerate}
\end{exo}

\begin{comment}
\begin{enumerate}
\item Il est clair que $d(X,Y) = d(Y,X)$. 

Le fait que $d(X,Y) = 0 \Leftrightarrow X = Y$ p.s. ce montre en utilisant qu'une v.a. positive d'espérance nulle est nulle p.s.

Pour l'inégalité triangulaire, on vérifie que, pour $X,Y,Z$ des v.a. réelles,
\[
\min(1,\abs{X-Z})
\leq \min(1,\abs{X-Y}) + \min(1,\abs{Y-Z})
\] 
en distinguant selon si $\abs{X-Y} \leq 1$ ou non et si $\abs{Y-Z} \leq 1$ ou non.
Cela donne $d(X,Z) \leq d(X,Y) + d(Y,Z)$.

Donc $d$ définit bien une distance.
%%
\item Soit $X,X_1,X_2,\dots \in L^0$.

Supposons que $X_n$ converge en probabilité vers $X$.
Soit $\varepsilon > 0$. 
Alors, en coupant l'espérance en deux en distinguant selon si $\abs{X_n-X} < \varepsilon$ ou $\abs{X_n-X} \geq \varepsilon$, on a
\begin{align*}
d(X_n,X) 
& = \Ec{\min(1,\abs{X_n-X})} \\
& = \Ec{\min(1,\abs{X_n-X}) \1_{\{\abs{X_n-X} < \varepsilon\}}}
+ \Ec{\min(1,\abs{X_n-X}) \1_{\{\abs{X_n-X} \geq \varepsilon\}}} \\
& \leq \Ec{\min(1,\varepsilon) \1_{\{\abs{X_n-X} < \varepsilon\}}}
+ \Ec{\1_{\{\abs{X_n-X} \geq \varepsilon\}}} \\
& \leq \varepsilon + \Pp{\abs{X_n-X} \geq \varepsilon} \\
& \leq 2 \varepsilon,
\end{align*}
pour $n$ suffisamment grand. Ça montre que $d(X_n,X) \to 0$.


Réciproquement, supposons que $d(X_n,X) \to 0$.
Soit $\varepsilon > 0$. Si $\varepsilon \leq 1$, alors
\[
\Pp{\abs{X_n-X} \geq \varepsilon } 
= \Pp{\min(1,\abs{X_n-X}) \geq \varepsilon} 
\leq \frac{\E[\min(1,\abs{X_n-X})]}{\varepsilon}
= \frac{d(X_n,X)}{\varepsilon},
\]
où l'on a utilisé l'inégalité de Markov.
Donc $\Pp{\abs{X_n-X} \geq \varepsilon }  \to 0$.
Si $\varepsilon \geq 1$, alors $\Pp{\abs{X_n-X} \geq \varepsilon} \leq \Pp{\abs{X_n-X} \geq 1}  \to 0$.
Cela montre la convergence en probabilité.
%%
%\item On procède d'une manière proche de la démonstration de la complétude de $L^p$.
%Soit $(X_n)_{n\in\N}$ une suite de Cauchy pour $d$. Il existe une extractrice $\phi$ telle que
%\[
%\forall n \in \N, \Pp{\abs{X_{\phi(n+1)}-X_{\phi(n)}} \geq \frac{1}{2^n}} \leq \frac{1}{2^n}.
%\]
%Alors, par Borel-Cantelli, p.s. il existe seulement un nombre fini de $n$ tels que $\abs{X_{\phi(n+1)}-X_{\phi(n)}}\geq \frac{1}{2^n}$ donc p.s. $\sum_{n\geq 0} \abs{X_{\phi(n+1)}-X_{\phi(n)}} < \infty$.
%Ainsi on peut définir $X = X_0 + \sum_{n\geq 0} X_{\phi(n+1)}-X_{\phi(n)}$ et $X$ est limite de ses sommes partielles p.s., c'est-à-dire $X_{\phi(n)} \to X$ p.s.
%En particulier, $X_{\phi(n)} \to X$ en probabilité.
\end{enumerate}
\end{comment}


%%%%%%
\separationexos
%%%%%%


\begin{exo} 
	Soit $(\Omega, \cA, \mathbb{P})$ un espace de probabilité. 
	On suppose que $\Omega$ est dénombrable et que $\cA = \cP(\Omega)$. 
	Montrer que les convergences ``presque-sûre'' et ``en probabilité'' sont équivalentes pour des v.a. réelles définies sur cet espace.
\end{exo}


\begin{comment}
Il suffit de montrer que la convergence en probabilité implique la convergence p.s. (car la réciproque est toujours vraie).
Soit $X,X_1,X_2,\dots$ des v.a. réelles telles que  
\[ 
X_n \xrightarrow[n\to\infty]{\P} X.
\] 
Soit $A = \{ \omega \in \Omega : \P(\{\omega\})>0 \}$.
On a 
\begin{align*}
\P(A^c) & = \P \Biggl( \bigcup_{\omega \in A^c} \{\omega\} \Biggr) \\
& = \sum_{\omega \in A^c} \P(\{\omega\}) & \text{(car $A^c$ est dénombrable)}\\
& = 0 & \text{(car $\P(\{\omega\}) = 0$ pour $\omega \in A^c$)}.
\end{align*}
Donc $\P(A)=1$.
Soit $\omega \in A$.
On va montrer que
\[ 
X_n(\omega) \xrightarrow[n\to\infty]{} X(\omega).
\] 
Soit $\varepsilon > 0$. On sait que 
\[
	\Pp{ \abs{X_n-X} \geq \varepsilon } \xrightarrow[n\to\infty]{} 0.
\]
Or $\mathbb{P}(\{\omega\})>0$ donc, à partir d'un certain rang, $\Pp{ \abs{X_n-X} \geq \varepsilon } < \mathbb{P}(\{\omega\})$ et donc $\omega \notin \{\abs{X_n-X} \geq \varepsilon \}$.
On a donc montré que, pour tout $\varepsilon >0$, à partir d'un certain rang, 
\[
	\abs{X_n(\omega)-X(\omega)} < \varepsilon.
\]
Cela montre que $X_n(\omega) \to X(\omega)$ et conclut la preuve (car c'est vrai pour tout $\omega$ dans $A$ qui est de probabilité 1).
\end{comment}

\end{document}