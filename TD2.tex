\documentclass[a4paper,11pt]{article}
\usepackage[utf8]{inputenc}
\usepackage[T1]{fontenc}
\usepackage{lmodern}

\usepackage{amsthm,amsmath,amsfonts,amssymb,bbm,mathrsfs,stmaryrd}
\usepackage{mathtools}
\usepackage{enumitem}
\usepackage{url}
\usepackage{dsfont}
\usepackage{appendix}
\usepackage{amsthm}
\usepackage[dvipsnames,svgnames]{xcolor}
\usepackage{graphicx}

\usepackage{fancyhdr,lastpage,titlesec,verbatim,ifthen}

\usepackage[colorlinks=true, linkcolor=black, urlcolor=black, citecolor=black]{hyperref}

\usepackage[french]{babel}

\usepackage{caption,tikz,subfigure}

\usepackage[top=2cm, bottom=2cm, left=2cm, right=2cm]{geometry}

%%%%%%%%%%%%%%%%%%%%%%%%%%%%%%%%%%%%%%%%%%%%%%%%%%%%%%%%%%%%%%%%%%%%%%%

%%%%%%%% Taille de la legende des images %%%%%%%%%%%%%%%%%%%%%%%%%%%%%%%
\renewcommand{\captionfont}{\footnotesize}
\renewcommand{\captionlabelfont}{\footnotesize}

%%%%%%%% Numeration des enumerates en romain et chgt de l'espace %%%%%%%
\setitemize[1]{label=$\rhd$, font=\color{NavyBlue},leftmargin=0.8cm}
\setenumerate[1]{font=\color{NavyBlue},leftmargin=0.8cm}
\setenumerate[2]{font=\color{NavyBlue},leftmargin=0.47cm}
%\setlist[enumerate,1]{label=(\roman*), font = \normalfont,itemsep=4pt,topsep=4pt} 
%\setlist[itemize,1]{label=\textbullet, font = \normalfont,itemsep=4pt,topsep=4pt} 

%%%%%%%% Pas d'espacement supplementaire avant \left et apres \right %%%
%%%%%%%% Note : pour les \Big(, utiliser \Bigl( \Bigr) %%%%%%%%%%%%%%%%%
\let\originalleft\left
\let\originalright\right
\renewcommand{\left}{\mathopen{}\mathclose\bgroup\originalleft}
\renewcommand{\right}{\aftergroup\egroup\originalright}

%%%%%%%%%%%%%%%%%%%%%%%%%%%%%%%%%%%%%%%%%%%%%%%%%%%%%%%%%%%%%%%%%%%%%%%

\newcommand{\N}{\mathbb{N}}
\newcommand{\Z}{\mathbb{Z}}
\newcommand{\Q}{\mathbb{Q}}
\newcommand{\R}{\mathbb{R}}
\newcommand{\C}{\mathbb{C}}
\newcommand{\T}{\mathbb{T}}
\renewcommand{\P}{\mathbb{P}}
\newcommand{\E}{\mathbb{E}}

\newcommand{\1}{\mathbbm{1}}

\newcommand{\cA}{\mathcal{A}}
\newcommand{\cB}{\mathcal{B}}
\newcommand{\cC}{\mathcal{C}}
\newcommand{\cE}{\mathcal{E}}
\newcommand{\cF}{\mathcal{F}}
\newcommand{\cG}{\mathcal{G}}
\newcommand{\cH}{\mathcal{H}}
\newcommand{\cI}{\mathcal{I}}
\newcommand{\cJ}{\mathcal{J}}
\newcommand{\cL}{\mathcal{L}}
\newcommand{\cM}{\mathcal{M}}
\newcommand{\cN}{\mathcal{N}}
\newcommand{\cP}{\mathcal{P}}
\newcommand{\cT}{\mathcal{T}}
\newcommand{\cU}{\mathcal{U}}

\newcommand{\Ec}[1]{\mathbb{E} \left[#1\right]}
\newcommand{\Pp}[1]{\mathbb{P} \left(#1\right)}
\newcommand{\Ppsq}[2]{\mathbb{P} \left(#1\middle|#2\right)}

\newcommand{\ii}{\mathrm{i}}
\DeclareMathOperator{\re}{Re}
\DeclareMathOperator{\im}{Im}
\DeclareMathOperator{\Arg}{Arg}

\newcommand{\diff}{\mathop{}\mathopen{}\mathrm{d}}
\DeclareMathOperator{\Var}{Var}
\DeclareMathOperator{\Cov}{Cov}
\newcommand{\supp}{\mathrm{supp}}

\newcommand{\abs}[1]{\left\lvert#1\right\rvert}
\newcommand{\abso}[1]{\lvert#1\rvert}
\newcommand{\norme}[1]{\left\lVert#1\right\rVert}
\newcommand{\ps}[2]{\langle #1,#2 \rangle}

\newcommand{\petito}[1]{o\mathopen{}\left(#1\right)}
\newcommand{\grandO}[1]{O\mathopen{}\left(#1\right)}

\newcommand\relphantom[1]{\mathrel{\phantom{#1}}}

\newcommand{\NB}[1]{{\color{NavyBlue}#1}}
\newcommand{\DSB}[1]{{\color{DarkSlateBlue}#1}}
\newcommand{\emphb}[1]{\emph{{\color{DarkSlateBlue}#1}}}

%%%%%%%% Theorems styles %%%%%%%%%%%%%%%%%%%%%%%%%%%%%%%%%%%%%%%%%%%%%%
\theoremstyle{plain}
\newtheorem{theorem}{Theorem}[section]
\newtheorem{proposition}[theorem]{Proposition}
\newtheorem{lemma}[theorem]{Lemme}
\newtheorem{corollary}[theorem]{Corollaire}
\newtheorem{conjecture}[theorem]{Conjecture}
\newtheorem{definition}[theorem]{Définition}

\theoremstyle{definition}
\newtheorem{remark}[theorem]{Remarque}
\newtheorem{example}[theorem]{Exemple}
\newtheorem{question}[theorem]{Question}

%%%%%%%% Macros spéciales TD %%%%%%%%%%%%%%%%%%%%%%%%%%%%%%%%%%%%%%%%%%

%%%%%%%%%%%% Changer numérotation des pages %%%%%%%%%%%%%%%%%%%%%%%%%%%%
\pagestyle{fancy}
\cfoot{\thepage/\pageref{LastPage}} %%% numéroter page / total de pages
\renewcommand{\headrulewidth}{0pt} %%% empêcher qu'il y ait une ligne horizontale en haut
%%%%%%%%%%%% Ne pas numéroter les pages %%%%%%%%%%%%%%%%%
%\pagestyle{empty}

%%%%%%%%%%%% Supprimer les alineas %%%%%%%%%%%%%%%%%%%%%%%%%%%%%%%%%%%%%
\setlength{\parindent}{0cm} 

%%%%%%%%%%%% Exercice %%%%%%%%%%%%%%%%%%%%%%%%%%%%%%%%%% 
\newcounter{exo}
\newenvironment{exo}[1][vide]
{\refstepcounter{exo}
	{\noindent \textcolor{DarkSlateBlue}{\textbf{Exercice \theexo.}}}
	\ifthenelse{\equal{#1}{vide}}{}{\textcolor{DarkSlateBlue}{(#1)}}
}{}

%%%%%%%%%%%% Partie %%%%%%%%%%%%%%%%%%%%%%%%%%%%%%%%%%%%
\newcounter{partie}
\newcommand\partie[1]{
	\stepcounter{partie}%
	{\bigskip\large\textbf{\DSB{\thepartie.~#1}}\bigskip}
	}

%%%%%%%%%%%% Separateur entre les exos %%%%%%%%%%%%%%%%%
\newcommand{\separationexos}{
	\bigskip
%	{\centering\hfill\DSB{\rule{0.4\linewidth}{1.2pt}}\hfill}\medskip
	}

%%%%%%%%%%%% Corrige %%%%%%%%%%%%%%%%%%%%%%%%%%%%%%%%%%% 
\renewenvironment{comment}{\medskip\noindent \textcolor{BrickRed}{\textbf{Corrigé.}}}{}

%%%%%%%%%%%% Titre %%%%%%%%%%%%%%%%%%%%%%%%%%%%%%%%%%%%%%
\newcommand\titre[1]{\ \vspace{-1cm}
	
	\DSB{\rule{\linewidth}{1.2pt}}
	{\small Probabilités et statistiques continues avancées}
	\hfill {\small Université Paul Sabatier}
	
	{\small KMAXPP03}
	\hfill {\small Licence 3, Printemps 2023}\medskip
	\begin{center}
		{\Large\textbf{\DSB{#1}}}\vspace{-.2cm}
	\end{center}
	\DSB{\rule{\linewidth}{1.2pt}}\medskip
}

%%%%%%%%%%%%%%%%%%%%%%%%%%%%%%%%%%%%%%%%%%%%%%%%%%%%%%%%%%%%%%%%%%%%%%%
\begin{document}
%%%%%%%%%%%%%%%%%%%%%%%%%%%%%%%%%%%%%%%%%%%%%%%%%%%%%%%%%%%%%%%%%%%%%%%

\titre{TD 2 -- Variables aléatoires, lois et espérance}


%%%%%%%%%%%%%%%%%%%%%%%%%%%%%%%%%%%%%%%%%%%%%%%%%%%
\partie{Variables aléatoires}
%%%%%%%%%%%%%%%%%%%%%%%%%%%%%%%%%%%%%%%%%%%%%%%%%%%

\begin{exo} Soient $X$, $Y$ et $Z$ des v.a. réelles définies sur un espace de probabilité $(\Omega,\cA,\P)$.
	\begin{enumerate}
		\item On suppose que $X=Y$ p.s. Montrer que $X$ et $Y$ ont la m\^eme loi. Que dire de la réciproque ?
		%%
		\item On suppose que $X$ et $Y$ ont la m\^eme loi. 
		\begin{enumerate}
			\item Soit $f:\R\to\R$ une fonction mesurable. Montrer que les variables aléatoires $f(X)$ et $f(Y)$ ont la m\^eme loi.
			%%
			\item Montrer que les variables aléatoires $XZ$ et $YZ$ n'ont pas nécessairement la m\^eme loi. 
		\end{enumerate}
	\end{enumerate}
\end{exo}

\begin{comment}
\begin{enumerate} 
\item Si $X=Y$ p.s.~alors, pour toute fonction mesurable $f \colon \R\to\R_+$, $f(X)=f(Y)$ p.s. et donc $\Ec{f(X)}=\Ec{f(Y)}$, ce qui montre que $X$ et $Y$ ont la m\^eme loi : en effet, pour tout $B \in \cB(\R)$, en prenant $f = \1_B$, on a
\[
	P_X(B) = \Ec{f(X)} = \Ec{f(Y)} = P_Y(B).
\]

La réciproque est fausse. 
Considérons une variable aléatoire $X$ une v.a. à valeurs dans $\{-1,1\}$ telle que $\P(X = 1) = \P(X=-1) = \frac{1}{2}$, i.e. de loi $\frac{1}{2} \delta_1 + \frac{1}{2} \delta_{-1}$.
On pose $Y = -X$. Alors $Y$ a la même loi que $X$, mais $X$ et $Y$ ne sont jamais égales (pour tout $\omega \in \Omega$, $X(\omega) \neq Y(\omega)$).
%%
\item 
\begin{enumerate}
\item Pour toute fonction mesurable $g:\R\to\R_+$, la fonction $g\circ f$ est mesurable. Comme $X$ et $Y$ ont la m\^eme loi, on a 
\[
\Ec{g(f(X))} = \Ec{(g\circ f)(X)} =\Ec{(g\circ f)(Y)} = \Ec{g(f(Y))}.
\] 
Comme c'est vrai pour toute fonction mesurable $g:\R\to\R_+$, cela montre que $f(X)$ et $f(Y)$ ont la m\^eme loi (pour la même raison qu'en question 1.). 
%%
\item On reprend les variables $X$ et $Y$ de la question 1. 
Soit $Z=X$. 
Alors $XZ=X^2$ et $YZ=-X^2$. 
La loi de $X^2$ est $\delta_1$ et la loi de $-X^2$ est $\delta_{-1}$ donc $XZ$ et $YZ$ n'ont pas la m\^eme loi.
\end{enumerate}
\end{enumerate} 
\end{comment}


%%%%%%%%%%%%%%%%%%%%%%%%%%%%%%%%%%%%%%%%%%%%%%%%%%%
\partie{Formule de transfert}
%%%%%%%%%%%%%%%%%%%%%%%%%%%%%%%%%%%%%%%%%%%%%%%%%%%


\begin{exo}
	Soit $X$ une v.a. de loi de Poisson de paramètre $\lambda > 0$, i.e. de loi $\sum_{k=0}^\infty e^{-\lambda} \frac{\lambda^k}{k!} \delta_k$.
	Pour tout $t \in \R$, calculer $\E[e^{-tX}]$.
%	\begin{enumerate}
%		\item Calculer $\E[X]$.
%		\item Pour tout $t \in \R$, calculer $\E[e^{-tX}]$.
%	\end{enumerate}
\end{exo}

\begin{comment}
On trouve $\E[e^{-tX}] = e^{\lambda (e^{-t}-1)}$.
\end{comment}

%%%%%%
\separationexos
%%%%%%

\begin{exo}
	Soit $X$ une v.a. réelle de densité $p_X \colon x \mapsto c(x\1_{[0,1]}(x) + (2-x)\1_{]1,2]}(x))$ par rapport à la mesure de Lebesgue. 
	Déterminer la valeur de la constante $c$. 
	Calculer $\mathbb{E}[X^p]$ pour tout $p \geq 0$.
\end{exo}

\begin{comment}
On obtient $c$ en utilisant que l'intégrale de $p_X$ doit être égale à 1 : on a
\[
	\int_\R p_X(x) \diff x 
	= c \left( \int_0^1 x \diff x + \int_1^2 (2-x) \diff x \right)
	= c \left( \frac{1}{2} + \frac{1}{2} \right)
	= c
\]
et donc $c = 1$.

Soit $p \geq 0$. Par le théorème de transfert,
\begin{align*}
	\E[X^p] & = \int_\R x^p p_X(x) \diff x
	= \int_0^1 x^{p+1} \diff x + \int_1^2 (2x^p-x^{p+1}) \diff x \\
	& = \left[ \frac{x^{p+2}}{p+2} \right]_0^1 + \left[ \frac{2x^{p+1}}{p+1} - \frac{x^{p+2}}{p+2} \right]_1^2
	= \frac{1}{p+2} + \frac{2\cdot 2^{p+1}}{p+1} - \frac{2^{p+2}}{p+2} - \frac{2}{p+1} + \frac{1}{p+2} \\
	& = \frac{2^{p+2}-2}{(p+1)(p+2)}.
\end{align*}
\end{comment}


%%%%%%
\separationexos
%%%%%%

\begin{exo}
	Rappelons que $\int_\R e^{-x^2/2} \diff x = \sqrt{2\pi}$, voir l'exercice \ref{exo:gaussienne} pour se rappeler la démonstration.
	Soit $X$ une v.a. de loi gaussienne standard, i.e. de loi $\frac{1}{\sqrt{2\pi}} e^{-x^2/2} \diff x$. 
	Calculer $\E[X]$ et $\E[X^2]$.
\end{exo}

\begin{comment}
Notons que $x \in \R \mapsto x$ est intégrable par rapport à $P_X$ (car $\abs{x} e^{-x^2/2}$ décroit plus vite que $e^{-\abs{x}}$ quand $x \to \infty$), donc, par le théorème de transfert, 
\[
\E[X] = \int_\R x \cdot \frac{1}{\sqrt{2\pi}} e^{-x^2/2} \diff x = 0,
\]
car c'est une fonction impaire.

Comme $x^2 \geq 0$, par le théorème de transfert, 
\begin{align*}
\E[X^2] 
& = \int_\R x^2 \cdot \frac{1}{\sqrt{2\pi}} e^{-x^2/2} \diff x \\
& = - \frac{1}{\sqrt{2\pi}} \int_\R x \cdot \left( \frac{\diff}{\diff x} e^{-x^2/2} \right) \diff x \\
& = - \frac{1}{\sqrt{2\pi}} 
\left( \left[ x e^{-x^2/2}\right]_{-\infty}^\infty - \int_\R e^{-x^2/2} \diff x \right) & (\text{intégration par parties})\\
& = - \frac{1}{\sqrt{2\pi}} \left( 0 - \sqrt{2\pi} \right) = 1.
\end{align*}
\end{comment}


%%%%%%%%%%%%%%%%%%%%%%%%%%%%%%%%%%%%%%%%%%%%%%%%%%%
\partie{Calcul de loi}
%%%%%%%%%%%%%%%%%%%%%%%%%%%%%%%%%%%%%%%%%%%%%%%%%%%

\emphb{Méthode.} Pour déterminer la loi d'une v.a. $Y$, on pourra utiliser la méthode suivante, présentée plus en détails en Section 1.3.3 des notes de cours : considérer une fonction $f \colon \R \to \R_+$ mesurable quelconque et écrire $\E[f(Y)]$ sous la forme $\int_\R f(y) \diff \mu(y)$. Alors on peut en conclure que $P_Y = \mu$.

%%%%%%
\separationexos
%%%%%%

\begin{exo} 
	Soit $X$ une v.a. de loi gaussienne standard, i.e. de loi $\frac{1}{\sqrt{2\pi}} e^{-x^2/2} \diff x$. 
	\begin{enumerate}
		\item Montrer que $\mathbb{P}(X=0)=0$. 
		%%
		\item On pose alors $Y = 1/X^2$ (qui est bien définie p.s.). Déterminer la loi de $Y$.
		%%
		\item Pour $m \in \R$ et $\sigma > 0$, on pose $Z = \sigma X + m$. Déterminer la loi de $Z$.
	\end{enumerate}
\end{exo}

\begin{comment}
\begin{enumerate}
\item On a $\mathbb{P}(X=0)= P_X(\{0\}) = 0$ car $P_X$ est à densité par rapport à la mesure de Lebesgue. 
%%
\item On utilise la méthode ci-dessus. Soit $f \colon \R \to\R_+$ une fonction mesurable. On a, par théorème de transfert,
\[
\Ec{f(Y)}
= \Ec{f \left( \frac{1}{X^2} \right)}
= \frac{1}{\sqrt{2\pi}} \int_\R f\left(\frac{1}{x^2}\right) e^{-x^2/2} \diff x
= \frac{2}{\sqrt{2\pi}} \int_0^\infty f\left(\frac{1}{x^2}\right) e^{-x^2/2} \diff x,
\] 
où l'on utilise la parité de la fonction intégrée dans la dernière égalité.
Et $x\in\R_+^*\mapsto x^{-2}\in\R_+^*$ est un $\cC^1$ difféomorphisme, donc avec le changement de variables $y=x^{-2}$, on a 
\[
\Ec{f(Y)}
= \frac{2}{\sqrt{2\pi}} \int_0^\infty f(y) e^{-1/(2y)} \frac{1}{2y^{3/2}} \diff y.
\] 
Donc la loi de $Y$ est 
\[
P_Y(\diff y) = \frac{1}{\sqrt{2\pi y^3}}e^{-1/(2y)}\1_{[0,\infty[}(y) \diff y.
\]
%%
\item Soit $f \colon \R \to\R_+$ une fonction mesurable. On a, par théorème de transfert,
\[
\Ec{f(Z)}
= \Ec{f(\sigma X + m)}
= \frac{1}{\sqrt{2\pi}} \int_\R f(\sigma x + m) e^{-x^2/2} \diff x
= \frac{1}{\sqrt{2\pi}} \int_\R f(y) e^{-(y-m)^2/(2\sigma^2)} \frac{1}{\sigma}\diff y,
\] 
avec le changement de variable $y= \sigma x + m$.
Donc, la loi de $Y$ est 
\[
P_Y(\diff y) = \frac{1}{\sigma \sqrt{2\pi}} e^{-(y-m)^2/(2\sigma^2)} \diff y.
\]
\end{enumerate}
\end{comment}


%%%%%%
\separationexos
%%%%%%

\begin{exo}
	Soit $X$ une v.a. de loi exponentielle de paramètre $1$, i.e. de loi 
	$e^{-x} \1_{[0,\infty)}(x) \diff x$.
	On pose $Y=\min (X,\frac{1}{X})$ qui est bien définie p.s. car $\mathbb{P}(X=0)=0$. 
	Déterminer la loi de $Y$.
\end{exo}

\begin{comment}
Soit $f \colon \R \to\R_+$ une fonction mesurable. On a, par théorème de transfert,
\[
\Ec{f(Y)}
= \Ec{f \left( \min \left( X,\frac{1}{X} \right)\right)}
= \int_0^\infty f\left( \min \left( x,\frac{1}{x} \right)\right) e^{-x} \diff x
\] 
On sépare alors les cas $x \leq 1$ et $x > 1$ :
\[
\Ec{f(Y)}
= \int_0^1 f(x) e^{-x} \diff x
+ \int_1^\infty f\left( \frac{1}{x} \right) e^{-x} \diff x
= \int_0^1 f(x) e^{-x} \diff x
+ \int_0^1 f(y) e^{-1/y} \frac{1}{y^2} \diff y,
\] 
en utilisant le changement de variables $y=1/x$ dans la 2ème intégrale.
On a donc
\[
\Ec{f(Y)}
= \int_0^1 f(y) \left( e^{-y} + e^{-1/y} \frac{1}{y^2} \right) \diff y.
\] 
Donc la loi de $Y$ est 
\[
P_Y(\diff y) = \left( e^{-y} + e^{-1/y} \frac{1}{y^2} \right) \1_{[0,\infty[}(y) \diff y.
\]
\end{comment}

%%%%%%
\separationexos
%%%%%%

\begin{exo}[Transformation de loi discrète]
	Soit $X$ une v.a. réelle de loi discrète $\sum_{i\in \cI} p_i \delta_{x_i}$, avec $\cI \subset \N$, %$p_i > 0$ et $x_i \in \R$ pour $i \in \cI$.
	$(p_i)_{i\in\cI} \in (\R_+^*)^\cI$ et $(x_i)_{i\in\cI} \in \R^\cI$.
	Soit $f \colon \R \to \R$ mesurable.
	Quelle est la loi de $f(X)$ ?
\end{exo}

\begin{comment}
    On pose $Y = f(X)$.
    
    \emph{Méthode 1.} Soit $g \colon \R \to\R_+$ une fonction mesurable. On a, par théorème de transfert,
    \[
    \Ec{g(Y)}
    = \Ec{g(f(X))}
    = \int_\R g(f(x)) \diff P_X(x)
    = \sum_{i\in \cI} p_i g(f(x_i))
    = \int_\R g(y) \diff \left( \sum_{i\in \cI} p_i \delta_{f(x_i)} \right)(y),
    \] 
    en réécrivant la somme comme une intégrale par rapport à une autre mesure discrète.
    Cela montre que
    \[
    P_Y = \sum_{i\in \cI} p_i \delta_{f(x_i)}.
    \]
    
    
    \emph{Méthode 2.} La v.a. $Y$ est à valeurs dans l'ensemble dénombrable $\{f(x_i), i\in\cI\}$, donc sa loi est discrète et est caractérisée par sa valeur sur les singletons. On calcule donc, pour tout $i \in \cI$,
    \[
    \Pp{Y = f(x_i)} = \Pp{f(X) = f(x_i)} = \Pp{ X \in \{ x_j : j\in\cI, f(x_j) = f(x_i) \}} = \sum_{j \in \cI \text{ tq } f(x_j) = f(x_i)} p_j
    \] 
    et cela montre que
    \[
    P_Y = \sum_{i\in \cI} p_i \delta_{f(x_i)},
    \]
    car ces mesure coïncident sur les singletons $\{f(x_i)\}$.
\end{comment}



%%%%%%%%%%%%%%%%%%%%%%%%%%%%%%%%%%%%%%%%%%%%%%%%%%%
\partie{Modélisation continue}
%%%%%%%%%%%%%%%%%%%%%%%%%%%%%%%%%%%%%%%%%%%%%%%%%%%



\begin{exo} 
	On considère une source lumineuse ponctuelle située au point $(-1,0)$ dans le plan. 
	On suppose que la source émet un rayon lumineux en direction de l'axe des ordonnées en faisant un angle aléatoire $\Theta$ avec l'axe des abscisses, où $\Theta$ est tiré uniformément sur $]-\pi/2,\pi/2[$. 
	Déterminer la loi de l'ordonnée du point d'impact du rayon avec l'axe des ordonnées.
\end{exo} 

\begin{comment}
On commence par faire un dessin.
\begin{figure}[h]
	\centering
	\begin{tikzpicture}[scale=1.4]
	\draw[->,>=latex] (-2.3,0) -- (2.5,0);
	\draw (-2,0.05) -- (-2,-0.05);
	\draw (2,0.05) -- (2,-0.05);
	\draw (-1,0.05) -- (-1,-0.05);
	\draw (1,0.05) -- (1,-0.05) node[below]{1};
	\draw[->,>=latex] (0,-2.3) -- (0,2.5);
	\draw (0.05,-2) -- (-0.05,-2);
	\draw (0.05,2) -- (-0.05,2);
	\draw (0.05,-1) -- (-0.05,-1);
	\draw (0.05,1) -- (-0.05,1);
	%%
	\draw (-1,0) -- (0,1.5);
	\draw (-.7,0) arc (0:57:.3);
	\draw (-.6,.25) node{$\Theta$};
	\draw[FireBrick,fill] (-1,0) circle (0.05);
	\draw[NavyBlue,fill] (0,1.5) circle (0.05);
	\draw[NavyBlue] (0,1.5) node[right]{$(0,Y)$};
	\end{tikzpicture}
\end{figure}

Soit $\Theta$ une v.a. de loi uniforme sur $]-\pi/2,\pi/2[$, i.e. de loi $\frac{1}{\pi} \1_{]-\pi/2,\pi/2[}(t) \diff t$.
Le point d'impact du rayon lumineux sur l'axe des ordonnées est la v.a. $Y=\tan(\Theta)$. 
L'objectif est de déterminer la loi de $Y$.
Soit $f\colon\R\to\R_+$ mesurable. On a alors
\[
	\Ec{f(Y)} 
	= \frac{1}{\pi}\int_{-\pi/2}^{\pi/2}f(\tan(t))\diff t=\frac{1}{\pi}\int_{-\infty}^{\infty}f(y)\frac{\diff y}{1+y^2},
\]
en utilisant le changement de variable $y = \tan(t)$ (car $t\in\,]-\pi/2,\pi/2[\,\mapsto\tan(t)\in\R$ est un $C^1$-difféomorphisme).
Donc la loi de $Y$ est $\frac{1}{\pi(1+y^2)} \diff y$, i.e. la loi de Cauchy.
\end{comment}

%%%%%%
\separationexos
%%%%%%

\begin{exo}[Paradoxe de Bertrand]
	On s'intéresse à la loi de la longueur d'une corde tirée ``au hasard'' sur un cercle de rayon $1$. 
	Dans chacun des cas suivants, formaliser et calculer cette loi puis en déduire la probabilité que la corde soit plus longue que $\sqrt{3}$, c'est-à-dire que la longueur d'un côté d'un triangle équilatéral inscrit.
	\begin{enumerate}
		\item On choisit les deux extrémités de la corde au hasard sur le cercle.
		%%
		\item On choisit le centre de la corde au hasard sur le disque unité.
		%%
		\item On choisit au hasard la direction du rayon orthogonal à la corde, puis le centre de la corde uniformément sur ce rayon.
	\end{enumerate}
\end{exo}


\begin{comment}
\begin{enumerate}
\item On identifie le cercle à $[0,2\pi[$, donc on prend ici $\Omega \coloneqq [0,2\pi[^2$ muni de la tribu $\cA \coloneqq \cB([0,2\pi[^2)$ et de la mesure de probabilité $\P(\diff \omega) = \frac{1}{4 \pi^2} \diff \theta \diff \theta'$, où on note $\omega = (\theta,\theta')$ pour $\omega \in \Omega$.
La longueur de la corde est alors (faire un dessin)
\[
X(\omega) = 2 \abs{ \sin \left( \frac{\theta - \theta'}{2} \right)}.
\]
Déterminons la loi de $X$. 
Soit $f \colon \R \to \R_+$ mesurable.
On a
\begin{align*}
\Ec{f(X)}
& = \int_0^{2\pi} \int_0^{2\pi} 
f \left( 2 \abs{ \sin \left( \frac{\theta - \theta'}{2} \right)} \right)
\frac{1}{4 \pi^2} \diff \theta \diff \theta' \\
& = \int_0^{2\pi} \diff v \int_{v-2\pi}^v \diff u 
f \left( 2 \abs{\sin \left( \frac{u}{2} \right)} \right) \frac{1}{4 \pi^2} 
\end{align*}
avec le changement de variable $\varphi \colon (\theta,\theta') \mapsto (\theta - \theta',\theta)$ de $]0,2\pi[^2 \to \{ (u,v) : v \in ]0,2\pi[, u \in ]v-2\pi,v[ \}$ et de jacobien $1$.
Or, on a, par $2 \pi$-périodicité de $u \mapsto \abs{\sin(u/2)}$,
\begin{align*}
\int_{v-2\pi}^v f \left( 2 \abs{\sin \left( \frac{u}{2} \right)} \right) \diff u
= \int_{-\pi}^{\pi} f \left( 2 \abs{\sin \left( \frac{u}{2} \right)} \right) \diff u
= 2 \int_0^{\pi} f \left( 2 \abs{\sin \left( \frac{u}{2} \right)} \right) \diff u,
\end{align*}
par parité de $u \mapsto \abs{\sin(u/2)}$. 
On obtient donc
\begin{align*}
\Ec{f(X)}
= \frac{1}{\pi} \int_0^{\pi} f \left( 2 \sin \left( \frac{u}{2} \right) \right) \diff u 
= \frac{1}{\pi} \int_0^2 f(x) \left( 1 - \frac{x^2}{4} \right)^{-1/2} \diff x,
\end{align*}
avec le changement de variable $u = 2 \arcsin(x/2)$.
La loi de $X$ est donc
\[
\frac{1}{\pi} \left( 1 - \frac{x^2}{4} \right)^{-1/2} \1_{x \in [0,2]} \diff x.
\]
On peut alors calculer
\[
	\P(X \geq \sqrt{3}) 
	= \int_{\sqrt{3}}^2 \frac{1}{\pi} \left( 1 - \frac{x^2}{4} \right)^{-1/2} \diff x
	= \frac{2}{\pi} \left[ \sin^{-1} \left( \frac{x}{2} \right) \right]_{\sqrt{3}}^2
	= \frac{2}{\pi} \left( \frac{\pi}{2} - \frac{\pi}{3} \right)
	= \frac{1}{3}.
\]
%%
\item On prend à présent $\Omega \coloneqq \{ (x,y) \in \R^2 : x^2 + y^2 \leq 1 \}$ le disque unité muni de la tribu $\cA \coloneqq \cB(\Omega)$ et de la mesure de probabilité $\P(\diff \omega) = \frac{1}{\pi} \diff x \diff y$, où on note $\omega = (x,y)$ pour $\omega \in \Omega$.
La corde n'est bien définie que pour $\omega \neq (0,0)$ mais $\omega = (0,0)$ arrive avec probabilité nulle (et dans ce cas on peut malgré tout définir la longueur de la corde comme étant 2).
La longueur de la corde est (encore faire un dessin)
\[
X(\omega) = 2 \sqrt{1-x^2-y^2}.
\]
Déterminons la loi de $X$. 
Soit $f \colon \R \to \R_+$ mesurable.
On a, avec un passage en coordonnées polaires,
\begin{align*}
\Ec{f(X)}
& = \frac{1}{\pi} \int_{\R^2} 	f \left( 2 \sqrt{1-x^2-y^2} \right)
\1_{x^2 + y^2 \leq 1} \diff x \diff y \\
& = \frac{1}{\pi} \int_0^{2\pi} \int_0^\infty f \left( 2 \sqrt{1-r^2} \right)
\1_{r^2 \leq 1} r \diff r \diff \theta \\
& = 2 \int_0^1 f \left( 2 \sqrt{1-r^2} \right) r \diff r 
= \frac{1}{2} \int_0^2 f(u) u \diff u,
\end{align*}
avec le changement de variable $u = 2 \sqrt{1-r^2}$ (pour lequel on a $r \diff r = \frac{u}{4} \diff u$).
La loi de $X$ est donc
\[
\frac{1}{2} x \1_{x \in [0,2]} \diff x.
\]
On peut alors calculer
\[
\P(X \geq \sqrt{3}) 
= \int_{\sqrt{3}}^2 \frac{x}{2} \diff x
= \left[ \frac{x^2}{4} \right]_{\sqrt{3}}^2
= \frac{1}{4}.
\]
%%
\item On prend $\Omega \coloneqq [0,2\pi[ \times [0,1]$ muni de la tribu $\cA \coloneqq \cB(\Omega)$ et de la mesure de probabilité $\P(\diff \omega) = \frac{1}{2\pi} \diff \theta \diff r$, où on note $\omega = (\theta,r)$ pour $\omega \in \Omega$.
La longueur de la corde est (c'est une configuration analogue à la question 2.)
\[
X(\omega) = 2 \sqrt{1-r^2}.
\]
Déterminons la loi de $X$. 
Soit $f \colon \R \to \R_+$ mesurable.
On a
\begin{align*}
\Ec{f(X)}
& = \frac{1}{2\pi} \int_0^{2\pi} \int_0^1 f \left( 2 \sqrt{1-r^2} \right) \diff r \diff \theta  \\
& = \int_0^1 f \left( 2 \sqrt{1-r^2} \right) \diff r \\
& = \int_0^2 f(u) \frac{u}{4} \left( 1 - \frac{u^2}{4} \right)^{-1/2} \diff u,
\end{align*}
avec le changement de variable $u = 2 \sqrt{1-r^2}$.
La loi de $X$ est donc
\[
\frac{x}{4} \left( 1 - \frac{x^2}{4} \right)^{-1/2} \1_{x \in [0,2]} \diff x.
\]
On peut alors calculer
\[
\P(X \geq \sqrt{3}) 
= \int_{\sqrt{3}}^2 \frac{x}{4} \left(1 - \frac{x^2}{4}\right)^{-1/2} \diff x
= \left[ - \left( 1 - \frac{x^2}{4} \right)^{1/2} \right]_{\sqrt{3}}^2
= \frac{1}{2}.
\]
\end{enumerate}
%%
%\emph{Conclusion.} Bertrand s'intéressait à la probabilité que la corde soit plus longue que $\sqrt{3}$, c'est-à-dire la longueur d'un côté d'un triangle équilatéral inscrit.
%Sans formaliser les espaces de probabilité, il a calculé cette probabilité dans les 3 cas étudiés ici : elle est de $1/3$ dans le cas 1., $1/4$ dans le cas 2. et $1/2$ dans le cas 3.
%Il y voyait un paradoxe mais il n'y en a pas : il y a plusieurs manières de ``tirer une corde au hasard sur le cercle'', qui se précisent en définissant l'espace de probabilité sur lequel on travaille et il est normal que selon l'espace de probabilité, le résultat diffère.
\end{comment}


%%%%%%%%%%%%%%%%%%%%%%%%%%%%%%%%%%%%%%%%%%%%%%%%%%%
\partie{Compléments}
%%%%%%%%%%%%%%%%%%%%%%%%%%%%%%%%%%%%%%%%%%%%%%%%%%%


\begin{exo}
	Soient $X$ et $Y$ deux v.a. réelles définies sur $(\Omega, \cA, \mathbb{P})$.
	\begin{enumerate}
		\item Supposons que, pour tout $t \in \R$, $\mathbb{P}(Y < t \leq X)=0$. 
		Montrer que $\mathbb{P}(Y < X)=0$.
		\item Supposons maintenant que $X$ et $Y$ ont m\^eme loi. Montrer
		que, si $X \geq Y$ p.s., alors $X=Y$ p.s.
		
		\emph{Indication}. Considérer $\Omega' = \{X \geq Y\}$ et remarquer que 
		\[
		\mathbb{P}(\{X \geq t\} \setminus \{Y \geq t\})=
		\mathbb{P}([\{X \geq t\} \cap \Omega'] \setminus [\{Y \geq t\}\cap \Omega']).
		\]
	\end{enumerate}
\end{exo}

%%%%%%
\separationexos
%%%%%%

\begin{exo}
	Soit $X$ une v.a. de loi de Cauchy, i.e. de loi $\frac{1}{\pi (1+x^2)} \diff x$.
	\begin{enumerate}
		\item Montrer que $\mathbb{P}(X=0)=0$ et $\mathbb{P}(X<0)= \frac{1}{2}$.
		\item On pose $Y=1/X$ (qui est bien définie p.s.). Déterminer la loi de $Y$.
		\item Que dire de $\mathbb{E}[Y]$ ?
	\end{enumerate}
\end{exo}

%%%%%%
\separationexos
%%%%%%

\begin{exo}
	Soit $X$ une v.a. de loi exponentielle de paramètre $1$, i.e. de loi 
	$e^{-x} \1_{[0,\infty)}(x) \diff x$. Déterminer la loi de $\lfloor X \rfloor$ (la partie entière de $X$).
\end{exo}


%%%%%%
\separationexos
%%%%%%

\begin{exo}[Quelques propriétés de la gaussienne] \label{exo:gaussienne}
	\begin{enumerate}
		\item On veut montrer que $\int_\R e^{-x^2/2} \diff x = \sqrt{2\pi}$. Pour cela, calculer l'intégrale 
		\[
		\int_{\R^2} e^{-(x^2+y^2)/2} \diff x \diff y
		\]
		à l'aide d'un changement de variables polaire. 
		En déduire la formule désirée.
		%%
		\item Soit $X$ une v.a. de loi gaussienne standard, i.e. de loi $\frac{1}{\sqrt{2\pi}} e^{-x^2/2} \diff x$. Montrer par récurrence que
		\[
		\forall n \in \N, \quad 
		\Ec{X^{2n}} = \frac{(2n)!}{2^n n!}
		\quad \text{et} \quad 
		\Ec{X^{2n+1}} = 0.
		\]
		%%
		\item Soit $X$ une v.a. de loi gaussienne standard. Montrer que, pour tout $x >0$,
		\[
		\frac{1}{\sqrt{2 \pi}} \left( \frac{1}{x} - \frac{1}{x^3} \right) e^{{-x^2}/{2}}
		\leq \P(X > x)
		\leq \frac{1}{x \sqrt{2 \pi}} e^{{-x^2}/{2}}.
		\]
		\emph{Indication.} On pourra vérifier que
		\[
		\frac{1}{x} e^{{-x^2}/{2}} 
		= \int_x^{\infty} e^{{-t^2}/{2}} \left( 1 + \frac{1}{t^2} \right) \diff t
		\qquad \text{et} \qquad 
		\left( \frac{1}{x} - \frac{1}{x^3} \right) e^{{-x^2}/{2}} 
		= \int_x^{\infty} e^{{-t^2}/{2}} \left( 1 - \frac{3}{t^4} \right) \diff t.
		\]
	\end{enumerate}
\end{exo}


%%%%%%
\separationexos
%%%%%%

\begin{exo}[Tribu engendrée par une variable aléatoire]
	Soit $(\Omega,\cA,\P)$ un espace de probabilités et $X$ une variable aléatoire définie sur $(\Omega,\cA,\P)$ à valeurs dans $(\R,\cB(\R))$.
	On note
	\[
	\sigma(X) \coloneqq \{ X^{-1}(B) : B\in\cB(\R) \}.
	\]
	%%
	\begin{enumerate}
		\item Montrer que $\sigma(X)$ est la plus petite tribu $\cA$ sur $\Omega$ telle que $X$ soit mesurable de $(\Omega,\cA)$ dans $(\R,\cB(\R))$.
		La tribu $\sigma(X)$ est appelée la \emph{tribu engendrée par la 
			variable aléatoire~$X$}.
		%%
		\item Soit $Y$ une v.a. qui soit $\sigma(X)$-mesurable, i.e. $Y$ est une application mesurable de $(\Omega,\sigma(X))$ dans $(\R,\cB(\R))$.
		Montrer qu'il existe une fonction $f \colon \R\to\R$ mesurable telle que $Y=f(X)$.
		
		\emph{Indication.} On commencera par le cas où $Y = \1_A$ avec $A \in \sigma(X)$, puis le cas où $Y$ est étagée. Finalement, pour $Y$ générale, on utilisera qu'elle est limite de v.a. $\sigma(X)$-mesurables étagées.
		%%
		\item Expliciter $\sigma(X)$ et la loi de $X$
		dans les cas suivants ($\lambda$ est la mesure de Lebesgue) :
		\begin{enumerate}
			\item $(\Omega,\cA,\P) \coloneqq ([0,1],\cB([0,1]),\lambda)$ et
			$X(\omega) \coloneqq 2 \omega \1_{[0,1/2]}(\omega) + \1_{]1/2,1]}(\omega)$ pour $\omega\in [0,1]$.
			%%
			\item $X \coloneqq a \1_A + b \1_B$, où $A,B \in \cA$ et $a,b \in \R^*$.
			%%
			\item $(\Omega,\cA,\P) \coloneqq ([-1,1],\cB([-1,1]),\lambda/2)$ et
			$X(\omega) \coloneqq \omega^2$ pour $\omega\in [0,1]$.
		\end{enumerate}
	\end{enumerate}
\end{exo}

\begin{comment}
\begin{enumerate}
\item La famille $\sigma(X)$ est une tribu : c'est la tribu réciproque de $\cB(\R)$ par $X$.

Il est clair que $\sigma(X)$ rend $X$ mesurable (car  pour tout $B \in \cB(\R)$, on a bien $X^{-1}(B) \in \sigma(X)$). 
D'autre part, toute tribu rendant $X$ mesurable contient $\sigma(X)$, car elle doit contenir les ensembles de la forme $X^{-1}(B)$ pour $B \in \cB(\R)$. 
Donc $\sigma(X)$ est bien la plus petite tribu sur $\Omega$ rendant $X$ mesurable.
%%
\item La fonction $Y$ est mesurable de $(\Omega,\sigma(X))$ dans $(\R,\cB(\R))$, donc il existe $(Y_n)_{n\in\N}$ une suite de fonctions étagées (pour $\sigma(X)$) telles que $Y = \lim_{n\to\infty} Y_n$.
Pour $n \in\N$, $Y_n$ est de la forme
\[
Y_n = \sum_{i=1}^{k_n} c_{i,n} \1_{X^{-1}(B_{i,n})},
\]
avec $c_{i,n} \in \R$ et $B_{i,n} \in \cB(\R)$.
Alors, on remarque que $Y_n = f_n \circ X$, où l'on a posé
\[
f_n \coloneqq \sum_{i=1}^{k_n} c_{i,n} \1_{B_{i,n}},
\]
qui est bien une fonction mesurable de $\R \to \R$.
On pose alors 
\[
f(x) \coloneqq 
\left\{ \begin{array}{l}  
\lim_{n\to\infty} f_n(x) \text{ si la limite existe dans } \R, \\ 
0 \mbox{ sinon}. \end{array} 
\right.
\]
La fonction $f$ est mesurable car elle s'écrit
\[
f(x) 
= \1_{-\infty< \liminf_{n\to\infty} f_n (x) = \limsup_{n\to\infty} f_n (x)< \infty}
\limsup_{n\to\infty} f_n (x)
\]
et $\liminf_{n\to\infty} f_n$ et $\limsup_{n\to\infty} f_n$ sont mesurables.
%%
En outre, pour tout $\omega \in \Omega$, on a
\[
f_n (X(\omega)) = Y_n(\omega) \xrightarrow[n\to\infty]{} Y(\omega),
\]
donc $Y= f \circ X$.
%%
\item 
\begin{enumerate}
\item Soit $A \in \cB([0,1])$. On a $A = X^{-1}(\frac{1}{2}A) \in \sigma(X)$, car $\frac{1}{2}A \subset [0,\frac{1}{2}]$.
Donc $\cB([0,1]) \subset \sigma(X)$ et ainsi $\cB([0,1]) = \sigma(X)$ (l'inclusion réciproque revient exactement à dire que $X$ est mesurable de $([0,1],\cB([0,1]),\lambda) \to (\R,\cB(\R),\lambda)$).
La loi de $X$ est $\frac{1}{2} \lambda + \frac{1}{2} \delta_1$.
%%
\item \emph{Cas 1 : $a \neq b$ et $a + b \neq 0$.} Alors $X$ peut prendre uniquement les 4 valeurs distinctes suivantes : $0$, $a$, $b$ et $a+b$.
Comme $\cB(\R)$ est engendrée par $\cC \coloneqq \{ \{0\}, \{a\}, \{b\}, \{a+b\}\} \cup \cB(\R \setminus \{0,a,b,a+b\})$, $\sigma(X)$ est engendrée par les images réciproques d'éléments de $\cC$.
Comme les éléments de $\cB(\R \setminus \{0,a,b,a+b\})$ ont pour image réciproque $\varnothing$ et $0$, $a$, $b$ et $a+b$ ont pour images réciproques $A^c\cap B^c$, $A \cap B^c$, $A^c \cap B$ et $A \cap B$, on en déduit que
\[
\sigma (X) = \{ \varnothing, A, B, A^c,B^c, A\cap B, A\cup B, A^c\cap B, A^c\cup B, A\cap B^c, A\cup B^c, A^c\cap B^c, A^c\cup B^c, A\Delta B, (A \Delta B)^c, \Omega \},
\]
où $A \Delta B \coloneqq A\cup B \setminus A\cap B$ est la différence symétrique entre $A$ et $B$ ($\sigma(X)$ contient les $2^4$ ensembles que l'on peut former à partir de la partition à 4 éléments : $A^c\cap B^c$, $A \cap B^c$, $A^c \cap B$ et $A \cap B$).
La loi de $X$ est
\[
\Pp{A^c\cap B^c} \delta_0
+ \Pp{A \cap B^c} \delta_a 
+ \Pp{A^c \cap B} \delta_b
+ \Pp{A \cap B} \delta_{a+b}.
\]

\emph{Cas 2 : $a = b$.} Alors, on a $X = a 1_{A \Delta B} + 2a 1_{A\cap B}$.
Comme précédemment, $\sigma(X)$ est engendrée par $A^c\cap B^c$, $A \Delta B$, $A \cap B$, donc on obtient
\[
\sigma (X) = \{ \varnothing, A^c\cap B^c, A \Delta B, A \cap B, A\cup B, A^c\cup B^c, (A \Delta B)^c, \Omega \}.
\]
La loi de $X$ est
\[
\Pp{A^c\cap B^c} \delta_0
+ \Pp{A \Delta B} \delta_a 
+ \Pp{A \cap B} \delta_{2a}.
\]

\emph{Cas 3 : $a + b = 0$.} Alors, on a $X$ ne peut prendre que les 3 valeurs suivantes : $0$, $a$ et $-a$, donc $\sigma(X)$ est engendrée par $(A \Delta B)^c$, $A \cap B^c$ et $A^c \cap B$, donc on obtient
\[
\sigma (X) = \{ \varnothing, (A \Delta B)^c, A \cap B^c, A^c \cap B, A \Delta B, A^c \cup B, A \cap B^b, \Omega \}.
\]
La loi de $X$ est
\[
\Pp{(A \Delta B)^c} \delta_0
+ \Pp{A \cap B^c} \delta_a 
+ \Pp{A^c \cap B} \delta_{-a}.
\]
%%
\item Pour $A \in \cB([-1,1])$, on a $X^{-1}(A) = A \cup (-A)$, donc
\[
\sigma(X) 
= \{A \cup (-A) : A \in \cB([-1,1])\},
\]
qui est aussi l'ensemble des boréliens de $[-1,1]$ symétriques par rapport à 0.
Pour $f \colon \R \to \R_+$ mesurables, on a
\[
\Ec{f(X)} 
= \int_\Omega f(X(\omega)) \diff \P(\omega)
= \int_{-1}^1 f(\omega^2) \frac{\diff \omega}{2}
= \int_0^1 f(\omega^2) \diff \omega
= \int_0^1 f(x) \frac{\diff x}{2 \sqrt{x}}
\]
donc $X$ a pour loi $(2\sqrt{x})^{-1} \1_{x \in ]0,1]} \diff x$.
\end{enumerate}
\end{enumerate}
\end{comment}



\end{document}
